% \chapter{Portal público}
% \section{Perfil de usuario}
% \section{Encuesta de aversión al riesgo}
% \section{Ahorro precaucional}
% \section{Sugerencia de apuestas}
% \section{Pagos en línea}
% \section{Power ranking}
%
\chapter{Conclusiones}

En esta tesis se describió el sistema de recomendación de apuestas personalizadas para las ligas europeas de fútbol: \textbf{``Egobets''}. En conclusión, el sistema se puede pensar como un compendio de técnicas, teoremas y resultados ya conocidos en las ramas de optimización, estadística, finanzas, probabilidad y economía; aplicados en la creación de un sistema computacional robusto.

De este trabajo se puede concluir lo siguiente:



\begin{itemize}


	\item Existe más de un modelo que permita predecir los resultados de los partidos de las ligas europeas de fútbol. Más aún, el modelo de Maher (1982)\cite{maher1982modelling} proporciona predicciones lo suficientemente confiables como para considerar su uso en una estrategia de apuestas de hoy en día.

	\item Conociendo las probabilidades de los partidos, aún siendo estimadas, es posible encontrar una estrategia de apuestas que genere rendimientos positivos. Por ejemplo, Koopman en su artículo \cite{koopman2013dynamic} expone un ejemplo donde, mediante una estrategia de apuestas simple, por cada $75$ unidades apostadas recibe $25$ unidades de ganancia.

	\item La aversión al riesgo de un apostador se puede modelar con funciones de utilidad, esto permite que las recomendaciones de apuestas sean personalizadas. 
	
	\item Considerando que las temporadas del fútbol tienen varias jornadas, se integra un sistema de reservas cuya función principal consiste en maximizar la tasa de crecimiento de las ganancias del jugador. Vancura \cite{vancura2000finding} muestra varias estrategias de reservas que buscan conseguir estos resultados.

\item En el sistema de Egobets.com se provee de una asesoría de apuestas integral para cada semana de la temporada donde se utilizan las predicciones de los resultados de los partidos, el perfil de riesgo del usuario y el sistema de reservas.

\item Los momios que publican las casas de apuestas se enfocan a predecir el mercado, por lo tanto se pueden mejorar las probabilidades propuestas.

\item  El porcentaje de aciertos en los pronósticos es del $70\%$ para la liga española, $61\%$ en la italiana, $52\%$ en la inglesa, $52\%$ en la alemana y $48\%$ en la francesa.

\item En los resultados observados, Egobets reporta un rendimiento desde el $29\%$ hasta el $83\%$ dependiendo del nivel de riesgo por apuesta del usuario.

\item El cómputo en la nube es esencial para los sistemas como Egobets, sus ventajas como escalabilidad permiten mantener costeables y funcionales los servicios.

\item El enfoque de un desarrollo al diseño y la usabilidad permiten que el usuario se enfoque en realizar únicamente lo que debe realizar.

\item MongoDB proporciona el beneficio de guardar información sin una estructura definida. Esto permite mucha mayor flexibilidad en los documentos que se persisten y la información que contienen\cite{puniaimplementing}.

\end{itemize}

% - Los juegos de azar de los casinos contemplan un margen para la casa que les garantiza ganancias a futuro.
% - Es claro que, más allá del azar, la habilidad y destreza de los equipos dominan los resultados de una temporada de fútbol.
% - Los momios que publican las casas de apuestas se enfocan a predecir el mercado, por lo tanto se pueden mejorar las probabilidades propuestas.
% - Existe más de un modelo que permita predecir los resultados de las ligas de fútbol, desde Maher \cite{maher1982modelling} estas predicciones han sido lo suficientemente confiables como para considerar su uso en apuestas.
% - Conociendo las probabilidades de los partidos, aún siendo estimadas, es posible encontrar una estrategia de apuestas que genere rendimientos positivos como demostró Koopman en su artículo \cite{koopman2013dynamic} mediante una simple estrategia de apuestas considerando valores esperados positivos.
%
%
% - Además de tener las probabilidades y la estrategia de apuestas, Vancura \cite{vancura2000finding} señala que el criterio de Kelly \cite{kelly1956new} maximiza asintóticamente la tasa de crecimiento de las ganancias, por lo que se integra esquema de reservas a la solución propuesta.
%
% - Con el fin de buscar una asesoría personalizada se propone un esquema de recomendaciones basadas en el perfil de riesgo del apostador.
%
% - En el sistema de Egobets.com se provee de una asesoría de apuestas integral para cada semana de la temporada donde se utilizan las predicciones de los resultados de los partidos, el perfil de riesgo del usuario y el sistema de reservas.
%
%
% - El porcentaje de aciertos en los pronósticos es del $70\%$ para la liga española, $61\%$ en la italiana, $52\%$ en la inglesa, $52\%$ en la alemana y $48\%$ en la francesa.
%
%
% - En los resultados observados, Egobets reporta un rendimiento desde el $29\%$ hasta el $83\%$ dependiendo del nivel de riesgo por apuesta del usuario.
%
% - Las ganancias dependen del acierto de las predicciones, sin embargo tener estas ganancias con la probabilidad de aciertos que se tiene verifica las sospechas de que se puede tener una estrategia redituable de apuestas a pesar del pronóstico de los partidos.
%
% - El cómputo en la nube es esencial para los sistemas como Egobets, sus ventajas como escalabilidad permiten mantener costeables y funcionales los servicios.
%
% - El enfoque de un desarrollo al diseño y la usabilidad permiten que el usuario se enfoque en realizar únicamente lo que debe realizar
%
% - MongoDB proporciona el beneficio de guardar información sin una estructura definida. Esto permite mucha mayor flexibilidad en los documentos que se persisten y la información que contienen\cite{puniaimplementing}. Mientras que las bases de datos relacionales resultan mucho más eficientes en consultas complejas. En general, un esquema combinado de ambos tipos de bases de datos pueden resultar en un sistema de información eficaz y completo \cite{faraj2014comparative}.
%
%
% Beneficios:
%
% - Egobets ofreca recomendaciones personalizadas de apuestas, cada persona es diferente y es tratada de forma única. Gracias a la encuesta se determina el perfil de riesgo del usuario y se le asesora de tal manera que obtenga ganancias y se sienta cómodo  al mismo tiempo.
%
%
% Acciones a futuro
% - Mejorar el sistema de predicción de resultados.
% - En un futuro este sistema podría modificarse para abarcar más mercado.
% 	- Otras ligas, por ejemplo la mexicana
% 	- Otros deportes, por ejemplo fútbol americano.
% - Incluso se varios de los autores hablan de que modelos parecidos a estos podrían servir para predecir elecciones.
%



\textbf{Acciones a futuro}
Un sistema tan complejo como éste genera muchas áreas de oportunidad, una de ellas sería implementar las nuevas investigaciones para mejorar el sistema de predicción de resultados, otro detalle sería encontrar las funciones de utilidad más prolíficas del sistema y centrar a los usuarios sobre esas. Adicionionalmente, se podría trabajar en hacer más autónomo el sistema para generar las recomendaciones. Finalmente, sobre este sistema se podría incluso podrían modificar la lógica y adaptar el modelo para abarcar más mercados, como:
\begin{itemize}

	\item Muchas más ligas de fútbol. La traslación del modelo a la liga de fútbol mexicana sería la primera opción y después una apertura a todas las ligas. Considerando ajustes necesarios como los aumentos en la varianza de los resultados, verificación de la significancia de las variables utilizadas en este modelo junto con las correcciones necesarias al modelo de predicción. De igual manera se tendría que buscar la integración de los momios de estas ligas y verificar las fuentes.

	\item Otros deportes. En una primera instancia se piensa utilizar este mismo modelo para fútbol americano, considerando que este deporte contiene mucho mayor información estadística de cada partido por lo que podría ayudar a tener una mejor predicción.

\end{itemize}

	
Para concluir, \textbf{``Egobets''} es un vivo ejemplo de como las Matemáticas y la Computación, conviviendo de manera simbiótica, aportan un sistema tangible que facilita a cualquier persona el acceso a información detallada, procesada y enfocada a mejorar su entendimiento acerca de un tema que, a la distancia, podría parecer complejo. Más aun, \textbf{``Egobets''} pone a la disposición de sus clientes un conjunto de poderosas herramientas computacionales basadas en estudiados modelos matemáticos que les proporcionan un servicio disponible a cualquier hora del día a través de un práctico portal Web. Finalmente, retomando la famosa frase de Bernardo de Chartres \cite{john1962metalogicon}: ``Somos enanos parados sobre hombros de gigantes''. Se invita a todos los lectores interesados, en usar este trabajo como excusa para desarrollar sistemas y plataformas innovadoras con usos prácticos de sus modelos matemáticos favoritos.
























