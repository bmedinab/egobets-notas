%%!TEX TS-program = pdflatex 
%%!TEX encoding = UTF-8 Unicode 
\documentclass[11pt,letterpaper,twoside,openright]{book}
\usepackage{geometry}                % See geometry.pdf to learn the layout options. There are lots.
\geometry{papersize={17cm,22.5cm}}                   % ...Tamaño tesis
%\geometry{landscape}                % Activate for for rotated page geometry
%\usepackage[parfill]{parskip}    % Activate to begin paragraphs with an empty line rather than an indent
\usepackage{graphicx}
\usepackage{amssymb}
\usepackage{amsmath}
\usepackage[utf8]{inputenc}
\usepackage[T1]{fontenc}
\usepackage[spanish,mexico]{babel}
\usepackage{epstopdf}
\usepackage{bbding}
\usepackage{color,colortbl}
\usepackage{lscape}
\usepackage{datetime}
\usepackage{mathrsfs}
\usepackage{verbatim}
\usepackage{enumerate}
\usepackage{cleveref}
\usepackage{caption}
%Cositas para hacer tablitas mas bonititas
\usepackage{booktabs}
\usepackage{multicol}
\usepackage{multirow}
%madres para hacer links de indices-itos
\usepackage[pdfpagelabels]{hyperref}
%a ver si sale el código bonito
\usepackage{listings}
\usepackage{color}

\definecolor{dkgreen}{rgb}{0,0.6,0}
\definecolor{gray}{rgb}{0.5,0.5,0.5}
\definecolor{mauve}{rgb}{0.58,0,0.82}

\lstset{frame=tb,
  language=HTML,
  aboveskip=3mm,
  belowskip=3mm,
  showstringspaces=false,
  columns=flexible,
  basicstyle={\small\ttfamily},
  numbers=none,
  numberstyle=\tiny\color{gray},
  keywordstyle=\color{blue},
  commentstyle=\color{dkgreen},
  stringstyle=\color{mauve},
  breaklines=true,
  breakatwhitespace=true,
  tabsize=3
}

\lstdefinelanguage{JavaScript}{
  keywords={typeof, new, true, false, catch, function, return, null, catch, switch, var, if, in, while, do, else, case, break},
  keywordstyle=\color{blue}\bfseries,
  ndkeywords={class, export, boolean, throw, implements, import, this},
  ndkeywordstyle=\color{darkgray}\bfseries,
  identifierstyle=\color{black},
  sensitive=false,
  comment=[l]{//},
  morecomment=[s]{/*}{*/},
  commentstyle=\color{purple}\ttfamily,
  stringstyle=\color{red}\ttfamily,
  morestring=[b]',
  morestring=[b]"
}
\usepackage{tikz}
\usetikzlibrary{trees}
\usepackage{textcomp}
\usepackage{placeins}
\usepackage[sorting=nyt]{biblatex}
\usepackage{url}
%%% --- Se supone que estas lineas aumentan el tamaño que tienen las urls en *.bib (luego las corta)--- %%%
\setcounter{biburllcpenalty}{7000}
\setcounter{biburlucpenalty}{8000}
\addbibresource{/Users/brunomedina/Dropbox/Tesis-Egobets/egobets-notas/Decidido.bib}
%----------------------------------------------------------------------------------------
%	Metadata
%----------------------------------------------------------------------------------------
\pdfinfo{
	/Author (Bruno Medina Bolaños Cacho)
	/Title (Egobets, un Sistema Computacional de Asesoría de Apuestas de Futbol)
	/Creator (pdfLatex)
	/Subject (Applied Mathematics and Computation)
	/Keywords (egobets, apuestas, pronosticos, computacion, matematicas)
	/CreationDate (D:\pdfdate)
}

%%% ESte codiguito hace que las imágenes se queden dentro de la subsección
\makeatletter
\AtBeginDocument{%
  \expandafter\renewcommand\expandafter\subsection\expandafter
    {\expandafter\@fb@secFB\subsection}%
  \newcommand\@fb@secFB{\FloatBarrier
    \gdef\@fb@afterHHook{\@fb@topbarrier \gdef\@fb@afterHHook{}}}%
  \g@addto@macro\@afterheading{\@fb@afterHHook}%
  \gdef\@fb@afterHHook{}%
}
% \makeatother
% %%% ESte codiguito hace los quotes de los famosos
% \makeatletter
\newenvironment{chapquote}[2][2em]
  {\setlength{\@tempdima}{#1}%
   \def\chapquote@author{#2}%
   \parshape 1 \@tempdima \dimexpr\textwidth-2\@tempdima\relax%
   \itshape}
  {\par\normalfont\hfill--\ \chapquote@author\hspace*{\@tempdima}\par\bigskip}
\makeatother


%%Parece que no les gustaron mis quotes de famosos
%% Use esta:
%\usepackage{quotchap}

%%%%%%%%%%%%%%%%%%%%%%%%%%%%%%%%%%%%%%%%%%%%%%%%%%%%%%%%%%%%%%%%%%%%%%%%%%%

%%%%%%%%%% Centrar así sin espacio
\newenvironment{tightcenter}{%
  \setlength\topsep{0pt}
  \setlength\parskip{0pt}
  \begin{center}
}{%
  \end{center}
}
%%%%%%%%%%%%%%%%%%%%%%%%%%%%%%%%



%\DeclareGraphicsRule{.tif}{png}{.png}{`convert #1 `dirname #1`/`basename  #1 .tif`.png}
%\DeclareGraphicsExtensions{.pdf,.png,.jpg}


%\date{}                                           % Activate to display a given date or no date
\begin{document}
%----------------------------------------------------------------------------------------
%	Portada, declaración, dedicatoria, agradecimientos, índices
%----------------------------------------------------------------------------------------	
%----------------------------------------------------------------------------------------
%	PORTADA
%----------------------------------------------------------------------------------------

\title{Egobets, un sistema computacional de asesoría en apuestas de fútbol}
\author{Bruno Medina Bolaños Cacho}

\begin{titlepage}
\begin{center}

\textsc{\Large Instituto Tecnológico Autónomo de México}\\[3em]

%Figura
\begin{figure}[ht]
\begin{center}
\includegraphics{resources/logo-ITAM}
\end{center}
\end{figure}

\vspace{2em}

\textsc\huge\textbf{EGOBETS, UN SISTEMA COMPUTACIONAL DE ASESORÍA DE APUESTAS DE FÚTBOL}\\[3em]


\textsc{\large Tesis}

\textsc{que para obtener los títulos de}\\[1em]

\textsc{Ingeniero en Computación y Licenciado en Matemáticas Aplicadas}\\[1em]

\textsc{presenta}\\[1em]

\textsc{\Large Bruno Medina Bolaños Cacho}\\[1em]

\textsc{\large Asesores:}

\textsc{\large Dr. Osvaldo Cairó Battistuti}

\textsc{\large Dr. Adolfo J. De Unánue Tiscareño}\\[1em]

\end{center}

\vspace*{\fill}
\textsc{México, D.F. \hspace*{\fill} 2014}

\end{titlepage}

%----------------------------------------------------------------------------------------
%	DECLARACIÓN
%----------------------------------------------------------------------------------------

\thispagestyle{empty}
\vspace*{\fill}
\begingroup
``Con fundamento en los artículos 21 y 27 de la Ley Federal del Derecho de Autor y como titular de los derechos moral y patrimonial de la obra titulada ``\textbf{EGOBETS, UN SISTEMA COMPUTACIONAL DE ASESORÍA DE APUESTAS DE FÚTBOL}'', otorgo de manera gratuita y permanente al Instituto Tecnológico Autónomo de México y a la Biblioteca Raúl Bailléres Jr., la autorización para que fijen la obra en cualquier medio, incluido el electrónico, y la divulguen entre sus usuarios, profesores, estudiantes o terceras personas, sin que pueda percibir por tal divulgación una contraprestación''.

\centering

\hspace{3em}

\textsc{Bruno Medina Bolaños Cacho}

\vspace{5em}

\rule[1em]{20em}{0.5pt} % Línea para la fecha

\textsc{Fecha}
 
\vspace{8em}

\rule[1em]{20em}{0.5pt} % Línea para la firma

\textsc{Firma}

\endgroup
\vspace*{\fill}

%----------------------------------------------------------------------------------------
%	DEDICATORIA
%----------------------------------------------------------------------------------------

\thispagestyle{empty}
\frontmatter

\chapter*{}
\begin{flushright}
\textit{A mis padres}
\end{flushright}


%----------------------------------------------------------------------------------------
%	AGRADECIMIENTOS
%----------------------------------------------------------------------------------------

\chapter*{Agradecimientos}
%\markboth{AGRADECIMIENTOS23}{AGRADECIMIENTOS} % encabezado 

A mis amigos por todo el apoyo que me han dado para terminar este arduo proyecto. Sin ustedes, no sería quien soy hoy en día, gracias por todas sus enseñanzas y correcciones.

Especialmente, quiero agradecer a los participantes de Egobets, mis amigos: Manuel Colin Hermida, Roberto Hidalgo, Jendanny Raña Custodio y Jaime Rodas. Sin ustedes, este trabajo nunca hubiera sido posible.


%%%%%%%%%%%%%%%%%%%%%%%%%%%%%%%%%%%%%%%%%%%%%%%%%%%%%%%%%%%%%%%%%%%%%%%%%%%%%%%%%%%%%%%%%%%%%%%%%%%%%%%%%%%%%%%5
%	INDICE
%%%%%%%%%%%%%%%%%%%%%%%%%%%%%%%%%%%%%%%%%%%%%%%%%%%%%%%%%%%%%%%%%%%%%%%%%%%%%%%%%%%%%%%%%%%%%%%%%%%%%%%%%%%%%%%5 
 \newpage
 \tableofcontents
 
%%%%%%%%%%%%%%%%%%%%%%%%%%%%%%%%%%%%%%%%%%%%%%%%%%%%%%%%%%%%%%%%%%%%%%%%%%%%%%%%%%%%%%%%%%%%%%%%%%%%%%%%%%%%%%%5
%    LISTA FIGURAS
%%%%%%%%%%%%%%%%%%%%%%%%%%%%%%%%%%%%%%%%%%%%%%%%%%%%%%%%%%%%%%%%%%%%%%%%%%%%%%%%%%%%%%%%%%%%%%%%%%%%%%%%%%%%%%%5
\cleardoublepage
\addcontentsline{toc}{chapter}{Lista de figuras} % para que aparezca en el indice de contenidos
\listoffigures


%----------------------------------------------------------------------------------------
%	Capítulos
%----------------------------------------------------------------------------------------
\mainmatter
\chapter{Introducción}\label{chap:introduccion}

Desde sus orígenes, las apuestas en los partidos de futbol han sido un controversial tema de interés \cite{udovicic1998special}. Predecir los resultados de los partidos y vencer a las casas de apuestas se ha vuelto una fascinación Hollywoodense\footnote{En la película \emph{``Moneyball''} \cite{moneyball} (basado en \cite{lewis2004moneyball}) un equipo de béisbol logra resultados sorprendentes al resolver un problema de optimización con fuertes restricciones monetarias. Mientras que en el filme \emph{``21''} \cite{21Movie}, basado en el libro ``Bringing Down the House'' \cite{patrick2008bringing}, un grupo de estudiantes del MIT (Massachusetts Institute of Technology) utiliza una estrategia de conteo infalible para ganar cientos de miles de dólares en el juego de cartas ``Black Jack''.}. Muchos supuestos ``oráculos'' han utilizado los métodos menos ortodoxos para la predicción de los marcadores \cite{prevos2010psychic}, e incluso se han llevado a cabo acciones fraudulentas para asegurar los marcadores finales de los partidos\footnote{Por ejemplo, en 2006 se suscitó uno de los mayores escándalos en la historia del futbol: \emph{``Calciopoli''}. Se descubrió que varios equipos de la liga italiana conspiraron para influenciar los resultados de los partidos de la temporada 2004/05 \cite{distaso2008corruption}.}. Sin embargo, en la actualidad, las matemáticas y la computación ofrecen un paradigma menos esotérico pero igual de fascinante: la predicción de resultados de partidos de futbol a través de modelos matemáticos.

En este trabajo se describe cómo funciona \textbf{``Egobets''}, una aplicación computacional de las matemáticas al estudio de las apuestas de futbol. Egobets provee asesoría de apuestas personalizadas para partidos de futbol de las siguientes ligas europeas: alemana, española, francesa, inglesa e italiana. Su objetivo es, dado un perfil de riesgo, indicar al usuario la cantidad de dinero y las apuestas que debe realizar para buscar tener ganancias al final de la temporada. Para tal fin, se combinan un conjunto de modelos matemáticos en un sistema robusto computacional.

El sistema Egobets es interesante e innovador ya que no sólo predice el resultado de un partido de futbol, sino que además utiliza la información de todas las ligas europeas para ofrecer una estrategia que maximice la cantidad de dinero a ganar del usuario tomando en cuenta su perfil de riesgo. Adicionalmente, el sistema le sugiere al usuario conservar un porcentaje de su dinero para apostar más agresivamente en caso de perder todas la apuestas de la jornada; garantizando así una mayor cantidad de apuestas durante la temporada y con esto, asegurar una mayor probabilidad de obtener ganancias.

El alcance de este trabajo es el de describir el sistema desarrollado para asesoría de apuestas Egobets. %Gracias a las Matemáticas se pueden desarrollar modelos de fenómenos tan particulares como lo son los partidos de futbol \cite{goddard2005regression}. Además, como se verá en este documento, las Matemáticas proveen las herramientas necesarias para encontrar el conjunto de apuestas a realizar en cada jornada, diversificando el riesgo sobre las apuestas que prometen mayores ganancias. Por otro lado, gracias a los sistemas computacionales y las nuevas tecnologías, se pueden crear las piezas de software de este sistema para ofrecer resultados reales de estas abstracciones matemáticas. Este ecosistema de modelos, aplicaciones y programas funcionan de manera armoniosa presentando resultados al usuario en una interfaz elegante, funcional, simple y fácil de usar. 
Se explicarán los distintos programas y sistemas que conforman el desarrollo, así como las teorías matemáticas que dan sustento al mismo. El documento cuenta con tres capítulos más las conclusiones y esta introducción; el primer capítulo que habla de las apuestas en general, el segundo describe la teoría matemática y el tercero detalla el proyecto realizado.

El primer capítulo, empieza hablando de las ilusiones que mueven a las personas a apostar, después describe como los casinos utilizan los juegos de azar para generar ganancias. En seguida, explica como funcionan los mercados de las apuestas deportivas y sus diferencias con los casinos. Y finalmente, se cierra el capítulo definiendo lo que son los ``momios'' y su papel como regulador en la demanda de la apuesta. 

En este capítulo se presentan, sin entrar al detalle técnico, la existencia de varios modelos matemáticos que describen el comportamiento de los partidos de futbol y como varios autores han logrado encontrar estrategias de apuestas sencillas que garantizan valores esperados positivos de un conjunto de apuestas.
% De igual manera se destaca la importancia de un sistema de reservas para optimizar las tasas de crecimiento de la cantidad de dinero dedicada a las apuestas.
Finalmente se retoman estos conceptos y se detalla el algoritmo que utiliza Egobets para la generación de las recomendaciones de apuestas.

% El segundo capítulo, habla de la existencia de varios modelos matemáticos que describen el comportamiento de los partidos de futbol y como varios autores han logrado encontrar estrategias de apuestas sencillas que garantizan valores esperados positivos de un conjunto de apuestas. Después, se narran los pasos necesarios para el proceso de selección de apuestas y se presenta un ejemplo práctico de como estas teorías generan un conjunto de apuestas apropiado para el perfil de riesgo del usuario.


Egobets.com proporciona al cliente los servicios de asesoría de apuestas personalizada a través de un portal Web usable, práctico y profesional. En el capítulo tercero, se presenta el sistema desarrollado con los fundamentos teóricos descritos en los capítulos anteriores, una verdadera aplicación computacional de las matemáticas. Se define el diseño y la arquitectura del sistema de Egobets en la nube. Además, se habla del conjunto de tecnologías desarrolladas y utilizadas en el sistema. Para terminar, se presentan los diagramas de base de datos y se expone el sistema a través de las tres dimensiones del patrón de diseño Modelo Vista Controlador. 

En el último capítulo, se concluye que se puede llevar un apuesta simple a una estructura de portafolio de inversión. De igual manera se observa que aunque un jugador tuviera en su poder las probabilidades verdaderas de los resultados de los partidos no podría hacer nada con ellas, por lo que es necesario un enfoque de un problema de optimización. Y finalmente que teniendo un sistema metódico que decida las apuestas, remueve la emoción de la apuesta y lo convierte en un riesgo calculado. Para finalizar, se sugieren los distintos campos al que este tipo de sistemas se podría extender: otras ligas, diferentes deportes, elecciones y otros fenómenos parecidos donde se involucre la habilidad humana.
% dan a conocer datos generales de la industria de las apuestas y su participación en la economía mexicana. Después se parte de la descripción de los describen los mercados de apuestas  que ayudan al lector a comprender la relevancia del sistema en la industria de las apuestas y en el ámbito científico; también, se revisan las teorías relacionadas más destacadas y los estudios previos más sobresalientes. Finalmente, se dan a conocer las ligas que se estarán analizando y los motivos por las que fueron elegidas.


% En el tercer capítulo, partiendo de los siguientes dos supuestos: a) un jugador promedio busca maximizar las ganancias de sus apuestas en función de su adversidad al riesgo y, b) apostar siempre conviene más que no apostar. Se plantea una manera de encontrar la apuesta óptima para un partido. Con base en este planteamiento se sigue el análisis a una jornada: ¿A qué partidos de la jornada el usuario le debería apostar? Finalmente, considerando que el usuario busca apostar en todas las jornadas de la temporada, se ataca el problema de la evolución del dinero a apostar durante toda la temporada.
%
%
% En el siguiente apartado del estudio, se habla del conjunto de módulos que conforman el Back Office: sistema de recopilación de información y estadísticas de los partidos, sistema de estimación de probabilidades de los partidos y portal administrativo. Se describe cómo el sistema de recolección de información descarga los datos de las ligas, partidos por jugar y estadísticas de los ya jugados. Se comienza detallando el funcionamiento del sistema recolector de datos, desde la ingestión de los equipos participantes en la temporada vigente, hasta la recolección de los tiros realizados en cada partido por cada jugador. Después, con toda la información obtenida de los desempeños de los equipos en los últimos partidos, se describen las ideas detrás de la predicción de las probabilidades de los resultados de los partidos de futbol. Posteriormente, se exhibe cómo en el portal administrativo se ingresan estas probabilidades junto con los datos de los próximos partidos a jugar. También se detalla cómo este portal, a través de su interfaz gráfica, permite gestionar usuarios, partidos y probabilidades.
%
%
% En la quinta sección de este estudio, se utilizan las teorías y propuestas de los primeros apartados de esta tesis para el diseño y desarrollo del portal público. Se describe cómo este portal también ofrece al usuario revisar y actualizar su perfil, retomar la encuesta de riesgo, revisar los últimos resultados de los partidos, ver la tabla de ``Power Ranking''\footnote{Una tabla que presenta a los equipos del más fuerte al más débil de la temporada.} y obtener la asesoría de apuestas para la jornada en curso.


% En el último capítulo, se concluye que se puede llevar un apuesta simple a una estructura de portafolio de inversión. De igual manera se observa que aunque un jugador tuviera en su poder las probabilidades verdaderas de los resultados de los partidos no podría hacer nada con ellas, por lo que es necesario un enfoque de un problema de optimización. Y finalmente que teniendo un sistema metódico que decida las apuestas, remueve la emoción de la apuesta y lo convierte en un riesgo calculado. Para finalizar, se sugieren los distintos campos al que este tipo de sistemas se podría extender: otras ligas, diferentes deportes, elecciones y otros fenómenos parecidos donde se involucre la habilidad humana.

 %Introducción
\graphicspath{{/Users/brunomedina/Dropbox/Tesis-Egobets/egobets-notas/resources/marco/}}
\chapter{Panorama general de las apuestas}


En este capítulo, se explora el panorama general de las apuestas para auxiliar al lector en la comprensión de la relevancia de Egobets en la industria de las apuestas y en el ámbito científico; también, se revisan las teorías relacionadas más destacadas y los estudios previos más sobresalientes. Finalmente, se dan a conocer las ligas que se estarán analizando y los motivos por las que fueron elegidas.

 \section{La fascinación por los juegos de azar}

Nadie conoce el origen de las apuestas, algunos dicen que todo comenzó con un anónimo paleolítico que rodó unos cuantos huesos para decidir hacia que dirección ir a cazar \cite{schwartz2013roll}. Más adelante, tanto los antiguos griegos como los etruscos examinarían la forma y las características del higado de una oveja para tomar las mejores decisiones para su futuro\footnote{Los adivinos llamados ``Arúspices'' eran los encargados de llevar la tarea de precedir el futuro en función de la examinación de las entrañas de varias bestias.}. Siglos después, los romanos usarían los huesos astrágalos (Ver figura~\ref{Fig:huesos}) de animales como precursores a los dados \cite{schwartz2013roll}.

\begin{figure}[!htb]\centering
   \begin {minipage}{0.85\textwidth}
     \frame{\includegraphics[width=\linewidth]{huesos}}
     \caption{Astrágalos, los predecesores de los dados}\label{Fig:huesos}
   \end{minipage}
\end{figure}

Hoy en día, las apuestas representan uno de los negocios más redituables del mundo. En 2013, se estimó que las ganancias brutas de esta industria sumaron más de cuatroscientos cuarenta mil millones de dólares\footnote{Acorde a la empresa de Inteligencia de Mercado ``H2 Gambling Capital'' \cite{economistHouseWins}.}. Como se puede observar en la figura~\ref{Fig:gasto-apuestas}, Estados Unidos encabeza la lista como el país que más gasta en apuestas, seguido por China. También se advierte que los residentes de Australia y Singapur apuestan mucho más agresivamente que los de cualquier otro país. Para terminar, en esta misma gráfica se estima que para el 2018 el gasto en apuestas será de más de quinientos mil millones de dólares.


\begin{figure}[!htb]\centering
   \begin {minipage}{0.85\textwidth}
     \frame{\includegraphics[width=\linewidth]{apuestas}}
     \caption{Miles de millones de dólares en apuestas}\label{Fig:gasto-apuestas}
   \end{minipage}
\end{figure}

Con estos datos, la relevancia de la industria de las apuestas en el mundo se vuelve evidente.  Por otra parte, con respecto a las apuestas en línea, la firma KPMG \cite{kpmgOnlineGaming} reporta que el mercado global de apuestas en línea creció un cuarenta y dos por ciento de Veintún mil doscientos millones de dólares en 2008 a Treinta mil millones de dólares en 2012. Este porcentaje es notablemente superior al quince por ciento esperado para el crecimiento del total de la industria de apuestas para el mismo periodo. 

Específicamente en Estados Unidos, Goldman Sachs valoró en 2009 que el mercado de apuestas en línea en caso de ser legalizado\footnote{El ``Unlawful Internet Gambling Enforcement Act of 2006'' (UIGEA)  prohíbe a los bancos y a las compañías de tarjetas de credito procesar cargos relacionados a casinos en línea. Según Alexander, G. \cite{alexander2008us} las cuatro preocupaciones federales principales detrás de este acto son: Primero, el internet provee un acceso fácil a las apuestas, esto podría exacerbar las tentaciones que enfrentan los apostadores compulsivos. Segundo, es muy complicado verificar la mayoría de edad a través de un sitio de apuestas. Tercero, los casinos en linea tienen un incentivo para defraudar a los usuarios gracias a la falta de regulación de la industria. Y cuarto, dado el volumen, la velocidad, el alcance internacional de la transacciones realizadas en internet y el alto nivel de anonimidad que tienen los operadores de casinos electrónicos; los oficiales federales creen que las apuestas en línea son particularmente suceptibles al lavado de dinero.} podría valer hasta doce mil millones de dólares \cite{goldmanParty}.En este mismo documento de KPMG \cite{kpmgOnlineGaming}, México se propone como un mercado potencialmente lucrativo. Una de las principales razones es la legislación que permite el juego en línea\footnote{En 2004, la Ley Federal de Juegos con Apuestas y Sorteos permitió y reguló los Juegos en Línea.}. La otra razón, el valor del mercado mexicano del juego en línea se estima en cuatro mil seiscientos millones de dólares \cite{yogonet}.

 \section{La casa siempre gana}



%Que no les gustó mi "INSPIRATIONAL QUOTE"
\begin{chapquote}{Adam Smith, \textit{Filósofo} \cite{smith1963wealth}}
	``No hay proposición más cierta en matemáticas que la siguiente: Entre más boletos [de lotería] compre, más probabilidades tiene de ser un perdedor. Compre todos los boletos de la lotería y pierda con certeza; cuanto más boletos compre, más cerca estará de esta certeza''
\end{chapquote}

%A ver así:
% \begin{savequote}[0.55\linewidth]
% ``No hay proposición más cierta en matemáticas que la siguiente: Entre más boletos [de lotería] compre, más probabilidades tiene de ser un perdedor. Compre todos los boletos de la lotería y pierda con certeza; cuanto más boletos compre, más cerca estará de esta certeza''
% \qauthor{Adam Smith, \textit{Filósofo} \cite{smith1963wealth}}
% \end{savequote}
% ok....  no jaló...


El principio básico detrás de un casino es muy sencillo: \emph{la ventaja de la casa}. Cada uno de los juegos que ofrece el casino tiene detrás un robusto sustento matemático, de manera que a pesar de aparentar ser un juego justo, le confiere a la casa una ventaja porcentual sobre el conjunto de jugadores. Al final del día, esta ventaja y la ley de los grandes números, le garantizan a los casinos que a largo plazo tendrán suficientes ganancias para subsistir, mantener su operación y gozar de utilidades sorprendentes. Sin embargo, no hay que olvidar que los fenomenos estudiados siguen siendo producto del azar, por lo que una buena racha de algunos ``Grandes Apostadores'' podría llegar a asustar aun a los dueños  más racionales de casinos \cite{hannum2005practical}.



Según Hannum \cite{hannum2005practical} hay dos grandes razones por las que la gente apuesta:

\begin{enumerate}
	\item \textbf{Entretenimiento.} Un individuo podría utilizar mil pesos para ir a un casino o a un concierto. Si la ventaja de la casa es muy grande y la persona pierde su dinero rápidamente, entonces la experiencia del entreteniminto del casino no sería apreciada por el jugador. Por el otro lado, si el casino logra entretener a la persona por una tarde mientras le regala bebidas y comida, entonces puede que este individuo repita la experiencia y nunca más asista a un concierto.
	\item \textbf{Cambio de Vida.} Si una persona ahorrara cien pesos semanalmente, al final del un año tendría cinco mil doscientos pesos. Pero si ese dinero lo gastara para comprar boletos de lotería, tendría la posibilidad de ganarse cuarenta millones de pesos. Claramente la probabilidad es muy cercana a cero; sin embargo, este gasto podría ser visto por esta persona como una única oportunidad para cambiar su vida.

\end{enumerate}

\emph{La ventaja de la casa}, se puede entender mejor analizando cada uno de los juegos que ofrece el casino y las probabilidades de ganar que tienen los jugadores. Tómese por ejemplo el juego de la ruleta americana\footnote{El juego de la ruleta americana consiste en 38 casillas que alternan 18 casillas rojas, 18 casillas negras y 2 casillas verdes. Cuando el crupier hace girar la ruleta, de manera aleatoria cae una pelotita en una de las casillas. Los jugadores apuestan sobre la posición final de la pelotita.}:
Cuando un jugador apuesta sobre el color negro (i.e. que la pelotita caiga sobre alguna de las casillas negras) entonces se tiene que la probabilidad de que el jugador gane la apuesta es de:\\
\[p\{\text{La pelotita caiga en casilla negra}\} = \frac{\text{\# casillas negras}}{ \text{\# casillas totales}}  = \frac{18}{38}\]\\

Afortunadamente para la casa, hay $2$ casillas que no son color negro ni rojo, por lo que de las $38$ casillas sólo $36$  tienen estos colores. Por lo tanto, la probabilidaad de que la pelotita caiga en una casilla verde es la siguente:

\[p\{\text{La pelotita no caiga ni en casillas rojas ni en negras}\} =\] 
\[\frac{\text{\# casillas totales - (\# casillas rojas + \# casillas negras)}}{ \text{\# casillas totales}}  =\]
\[\frac{38-18-18}{38} = \frac{2}{38}  \]

Estos $\frac{2}{38}$ son la ventaja de la casa, ya que cuando un jugador apuesta al color negro en la ruleta y acierta, recibe la misma cantidad de dinero que podría perder. Sin embargo, apostó a ganar con una probabilidad de $\frac{18}{38}$, pero la probabilidad de perder la apuesta es igual a $1 - \frac{18}{38} = \frac{20}{38}$. Este detalle hace importante ver el valor esperado que tiene esta apuesta para el jugador:
\[E[\text{Apostar }k\text{ pesos al color negro}] = k  \cdot   p\{\text{La pelotita caiga en casilla negras}\} \]
\[- k  \cdot   p\{\text{La pelotita no caiga en casilla negra}\} = k \cdot \frac{18-20}{38}= - k \cdot \frac{2}{38}\]

Dado que siempre que se apuesta $k>0$, esto implica que:
\[E[\text{Apostar }k\text{ pesos al color negro}] = - k \cdot \frac{2}{38} < 0; \forall k \in \mathbb{N} \]
Este resultado quiere decir que a la larga el jugador \textbf{siempre} va a terminar perdiendo dinero.
En un principio, $\frac{2}{38}$ de probabilidad pareciera poco, pero al multiplicarlo por la gran cantidad de jugadores y apuestas que se realizan en los casinos el monto final se vuelve exorbitante.

Este sencillo ejercicio ejemplifica como todos los juegos que se tienen en los casinos ofrecen una ventaja para la casa. Es interesante mencionar, que además de los juegos de azar como la ruleta, hay juegos que obligan al jugador a tener cierta habilidad para no perder su dinero tan rápidamente, este es el caso de juegos como el Blackjack que le dan a la casa ventajas más pequeñas al enfrentarse a jugadores expertos. La ventaja de la casa sustenta las ganancias del casino, sin embargo calcular esta ventaja puede llegar a ser  complicado y requerir un análisis matemático mucho más sofisticado e incluso se puede llegar a necesitar programar el juego para correr simulaciones y estimar estas probabilidades.

\begin{table}[ht]
\centering
\resizebox{\textwidth}{!}{%
\begin{tabular}{|l|c|}
\hline
\textbf{Juego}                            & \textbf{Ventaja de la Casa} \\ \hline
Ruleta (con doble cero)                   & 5.3\%                       \\ \hline
Dados (pass/come)                         & 1.4\%                       \\ \hline
Dados (pass/come con momios dobles)       & 0.6\%                       \\ \hline
Blackjack - jugador promedio              & 2.0\%                       \\ \hline
Blackjack - 6 barajas, estrategia básica  & 0.5\%                       \\ \hline
Blackjack - una baraja, estrategia básica & 0.0\%                       \\ \hline
Baccarat (sin apuestas de empate)         & 1.2\%                       \\ \hline
Caribbean Stud                            & 5.2\%                       \\ \hline
Let It Ride                               & 3.5\%                       \\ \hline
Poker de tres cartas                      & 3.4\%                       \\ \hline
Pai Gow Poker (ante/play)                 & 2.5\%                       \\ \hline
Tragamonedas                              & 5\% - 10\%                  \\ \hline
Video Poker                               & 0.5\% - 3\%                 \\ \hline
Keno (promedio)                           & 27.0\%                      \\ \hline
\end{tabular}
}
\caption{Ventajas de la casa para juegos populares de casino \cite{hannum2005practical}}
\label{ventaja-casa}
\end{table}

Finalmente, aun sin la ventaja de la casa se deber recordar que existe un famoso problema y su corolario que garantizan que la casa siempre gane: \emph{La ruina del Jugador} \cite[p.~95-99]{ross2006first}. Este problema enfrentado por varios famosos matemáticos\footnote{Se dice que Blaise Pascal se lo planteó a Pierre Fermat en 1656 \cite{edwards1983pascal}. Después Fermat se lo replanteo a Christian Huygens en 1657 y finalmente James Bernoulli lo resolvió en su forma general como el problema de la ``Duración de Juego''. Fue publicado ocho años después de la muerte de Bernoulli en 1713 \cite[p.~98]{ross2006first}.} deja la siguiente lección: La probabilidad de que un jugador pierda todo su dinero es igual a  $p_1 = \frac{n_2}{n_1 + n_2}$ donde $p_1$ es la probabildad de ganar del jugador $1$ y $n_i$ es la cantidad de dinero que va a apostar el jugador $i$. Desafortunadamente, usualmente la casa tendrá más dinero para apostar que cualquier jugador, por lo que con esta fórmula la casa siempre gana\footnote{Ver demostración en el apéndice~\ref{chap:ruina}.}.


 \section{Mercados de apuestas deportivas}
 
 A diferencia de las máquinas tragamonedas y los juegos de mesa de los casinos, las apuestas deportivas tienen una gran ventaja para los apostadores: los resultados de los encuentros deportivos \textbf{no son completamente aleatorios,} ya que dependen en cierta medida del nivel de juego que tienen los equipos participantes. Hannum menciona en su libro \cite{hannum2005practical} : \emph{``La ventaja de la casa existe para casi todas las apuestas en un casino (ignorando las salas de Poker y las apuestas deportivas donde algunos pocos profesionales pueden vivir de de las apuestas)''}

 Existen dos grandes mercados en esta rama de las apuestas \cite{chung2010empirical}:
 \begin{itemize} 
 	\item \emph{Bookies.}\footnote{``Bookie'' proviene de la palabra en inglés ``bookmakers'' que en español se utiliza como: ``Corredor de apuestas''.} El corredor de apuestas analiza los diferentes resultados de un encuentro deportivo en función de los participantes. Con base en este análisis, el bookie publica un número (llamado \emph{``momio''}) para cada uno de los posibles desenlaces del partido. Este ``momio'' representa la cantidad potencial de retorno de esa apuesta (incluída la cantidad de dinero arriesgada)\footnote{Por ejemplo, supongase que se apuestan cien dólares a favor del empate de un partido con un momio igual a $3.640$. En caso de que se ganara la apuesta, el jugador recibíria la cantidad de $(100)(3.640) = 364$ dólares. Sustrayendo los cien dólares que apostó, su ganancia neta sería de $264$ dólares.}. El bookie recibe apuestas sobre estos momios y cobra una comisión por cada operación recibida. 
 	\item \emph{Sistema parimutual.} En este mercado no existen  corredores, ni momios. El pago que reciben los jugadores depende de la cantidad de dinero recopilada por todas las apuestas recibidas. Por lo tanto, las ganancias de los jugadores no están determinadas hasta que todas las apuestas se reciben\footnote{Las afamadas ``quinielas'' son un tipo de apuesta de mercado parimutual.}.
 	\end{itemize}

 Aunque la eficiencia\footnote{Fama, E \cite{fama1998market} sugiere que la eficiencia de mercado es cuando los precios reflejan completamente toda la información disponible de una acción en particular.} de estos mercados no es un tema que se aborde a profundidad en esta tesis, es interesante mencionar que ha sido  muy cuestionada. Sauer \cite{sauer1998economics} y Williams \cite{williams1999information} critican varias de sus anomalías; por ejemplo hay un problema interesante en la eficiencia del mercado de las apuestas de los bookies conocido como: Sesgo del \emph{``favorite - long shot''.} Estos autores explican que las apuestas a los equipos favoritos generan un mayor retorno de dinero en comparación con el retorno generado por las apuestas a equipos cuyas probabilidades de ganar son mucho menores (``longshot bets''). Incluso se han realizado varios estudios proponiendo un nuevo mercado de tipo ``doble subasta'' que consiste en tener compradores y vendedores proponiendo los precios de la apuesta, cuando dos de ellos coinciden, se lleva acabo la transacción. Referirse a Ozgit\cite{ozgit2005posted}.

 El mercado pertinente para este trabajo es el de los ``bookies'' o corredores de apuestas. Sin embargo, bajo las condiciones adecuadas la asesoría de apuestas podría ayudar a los jugadores en el mercado parimutual. Por ejemplo, en una apuesta entre compañeros del trabajo el usuario del sistema podría indagar la apuesta realizada por cada uno de los participantes y conocer la cantidad de retorno que tiene de cada uno de los resultados. Si el sistema de Egobets le recomendara realizar esta apuesta en particular, bastaría verificar que el retorno del pago sea mejor que el pago que realizan las casas de apuestas en general. Realizando esta acción sistemáticamente, se podrían conseguir mejores ganancias que las ofrecidas por el mercado.

 Sin importar el mercado, es importante tomar en cuenta el hecho de que las probabilidades reales que tienen los distintos resultados de cualquier partido \textbf{son desconocidas}. Por lo que los ``bookies'' contratan a empresas consultoras, o designan un área de la compañía para calcular los momios que se publican en el mercado. Este punto es básico en la creación de sistemas como Egobets, ya que el hecho de que las probabilidades sean desconocidas permite que Egobets busque estimar probabilidades mucho más cercanas a las reales. Al minimizar el error de estimación que tienen las probabilidades sugeridas por las casas de apuestas, se pueden encontrar apuestas cuyo retorno sería mayor al que realmente debería ser. Y como se verá en la siguiente sección, este error de estimación siempre existirá, debido a que los momios publicados por las casas de apuestas no buscan reflejar las probabilidades reales de los resultados de los partidos.

 Una ventaja con la que cuentan los apostadores de eventos deportivos: si los bookies establecen mal sus momios\footnote{La cantidad de dinero que pagan las apuestas a eventos completamente aleatorios, como los juegos dentro de los casinos, se puede calcular explícitamente. Por lo que la única manera de que una de estas apuestas tuviera valor esperado positivo sería si se cometieron errores en los cálculos.}, entonces ciertos jugadores pueden llegar a tener valores esperados positivos en sus apuestas y con esto la casa podría llegar a perder mucho dinero, incluso a largo plazo (como se verá en la siguiente sección). Procédase a la siguiente sección para conocer los orígenes de los momios y las condiciones que permiten a los bookies ganar dinero con ellos.
 
 \section{Bookies y momios}
Ya que el objeto de estudio central de esta tesis toma lugar en el deporte del futbol, analícese la pregunta más recurrente de un partido de futbol: \emph{¿Qué equipo va a ganar este partido?}

La respuesta tiene $3$ posibilidades:
 \begin{enumerate}
  \item El partido lo gana el equipo de casa.
  \item Es un empate.
  \item El partido lo gana el equipo visitante.
 \end{enumerate}
 
En un principio se pudiera pensar que los tres eventos son equiprobables, pero como se mencionó en la sección anterior esto resulta imposible por muchísimas razones, como por ejemplo: ser el equipo visitante conlleva una desventaja importante en el desempeño del partido (Ver \cite{roffe2007crisis}), o la desventaja de tener a los jugadores estrella lesionados; incluso las condiciones climáticas (altitud, tipo de pasto, lluvia) durante el partido pueden ser determinantes para el resultado final. En un caso extremo, imagínese el escenario donde ambos equipos fueran igual de buenos en todos los aspectos. En este escenario, la probabilidad de empate sería mayor que las otras dos. Sin embargo, estas razones no son (tampoco) suficientes para calcular los resultados de los partidos de manera determinística. Gracias a estas particularidades es que el estudio de este mercado de apuestas se vuelve tan interesante.

Anteriormente se definió un momio como el número que indica la cantidad de dinero que obtiene un jugador al ganar una apuesta. Sin embargo, para generar los momios, las casas de apuesta realizan estudios minuciosos de mercado y de los deportes en sí mismos. Las probabilidades de cada resultado de un partido no son, de lejos, los únicos factores a considerar por el bookie a la hora de generar los momios.  

Levitt \cite{levitt2004gambling} menciona tres escenarios que permitirían a los bookies generar ganancias:
 \begin{enumerate}
  \item \textbf{Encontrar el precio de equilibrio.} Los bookies buscan predecir los momios\footnote{Antes del partido estos momios pueden llegar a tener ajustes tipicamente pequeños y relativamente infrecuentes.} que igualen el precio de equilibrio del mercado de apuestas. Es decir, se buscan los momios que equilibran la cantidad de dinero en cada lado de la apuesta. Esto implica que la diferencia de dinero entre todas las apuestas es igual a cero. Por lo que sin importar quien gane el partido, el bookie cobrará su comisión de las apuestas intacta y no tendrá deudas en ninguna de las apuestas.
  
  \item \textbf{Predecir los resultados del partido.} Si el bookie fuera sistemáticamente mejor en predecir los resultados de los partidos, entonces podría publicar el momio ``correcto'' del partido (i.e. el momio que equilibra la probabilidad de que una apuesta en cualquiera de los resultados gane). Aunque la cantidad de dinero no estaría equilibrada, en promedio el bookie ganaría la comisión cobrada a los jugadores. Nótese que a diferencia del primer escenario, en este esquema existe un riesgo enorme al que se exponen las casas de apuestas. Si llegaran a fijar un momio muy alejado del precio de equilibro del mercado, entonces podría darse el caso de que la cantidad de dinero a pagar a los ganadores de la apuesta sea mayor a la cantidad de dinero recaudada por las apuestas complementarias, lo que podría significar pérdidas millonarias para el bookie\footnote{Levitt \cite{levitt2004gambling} cuenta el ejemplo del Epson Derby de 1946, donde los momios fueron errados y la mitad de todas las casas de apuestas británicas se fueron a bancarrota.}.

 Ahora, tómese de ejemplo el siguiente partido con los siguientes momios:
 \begin{itemize}

 \item \textbf{Cagliari Vs Juventus.}
  \begin{itemize}
    \item \textbf{9.290} Gana local (Cagliari)
    \item \textbf{1.423} Empate.
    \item \textbf{4.760} Gana visitante (Juventus)
  \end{itemize}
 \end{itemize}

  Sea $m_i$ el momio que propone el bookie para que los jugadores apuesten a que se cumpla el evento $i$. Entonces, se define el momio de la siguiente manera:
 \[m_i = \frac{1}{1 - \hat{p_i}}\]
 Donde $\hat{p_i}$ es la probabilidad estimada que tiene el bookie de que suceda el evento $i$\footnote{Otros deportes tienen diferentes tipos de momios y su definición varía dependiendo del tipo de apuesta, se puede leer más de ellos en \cite{ignatin1984sports}.}

 Con estos momios se pueden calcular las ``probabilidades estimadas por el bookie'':
 \[\hat{p_L} = 1 - \frac{1}{9.290} \approx 0.107643...\] 
 \[\hat{p_E} = 1 - \frac{1}{1.423} \approx 0.702741...\]
 \[\hat{p_V} = 1 - \frac{1}{4.760} \approx 0.210084...\]
 Donde $\hat{p_L}$ es la probabilidad estimada por la casa de apuestas de que gane el equipo local, $\hat{p_E}$ es la probabilidad de que el partido termine en empate y $\hat{p_V}$ es la probabilidad de que gane el equipo visitante.
 
 Estas probabilidades estimadas proponen que el escenario más probable es un empate, después la victoria del Juventus y finalmente la victoria del Cagliari. Curiosamente en este ejemplo, se puede ver que aunque el equipo Cagliari es local, se enfrenta a un adversario que puede contra la desventaja de ser visitante.

 Uno de los teoremas básicos de la probabilidad, dice que la suma de las probabilidades de todos los resultados del evento debe sumar uno \cite{ross2006first}. Sin embargo, en el caso de los bookies, las probabilidades estimadas exceden la unidad. Este excedente, es la vieja conocida: ``Ventaja de la Casa''.

 Sea $\mathcal{O}$ el conjunto de todas las opciones sobre las que un jugador puede apostar para cierto evento. Para el caso de las apuestas sobre los resultados de los partidos, se tiene que: $\mathcal{O} = \{\text{Apostar a que gane local}, \text{Apostar al empate}, \text{Apostar a que gane visitante}\}$.

 Por lo tanto:
 \[\text{Ventaja de la casa} =  \left(\sum_{i \in \mathcal{O}}{\left(1 - \frac{1}{\text{Momio de la opción } i}\right)}\right) - 1\] 

 Siguiendo con el ejemplo, se tiene que la ventaja de la casa de este bookie para esta apuesta en particular es:
 \[\text{Ventaja de la casa} = \hat{p_L} + \hat{p_E} + \hat{p_V} - 1 \approx 0.020467\]

 Este $2.046\%$ es el punto de partida que la casa quiere obtener en ganancias. 


  \item \textbf{Predecir los resultados y encontrar el precio de equilibrio.} Si el bookie es mejor que los jugadores prediciendo el resultado de los partidos y también puede encontar el precio de equilibrio, entonces podría con esta información mejorar sus ganacias esperadas al publicar un momio ``equivocado'' de tal manera que el momio (precio) de equilibrio quede posicionado donde sus ganancias se incrementen. Ahora, hay ciertas restricciones en cuanto a la distancia a la que puede quedar posicionado el momio ``equivocado'', ya que pueden existir jugadores que conozcan el momio ``correcto'' por lo que entre mayor sea esta distancia podría generar mayores ganancias para estos jugadores. Por ejemplo, supongase el siguiente caso extremo: Se disputa un partido entre los equipos $A$ y $B$, el equipo $B$ es el favorito para ganar. Es por esto que la casa de apuestas arregló que el partido lo gane el equipo $A$. El bookie conoce que la probabilidad de que el equipo $A$ gane el partido es $1$, es por esto que fija los momios más competitivos del mercado a favor del equipo $B$ y a favor del empate, mientras que el momio a favor del equipo $A$ lo presenta mucho menos competitivo que el resto del mercado. Con estas acciones, la casa de apuestas maximiza las apuestas recibidas en contra del equipo $A$ y minimiza las apuestas a favor del equipo $A$ dado que las otras casas de apuestas pagan mucho mejor este resultado. Ahora, si los momios fueran absurdamente buenos en contra del equipo $A$ ciertos apostadores podrían intuir que la casa de apuestas sabe algo que los demás no saben y podrían apostar cantidades exorbitantes a favor del equipo $A$.

  % Para el ejemplo anterior, si la casa conociera las probabilidades reales de los resultados del partido, entonces podría (por ejemplo) presentar para ese partido al equipo visitante (Juventus) como la mejor opción para la apuesta, cuando en realidad no lo es. Este tipo de técnicas aplicadas sistematicamente pueden ayudar al bookie a obtener ingresos mayores a largo plazo.
 \end{enumerate}


 De estos tres escenarios, Levitt \cite{levitt2004gambling} examina de un bookie en línea, veinte mil apuestas de doscientos ochenta y cinco jugadores para los partidos de la NFL\footnote{La liga de futbol americano más popular: National Football League.}. Esto fue lo que encontró:
 \begin{itemize}
  \item A pesar de ser un estudio para una sola casa de apuestas, sugiere que se pueden generalizar ya que casi todas las casas ofrecieron casi los mismos momios para los mismos partidos
  \item El bookie no parece buscar el precio de equilibrio del mercado. Acorde a los resultados en casi la mitad de todos los juegos al menos dos tercios de las operaciones caen en un solo lado de la apuesta.
  \item El bookie parece colocar precios estratégicamente para explotar los sesgos de los apostadores. Es decir, los jugadores parecen tener un sesgo sistemático hacia los equipos favoritos y, en menor medida, hacia los equipos visitantes. En consecuencia, el bookie logró atraer mayor atención a partidos donde estos equipos no tuvieron buenas actuaciones; estos precios lograron elevar las ganancias hasta un veinte por ciento más que con lo que hubieran obtenido en el primer escenario.
  \item Hay poca evidencia de que existan jugadores que hayan sido capaces de vencer a los bookies sistemáticamente.
  \end{itemize}
 
 Estas conclusiones indican que, al menos en el deporte del futbol americano, las casas de apuestas buscan siempre tener mayores ganancias. Es por esta razón que podrían estar usando el tercer escenario descrito previamente. Es interesante destacar que los momios, en realidad reflejan la probabilidad que el mercado cree que tiene cada resultado de un partido. Es por esto que sistemas como Egobets, que funcionan al aprovecharse de los momios ``equivocados'' propuestos por las casas de apuestas, pudieran llegar a funcionar satisfactoriamente.
 
 
 \section{La apuesta ganadora}

 \begin{enumerate}[(a)]
  \item Sean $p_L$, $p_z$, $p_v$ las probabilidades de que gane local, empaten o gane visitante, respectivamente. Sean $\mu_L$, $\mu_z$ y $\mu_v$ los momios respectivos. El problema de decisión de apostar \$\,1 en esta situación es:\\
 
 % Set the overall layout of the tree
 \tikzstyle{level 1}=[level distance=3.5cm, sibling distance=2.5cm]
 \tikzstyle{level 2}=[level distance=3.5cm, sibling distance=2cm]
 \tikzstyle{level 3}=[level distance=3.5cm, sibling distance=2cm]


 \begin{figure}[ht]
 \begin{tikzpicture}[grow=right, sloped]
 \node[text width=4em, text centered] {$\square$}
 %%%%%%%%%%%%%%%%%%%%%%%%%%%%%%%%%%%%%%%%%%%%%%%%%
 %cuarto
 %%%%%%%%%%%%%%%%%%%%%%%%%%%%%%%%%%%%%%%%%%%%%%%%
 child {       
     node[text width=4em, text centered] {1}
              edge from parent         
  node[above] {$\delta_{NA}$}
     }
 %%%%%%%%%%%%%%%%%%%%%%%%%%%%%%%%%%%%%%%%%%%%%%%%%
 %tercero
 %%%%%%%%%%%%%%%%%%%%%%%%%%%%%%%%%%%%%%%%%%%%%%%%
 child {       
     node[text width=4em, text centered] {\textbigcircle}
     child {
                 node[circle, minimum width=1pt,fill, inner sep=0pt, label=right:
                     {$0$}] {}
                 edge from parent
                 node[above] {$1-p_v$}             
             }
             child {
                 node[circle, minimum width=1pt,fill, inner sep=0pt, label=right:
                     {$\mu_v$}] {}
                 edge from parent
                 node[above] {$p_v$}              
             }
              edge from parent         
  node[above] {$\delta_v$}
     }
 %%%%%%%%%%%%%%%%%%%%%%%%%%%%%%%%%%%%%%%%%%%%%%%
 %segundo
 %%%%%%%%%%%%%%%%%%%%%%%%%%%%%%%%%%%%%%%%%%%%%%%%
     child {       
     node[text width=4em, text centered] {\textbigcircle}
     child {
                 node[circle, minimum width=1pt,fill, inner sep=0pt, label=right:
                     {$0$}] {}
                 edge from parent
                 node[above] {$1-p_z$}             
             }
             child {
                 node[circle, minimum width=1pt,fill, inner sep=0pt, label=right:
                     {$\mu_z$}] {}
                 edge from parent
                 node[above] {$p_z$}              
             }
              edge from parent         
  node[above] {$\delta_E$}
     }
 %%%%%%%%%%%%%%%%%%%%%%%%%%%%%%%%%%%%%%%%%%%%%%%
 %primero
 %%%%%%%%%%%%%%%%%%%%%%%%%%%%%%%%%%%%%%%%%%%%%%%%
     child{
     node[text width=4em, text centered] {\textbigcircle}        
             child {
                 node[circle, minimum width=1pt,fill, inner sep=0pt, label=right:
                     {$0$}] {}
                 edge from parent
                 node[above] {$1-p_L$}             
             }
             child {
                 node[circle, minimum width=1pt,fill, inner sep=0pt, label=right:
                     {$\mu_L$}] {}
                 edge from parent
                 node[above] {$p_L$}              
             }    
  edge from parent         
  node[above] {$\delta_L$}
     };  
 \end{tikzpicture}
 \caption{Decidir por quién apostar}
 \end{figure}

   $E_p[U(\delta_i)]=p_i\mu_i;\quad i=L,Z,V$\\
  
   Sol: Se escoge $\rho_i \,\, \cdot \ni \cdot \,\, E_p[U(\delta_i)]=max\{p_L\mu_L,p_z\mu_z,p_v\mu_v,1\}$
  
   \item Se quiere decidir si apostar o no en la ocurrencia de un evento: Sea $p=p(E)$ y $f_p$ densidad de $p$. Sea $\mu$ el momio en el caso de ocurrencia. El problema de decisión asociado es el siguiente:\\
 
 \begin{figure}[!ht]
  \begin{tikzpicture}[grow=right, sloped]
 \node[text width=4em, text centered] {$\square$}
 %%%%%%%%%%%%%%%%%%%%%%%%%%%%%%%%%%%%%%%%%%%%%%%%%
 %segundo
 %%%%%%%%%%%%%%%%%%%%%%%%%%%%%%%%%%%%%%%%%%%%%%%%
 child {       
     node[text width=4em, text centered] {0}
              edge from parent         
  node[above] {$\delta_{NA}$}
     }
 %%%%%%%%%%%%%%%%%%%%%%%%%%%%%%%%%%%%%%%%%%%%%%%
 %primero
 %%%%%%%%%%%%%%%%%%%%%%%%%%%%%%%%%%%%%%%%%%%%%%%%
     child{
     node[text width=4em, text centered]{\textbigcircle}
    	    child {
    	        node[]{\textbigcircle}   	                 
                 %node[above] {$f_p$}
                 child{node[circle, minimum width=1pt,fill, inner sep=0pt, label=right:
                     {$-1$}] {}                    
                 edge from parent
                 node[above] {$1-p$}             
             }
             child {
                 node[circle, minimum width=1pt,fill, inner sep=0pt, label=right:
                     {$\mu-1$}] {}
                 edge from parent
                 node[above] {$p$}              
             }
             edge from parent
             node[above] {$f_p$}}    
  edge from parent         
  node[above] {$\delta_A$}
     };
 \end{tikzpicture}  
 \caption{Decidir si apostar o no apostar}
 \end{figure}
 
 \newpage
  
   $\rightarrow E_p[U(\delta_A)]=E_{f_p}[p(\mu-1)-(1-p)]\\
   =E_{f_p}[p(\mu)-1]\\
   =E_{f_p}(p)\mu-1$\\

   Apuestas si $E_{f_p}(P)\cdot \mu \ge 1$
  
   \item Mismo problema que el caso anterior, sólo que la utilidad depende de $p$ y $\mu$: $U: \Re\times[0,1]\rightarrow\Re$\\
   $(U(0,p)=0\quad\forall p)$.\\
 
 \begin{figure}[ht]
 \begin{tikzpicture}[grow=right, sloped]
 \node[text width=4em, text centered] {$\square$}
 %%%%%%%%%%%%%%%%%%%%%%%%%%%%%%%%%%%%%%%%%%%%%%%%%
 %segundo
 %%%%%%%%%%%%%%%%%%%%%%%%%%%%%%%%%%%%%%%%%%%%%%%%
 child {       
     node[text width=4em, text centered] {0}
              edge from parent         
  node[above] {$\delta_{NA}$}
     }
 %%%%%%%%%%%%%%%%%%%%%%%%%%%%%%%%%%%%%%%%%%%%%%%
 %primero
 %%%%%%%%%%%%%%%%%%%%%%%%%%%%%%%%%%%%%%%%%%%%%%%%
     child{
     node[text width=4em, text centered]{\textbigcircle}
    	    child {
    	        node[]{\textbigcircle}   	                 
                 %node[above] {$f_p$}
                 child{node[circle, minimum width=1pt,fill, inner sep=0pt, label=right:
                     {$U(-1,p)$}] {}                    
                 edge from parent
                 node[above] {$1-p$}             
             }
             child {
                 node[circle, minimum width=1pt,fill, inner sep=0pt, label=right:
                     {$U(\mu-1,p)$}] {}
                 edge from parent
                 node[above] {$p$}              
             }
             edge from parent
             node[above] {$f_p$}}    
  edge from parent         
  node[above] {$\delta_A$}
     };
 \end{tikzpicture}
 \caption{Decidir si apostar en función de una utilidad}
 \end{figure}

  
  
   Se apuesta si: \\
   $E_p(U(\delta_A))=E_p[p\,U(\mu-1,p)+(1-p)U(-1,p)]\ge0$
 \end{enumerate}

 Algunas funciones de utilidad posibles:
 \begin{itemize}
  \item $U_\mu(x,p)=x(\frac{1}{\mu}-p)^2$\\
 
  Notese que: $p\,U_\mu(\mu-1,p)+(1-p)U_\mu(-1,p)$\\
 
  $(\hat p=\frac{1}{\mu})=(\hat p-p)^2(p\mu-1)$\\
 
  Me duele más mientras más alejado esté de un trato beneficioso y me produce mayor placer mientras mayor sea el beneficio del trato.
 
  \item $U_{\mu,a}(x,p)= \left\{ \begin{array}{lcc}
              ax(\hat p-p)^2 &   si  & p \le \hat p \\
              & &\\
              x (\hat p-p)^2 &  si & p>\hat p\\             
              \end{array}
  \right.$
 
  Notese que: \\
  \[U_\mu=U_{\mu,1}\]\\
  \[p\,U_{\mu,a}(\mu-1,p)+(1-p)U_\mu(-1,p)= \left\{ \begin{array}{lcc}
              a(\hat p-p)^2 (p\mu-1)&   si  & p \le \hat p \\
              & &\\
              (\hat p-p)^2(p\mu-1) &  si & p>\hat p\\             
              \end{array}
  \right.\]

  Me duele ``a'' veces más un trato perjudicial  que un trato beneficioso si me encuentro a la mis ma distancia que $\hat p$.
 
  \item $U_{\mu,a,b}=U_{\frac{\mu}{1+\mu b},a}$\\
 
  y considerar el problema de decisión con $\mu'=\frac{\mu}{1+\mu b}$.\\
 
  Si $\mu'=\frac{\mu}{1+\mu b}\rightarrow \hat p'=\hat p+b$.\\
 
  Los tratos empiezan a ser beneficiosos hasta que el menos sea $b\%$ más probable que ocurra el evento de lo que sería justo.\\
 
  {\bf Nota:} En un problema de decisión sin aversión a la distribución de probabilidades (o con probabilidades fijas) si se desea apostar en apuestas con un mínimo de ganancias esperadas igual a $b\%$ se debe comparar $\mu_p$ con $1+b$ (i.e. apostar $\leftrightarrow \mu_p \ge 1+b$).
 
 \end{itemize}

 \section{Decidir la cantidad de dinero a apostar}

 Supongamos que $\mu_p \ge 1$ y que existen 2 funciones de utilidad:
 \[U_1:\Re^+ \rightarrow \Re^+\]
 \[U_2:\Re^+ \rightarrow \Re^+\]

 La primera es la función de utilidad del dinero para las ganancias y la segunda es la utilidad del dinero para las pérdidas monetarias.\\

 Se harán las siguientes supuestos:

 \begin{enumerate}[(i)]
  \item $U_1(0)=U_2(0)=0$. $U_1$, $U_2$ no decrecientes, una vez cont. dif.
  \item $U'_1(0)>U'_2(0)$ (por lo tanto convendrá apostar).
  \item $\forall M>0$ fija $\displaystyle \lim_{x\rightarrow \infty} \frac{U_1(\mu x)}{U_2(x)}=0$.\\
  (Perder duele muchisimo más que ganar).
 \end{enumerate}
 El problema de decisión asociado a  determinar la cantidad óptima a postar es:(con $0<p<1$ fija y $\mu$ momio)\\

 \begin{figure}[ht]
  \begin{center}
 \begin{tikzpicture}[grow=right, sloped]
 \node[text width=4em, text centered] {$\square$}
 %%%%%%%%%%%%%%%%%%%%%%%%%%%%%%%%%%%%%%%%%%%%%%%
 %primero
 %%%%%%%%%%%%%%%%%%%%%%%%%%%%%%%%%%%%%%%%%%%%%%%%
 child{
     node[text width=4em, text centered] {\textbigcircle}        
             child {
                 node[circle, minimum width=1pt,fill, inner sep=0pt, label=right:
                     {$-U_2(x)$}] {}
                 edge from parent
                 node[above] {$1-p$}             
             }
             child {
                 node[circle, minimum width=1pt,fill, inner sep=0pt, label=right:
                     {$U_1((\mu-1)x)$}] {}
                 edge from parent
                 node[above] {$p$}              
             }    
  edge from parent         
  node[above] {$\delta_x$}
     };
 \end{tikzpicture} 
 \end{center}
 \caption{Árbol de probabilidad 4}
 \end{figure}



 $\rightarrow E_p[U(\delta x)]=pU_1((\mu-1)x)-(1-p)U_2(x)$\\

 Sea $f(x)=E_p[U(\delta x)]$\\

 Encontrar el óptimo es encontrar $x \ge 0$ que resuelva el problema: $\displaystyle \max_{x\ge0}f(x)$\\

 \[f'(x)=p(\mu-1)U'_1((\mu-1)x)-(1-p)U'_2(x)=0\]
 \[\frac{p(\mu-1)}{(1-p)}=\frac{U'_2(x)}{U'_1((\mu-1)x)}\]
 P.d.$$\exists \quad x^* \quad \cdot \ni \cdot \quad \frac{p(\mu-1)}{1-p}=\frac{U'_2(x)}{U'_1(\mu x)}$$

 \begin{enumerate}[(i)]

  \item $f'(0)=p(\mu-1)U'_1(0)-(1-p)U'_2(0)>p(\mu-1)U'_2(0)-(1-p)U'_2(0)$\\
 
  $\,\,\,\quad\quad=U'_2(0)(p\,\mu-1)\ge 0$\\
 
  Con $U_2'(0)\ge 0$ y $p\mu\ge0$\\
  Por tanto $f'(0)>0$
 
  \item $f(0)=0$
  \item $\displaystyle\frac{f(x)}{U_2(x)}=p\displaystyle\frac{U_1((\mu-1)x)}{U_2(x)}-(1-p)$\\
  $\rightarrow \displaystyle \lim_{x\rightarrow\infty}\frac{f(x)}{U_2(x)}=-(1-p)$\\
 
  $\rightarrow \exists \, x\,\,\cdot \ni \cdot \,\, \displaystyle\frac{f(x)}{U_2(x)}=-(1+p)+\varepsilon<0$\\
 
  $\rightarrow \exists\,x\,\,\cdot \ni \cdot \,\,f(x)<0$
  \begin{itemize}
   \item Por $T.V.M.\,\,\,\exists\,\, x'\in(0,x)\,\,\cdot \ni \cdot \,\,xf'(x')=f(x)-f(0)=f(x)<0$\\
   $\rightarrow f'(x')<0$
   \item T.V.I. $\exists\,\, x^*\in(0,x')\,\,\cdot \ni \cdot \,\,f'(x^*)=0$. i.e. $\displaystyle\frac{p(\mu-1)}{1-p}=\displaystyle\frac{U'_2(x)}{U'_1(\mu x)}$\\
  
   Como $f$ es primero creciente y en algún punto decreciente:\\
   $\rightarrow x\,\,\cdot \ni\cdot\,\,f'(x)=0$ es un maximizador.
  \end{itemize}
 \end{enumerate}

 Algunas funciones a considerar:
 \begin{itemize}
  \item $U_{1,\alpha}(x)=x^{\alpha}\qquad\qquad 0<\alpha<1$\\ 
  $U_2(x)=x$\\
 
  Compruébense los supuestos:
  \begin{enumerate}[(i)]
   \item $U_{1,\alpha}(0)=0=U_2(0)$, son crecientes y una vez dif.
   \item $U'_{1,\alpha}(0)=+\infty$, $U'_2(0)=1\qquad{\therefore \,\, U'_{1,\alpha}(0)>U'_2(0)}$
   \item $\forall \,\, \mu>0$\\
  
   $\displaystyle\lim_{x\rightarrow +\infty}\displaystyle\frac{U_{1,\alpha}(\mu x)}{U_2(x)}=\mu^{\alpha}\displaystyle\lim_{x\rightarrow +\infty}\displaystyle\frac{x^{\alpha}}{x}=\mu^{\alpha}\displaystyle\lim_{x\rightarrow +\infty}\displaystyle\frac{1}{x^{1-\alpha}}=0$\\
  
   Para una apuesta con probabilidad $p$ y momio $\mu$ el óptimo se da en:\\
  
   $\displaystyle{\frac{p(\mu-1)}{(1-p)}=\frac{U'_2(x)}{U'_{1,\alpha}((\mu-1)x)}=\frac{1}{\alpha((\mu-1)x)^{\alpha-1}}=\frac{1}{\alpha}(\mu-1)^{1-\alpha}x^{1-\alpha}}$\\\\
  
   $\rightarrow \left(\displaystyle\frac{\alpha p}{(1-p)}\right)(\mu-1)^{\alpha}=x^{1-\alpha}\rightarrow x^*=\left(\displaystyle\frac{\alpha p}{1-p}\right)^{\frac{1}{1-\alpha}}(\mu-1)^{\alpha/1-\alpha}$\\
  \end{enumerate}

  \item $U_{1,\alpha}(x)=x^{\alpha}\qquad\qquad 0<\alpha<1$\\
  $U_{2,\beta}(x)=x^{\beta}\qquad\qquad \beta \le1$\\
 
  Es fácil revisar los supuestos. Para una apuesta con probabilidad $p$ y momio $\mu$ el óptimo se da en:\\
 
  ${\displaystyle\frac{p(\mu-1)}{(1-p)}=\frac{\beta x^{\beta-1}}{\alpha(\mu-1)^{\alpha-1}x^{\alpha-1}}=\frac{\beta}{\alpha}(\mu-1)^{1-\alpha}x^{\beta-\alpha}}$\\
 
  $\rightarrow{\displaystyle\left(\frac{\alpha p}{\beta(1-p)}\right)(\mu-1)^{\alpha}=x^{\beta-\alpha}\rightarrow x^*=\left(\frac{\alpha p}{\beta(1-p)}\right)^{1/\beta-\alpha}(\mu-1)^{\alpha/\beta-\alpha}}$
 
  \item $U_1(x)=\ln (x)$\\
  $U_2(x)=x$\\
 
  Es fácil revisar los supuestos. Para una apuesta con probabilidad $p$ y momio $\mu$ el óptimo se da en:\\
 
  ${\displaystyle \frac{p(\mu-1)}{(1-p)}=\frac{1}{(\frac{1}{(\mu-1) x})}=(\mu-1)x\rightarrow x^*=\frac{p}{1-p}}$\\
 
  \item $U_{1,\alpha}(x)=1-e^{-\alpha x}\qquad\qquad \alpha \ge 1$\\
  $U_2(x)=x$\\
 
  Es fácil revisar los supuestos. Para una apuesta con probabilidad $p$ y momio $\mu$ el óptimo se da en:\\
 
 \[{\displaystyle\frac{p(\mu-1)}{1-p}=\frac{1}{\alpha e^{-\alpha(\mu-1)x}}\,\,\rightarrow \,\,\ln \left(\frac{\alpha p(\mu-1)}{(1-p)}\right)=\alpha(\mu-1)x}\]
 \[\qquad\qquad\qquad\qquad\qquad\qquad\rightarrow\,\, x^*={\displaystyle\frac{1}{\alpha (\mu-1)}\ln \left(\frac{\alpha p(\mu-1)}{(1-p)}\right)}\]
 \end{itemize}

 Otras tres funciones de utilidad a considerar:

 \begin{itemize}
  \item $U_{1,\alpha}(x)=\alpha x \qquad\qquad \alpha \ge 1$\\
  $U_2(x)=e^x-1$\\
 
  $\rightarrow {\displaystyle\frac{p(\mu-1)}{1-p}=\frac{e^x}{\alpha}}$\\
 
  $\rightarrow x^*=\ln \left(\displaystyle\frac{p(\mu-1)}{1-p}\right)+\ln (\alpha)$
 
  \item $U_1(x)=\ln (x)\qquad\qquad \alpha \ge 1$\\
  $U_2(x)=x^{\alpha}$\\
 
  $\rightarrow {\displaystyle\frac{p(\mu-1)}{1-p}=\frac{\alpha x^{\alpha-1}}{\frac{1}{(\mu-1)x}}=\alpha(\mu-1)x^{\alpha}}$\\
 
  $\rightarrow x^*=\left(\displaystyle\frac{p}{\alpha(1-p)}\right)^{1/\alpha}$
 
  \item $U_{1,\alpha}(x)=\tan^{-1}(x)$\\
  $U_{2,\alpha}(x)=\alpha x\qquad\qquad 0<\alpha\le 1$\\
 
  $\rightarrow \displaystyle\frac{p(\mu-1)}{1-p}=\alpha(1+(\mu-1)^2x^2)$\\
 
  $\rightarrow \displaystyle\frac{p\mu-p-\alpha(1-p)}{1-p}=\alpha(\mu-1)^2x^2$\\
 
  $\rightarrow \displaystyle\frac{p\mu-(1-\alpha)p-\alpha}{1-p}=\alpha(\mu-1)^2x^2$\\

  $\rightarrow x^*={\displaystyle\frac{1}{\sqrt{\alpha}(\mu-1)}\left(\frac{p\mu-(1-\alpha)p-\alpha}{1-p}\right)^{1/2}}$\\
 
  equivalentemente:  $x^*={\displaystyle\frac{1}{\mu-1}\left(\frac{p\mu-(1-\alpha)p-\alpha}{1-p}\right)^{1/2}}$\\
 
  Basta probar que $p\mu-(1-\alpha)p-\alpha \ge 0$\\
 
  $p\mu-(1-\alpha)p-\alpha \ge p\mu-(1-\alpha)-\alpha=p\mu-1 >0$\\
 
  $x^*$ está bien definido.
 \end{itemize}
 

%  \section{Ligas europeas de futbol}
% El nivel de juego de los clubes europeos es sorprendente, tanto en la cancha como fuera de ella los Clubes de futbol de las ligas europeas hacen las cosas mejor que ningún otro. Ofrecen partidos de alta calidad, con jugadas complejas y rápidas que proveen de un espectáculo como ningún otro. Además de que la infraestructura, adminsitración y los recursos financieros con los que cuentan son envidiables. Y es por estos motivos que sus niveles de audiencia y la cantidad de sus fanáticos han llegado a niveles impresionantes. Las ligas europeas son en el mundo del futbol: \emph{El modelo a seguir.}
%
%
%
% Además de todas las cualidades con las que cuentan estas ligas, se tiene una premisa muy interesante incita el enfoque en ellas: La consistencia que tienen los equipos más populares de cada liga para conseguir victorias sobre los equipos más modestos y su habilidad para siempre permanecer en los mejores lugares de la tabla de posiciones.
%
% \section{Ingeniería de Software}
%  \section{Contribuyendo al Internet con un granito de arena}
% La era de la información nos golpeo tan fuerte, que ahora es imposible la vida sin nuestros sistemas de información y nuestros dispositivos de conexión. Gracias a las computadoras y las redes, se ha redefinido nuestra imaginación, se ha creado un espacio que expande nuestra mente y nuestra capacidad, nuestras barreras se han alejado más y ahora nuestra conciencia como especie humana, crece en tasas inimaginables. Los monopolios de la información se han ido disolviendo, permitiendo a la sociedad una mayor participación y voz.
%
% Internet es un organismo vivo y hambriento, con el que convivimos de manera simbiótica y nos une como especie. Es un espacio de comunicación y entendimiento. Nunca pudo haber existido algo más majestuoso y poderoso. La verdadera trascendencia del ser, vendrá con la evolución de la conciencia de la sociedad.
%
% Con esta idea en mente fue diseñado y desarrollado Egobets, un sistema que aporta a la comunidad en internet información últil para la toma de mejores decisiones. Y a su vez, esta información, proviene del procesamiento de varias fuentes de información que otros aportan al internet. El ideal detrás: un conjunto de círculos virtuosos que pongan al alcance de cualquier persona la información más precisa y útil acerca de cualquier tema que se pueda pensar.
%

% Estas ventajas fueron más que suficientes para realizar el desarrollo en la nube. Además de que la realización de un nuevo prouecto utilizando nuevas tecnologías siempre aporta mayor emoción y reto al desarrollo de software.
%
% Patrones de diseño
%
% Bases de Datos no Relacionales
%
% Lenguajes:
% PHP
% js
% fortran
%
%









 %Panorama general de apuestas
\chapter{Back Office}
\section{Descripción General}

\graphicspath{{/Users/brunomedina/Dropbox/Tesis-Egobets/egobets-notas/resources/diagramas/}}
El ecosistema de Egobets consiste principalmente de cuatro piezas de software. Ver la figura~\ref{Fig:Sistemas}. En este capítulo se describirán las tres piezas que el usuario adminstrativo debe usar para poder a echar a andar toda la maquinaria detrás del sistema. A todo este conjunto de herramientas y programas que el usuario necesita para esta tarea se le conocerá como \emph{Back Office}.

\begin{figure}[!htb]\centering
   \begin {minipage}{1\textwidth}
     \frame{\includegraphics[width=\linewidth]{sistemas}}
     \caption{Diagrama de sistemas y usuarios}\label{Fig:Sistemas}
   \end{minipage}
\end{figure}

El Sistema de recopilación de información y estadísiticas de los partidos (\emph{Sistema de recopilación}), el \emph{Portal administrativo} y el \emph{Portal público} corren bajo una arquitectura cliente servidor; mientras que el Sistema de estimación de probabilidades (\emph{Sistema de estimación}) corre en un ordenador personal. 

Grosso modo el proceso que se lleva a cabo en el \emph{Back Office} para alimentar el \emph{Portal público} (Ver la figura~\ref{Fig:flujo}), se puede describir de la siguiente manera:
\begin{enumerate}
	\item A través del \emph{Sistema de recopilación} los administradores descargan de la página de Internet de ESPN los resultados de todos los partidos de la temporada junto con la información de los próximos partidos por jugar  de cada una de las ligas Europeas.
	\item Los datos recopilados permiten a los administradores generar un conjunto de archivos de texto con toda la información de los resultados de los últimos partidos y las fechas de los próximos partidos.
	\item Los administradores usan estos archivos para alimentar el \emph{Sistema de estimación} y calcular los pronósticos de los próximos partidos y las probabilidades de los resultados.
	\item Se obtienen los archivos que contienen la información de los próximos partidos así como la información de los equipos por liga y su desempeño en la temporada en curso.
	\item En el \emph{Portal administrativo} se ingestan los archivos obtenidos con la información de los próximos partidos, resultados de partidos anteriores y las estadísticas de los equipos en la temporada en curso.
	\item Finalmente, con la nueva información ingresada, los usuarios podrán disfrutar en el \emph{Portal público} sus recomendaciones peronalizadas de apuestas.
\end{enumerate}

\begin{figure}[!htb]\centering
   \begin {minipage}{1\textwidth}
     \frame{\includegraphics[width=\linewidth]{flujo}}
     \caption{Diagrama de flujo de información}\label{Fig:flujo}
   \end{minipage}
\end{figure}


\section{Sistema de recopilación de información y estadísticas de los partidos}
A grandes rasgos el sistema de recopilación recupera todos los partidos que se juegan por temporada en cada una de las ligas\footnote{Egobets se enfoca en las ligas: alemana, españona, francesa, inglesa e italiana}. Para esto se dividirá este Sistema en los siguientes módulos:

\begin{enumerate}
	\item Recuperación de equipos.
	\item Recuperación de fechas de partidos próximos.
	\item Recuperación de información de partidos jugados.
	\item Generación de archivos con resultados.
\end{enumerate}

\section{Sistema de estimación de probabilidades}
\section{Portal administrativo}
\graphicspath{{/Users/brunomedina/Dropbox/Tesis-Egobets/egobets-notas/resources/admin/}}

\subsection{Inicio de Sesión}
Se ingresa al sistema a través la dirección de internet: 
\begin{tightcenter}
	\textbf{https://admin.egobets.com}
\end{tightcenter}Se presenta la pantalla de inicio de sesión donde se introcue el nombre de usuario y contraseña. Véase las figuras~\ref{Fig:Login1} y~\ref{Fig:Login2}
\pagebreak
\begin{figure}[!htb]\centering
   \begin{minipage}{0.49\textwidth}
     \frame{\includegraphics[width=\linewidth]{login}}
     \caption{Login}\label{Fig:Login1}
   \end{minipage}
   \begin {minipage}{0.49\textwidth}
     \frame{\includegraphics[width=\linewidth]{login-lleno}}
     \caption{Ingreso de datos}\label{Fig:Login2}
   \end{minipage}
\end{figure}

Al dar click en el botón de \underline{Entrar}, se habrá iniciado sesión, y se está habilitado para comenzar a trabajar.

Una vez que se haya terminado de usar el sistema, se debe cerrar la sesión, lo cual se puede hacer dando click sobre el botón de Salir. Véase figura~\ref{Fig:Logout}

\begin{figure}[!htb]\centering
   \begin {minipage}{0.49\textwidth}
     \frame{\includegraphics[width=\linewidth]{logout}}
     \caption[Cerrar Sesión]{Cerrar Sesión}\label{Fig:Logout}
   \end{minipage}
\end{figure}

Al hacerlo, se terminará de manera segura la sesión .
\subsection{Ingesta}
A través de este módulo se alimenta el sistema con la información necesaria para que la aplicación trabaje correctamente.

Lo elementos que se ingresan en este módulo son:

\begin{itemize}
\item Próximos partidos
\item Resultados de partidos anteriores
\item Estadísticas de los equipos en la temporada
\end{itemize}

Al entrar a esta sección, se encuentra un campo para seleccionar el tipo de ingesta que se quiere realizar, en este caso hay dos opciones:

\begin{itemize}
\item Partidos  
\item Equipos
\end{itemize}
Véase figura~\ref{Fig:ingesta}

\begin{figure}[!htb]\centering
   \begin {minipage}{0.8\textwidth}
     \frame{\includegraphics[width=\linewidth]{ingesta}}
     \caption[Subir y procesar archivos]{Subir archivos y procesarlos\footnotemark}
	 \label{Fig:ingesta}
   \end{minipage}
\end{figure}
\footnotetext{Los tipos de archivos que el sistema procesa son archivos de texto plano, en codificación UTF-8 con ó sin BOM, en formato CSV, en el que la separación de valores se logra mediante tabulaciones, ó el caracter ``\textbackslash t'', y cada línea se termina con el caracter de nueva línea de Unix, es decir ``\textbackslash n''; el archivo debe contener todos los campos y estar en el orden indicado.}



\subsubsection{Formato de archivo para Partidos}

\begin{enumerate}
	\item Identificador único del equipo local
	\item Identificador único del equipo visitante
	\item Marcador local
	\item Marcador visitante
	\item Probabilidad local
	\item Probabilidad empate
	\item Probabilidad visitante
	\item Fecha del partido, expresada en segundos desde la época Unix (1 de enero
	de 1970) en UTC.
\end{enumerate}

\subsubsection{Formato de archivo para Equipos}

\begin{enumerate}
	\item Identificador único del equipo, 
	\item Variable local 1,
	\item Variable local 2,
	\item Variable local 3,
	\item Variable local 4,
	\item Variable local 5,
	\item Variable local 6,
	\item Variable local 7,
	\item Variable de ataque 1-1, 
	\item Variable de ataque 1-2,
	\item Variable de ataque 1-3, 
	\item Variable de ataque 1-4, 
	\item Variable de ataque 1-5, 
	\item Variable de ataque 1-6, 
	\item Variable de ataque 1-1, 
	\item Variable de ataque 1-2, 
	\item Variable de ataque 1-3, 
	\item Variable de ataque 1-4, 
	\item Variable de ataque 1-5, 
	\item Variable de ataque 1-6,
	\item Variable de ataque 2-1,
	\item Variable de ataque 2-2,
	\item Variable de ataque 2-3,
	\item Variable de ataque 2-4,
	\item Variable de ataque 2-5, 
	\item Variable de ataque 2-6, 
	\item Variable de ataque 3-1, 
	\item Variable de ataque 3-2, 
	\item Variable de ataque 3-3, 
	\item Variable de ataque 3-4, 
	\item Variable de ataque 3-5, 
	\item Variable de ataque 3-6, 
	\item Variable de defensa 1-1, 
	\item Variable de defensa 1-2, 
	\item Variable de defensa 1-3, 
	\item Variable de defensa 1-4, 
	\item Variable de defensa 1-5, 
	\item Variable de defensa 1-6, 
	\item Variable de defensa 1-1,
	\item Variable de defensa 1-2, 
	\item Variable de defensa 1-3, 
	\item Variable de defensa 1-4, 
	\item Variable de defensa 1-5, 
	\item Variable de defensa 1-6, 
	\item Variable de defensa 2-1, 
	\item Variable de defensa 2-2, 
	\item Variable de defensa 2-3, 
	\item Variable de defensa 2-4, 
	\item Variable de defensa 2-5, 
	\item Variable de defensa 2-6, 
	\item Variable de defensa 3-1, 
	\item Variable de defensa 3-2, 
	\item Variable de defensa 3-3, 
	\item Variable de defensa 3-4, 
	\item Variable de defensa 3-5, 
	\item Variable de defensa 3-6, 
	\item Variable de posesión
\end{enumerate}

\subsubsection{Partidos}
\begin{figure}[!htb]\centering
   \begin {minipage}{1\textwidth}
     \frame{\includegraphics[width=\linewidth]{partidos}}
     \caption{Partidos procesados por el sistema}
	 \label{Fig:partidos}
   \end{minipage}
\end{figure}

Para subir la información de los partidos de la jornada que comienza, se selecciona del menú la opción de \textbf{Partidos}, y se presiona el botón \underline{Seleccionar archivo} donde se elige el archivo correspondiente.
Una vez seleccionado el archivo a ingestar y el tipo de datos que contiene, se oprime el botón de \underline{Procesar}, lo cual comienza el proceso de ingesta.\footnote{Una vez que comenzado el proceso de ingesta (ya sea de equipos ó partidos), se tienen sólo 10 minutos para verificar que los datos sean correctos. De no hacerlo, el sistema no procesará el archivo hasta que se suba nuevamente}

Con el archivo ya procesado, se puede verificar la interpretación que el sistema realizó del archivo. Se pueden observar en pantalla los siguientes datos:
\begin{itemize}
\item Identificador único y nombre del equipo local
\item Identificador único y nombre del equipo visitante
\item Marcadores de equipo local y visitante
\item Probabilidades de local, empate y visitante
\item Fecha en la que se llevará a cabo el partido
\end{itemize}

Si hay algún error en la información se puede presionar \underline{Cancelar} e intentarlo nuevamente, si la información es la correcta se oprime el botón de \underline{Aceptar}.


\subsubsection{Equipos}
\begin{figure}[!htb]\centering
   \begin {minipage}{1\textwidth}
     \frame{\includegraphics[width=\linewidth]{equipos}}
     \caption{Actualizando datos de los equipos}
	 \label{Fig:equipos}
   \end{minipage}
\end{figure}

Para la ingesta de datos de los equipos se selecciona la pestaña de \underline{Equipos} en la pestaña y luego se presiona el botón \textbf{Seleccionar archivo}.
Una vez que se seleccione el archivo a ingestar y el tipo de datos que contiene, se oprime el botón de \underline{Procesar}, para comenzar el proceso de ingesta.
Cuando el archivo termina de ser procesado el sistema presentará la interpretación del archivo, donde se podrán verificar los siguientes datos:
\begin{itemize}
\item Nombre del equipo
\item Indicadores de ataque y promedio de ataques:
	\begin{itemize}
		\item Medio Centro
		\item Delanteros
		\item Definición
	\end{itemize}
\item Indicadores de defensa y promedio de defensas:
	\begin{itemize}
		\item Medio centro,
		\item Defensa
		\item Portero
		\item Posesión
	\end{itemize}
\end{itemize}

Si hay algún error en la información se puede presionar \underline{Cancelar} e intentarlo nuevamente, si la información es la correcta se oprime el botón de \underline{Aceptar}.


\subsubsection{Resultados Anteriores}

Al dar clic en el botón de \underline{Resultados Anteriores} se pueden ver y modificar los resultados de los partidos de la semana pasada. En esta pantalla se actualizan los marcadores, al terminar se da click en \underline{Guardar Resultados}.

\begin{figure}[!htb]\centering
   \begin {minipage}{0.64\textwidth}
     \frame{\includegraphics[width=\linewidth]{resultados-anteriores}}
     \caption[Actualizar resultados Anteriores]{Actualizando marcadores de la liga italiana}
	 \label{Fig:Resultados-anteriores}
   \end{minipage}
\end{figure}

Si el marcador de un partido que ya tenía resultado se deja en blanco no será modificado al guardar y se mostrará el resultado que tenía previamente.

\subsection{Usuarios}

Se muestran de quince en quince todos los usuarios inscritos a Egobets, para ver los siguientes quince usuarios se da click en el botón \underline{Siguiente}. Cada usuario tiene un botón de \underline{Detalles} y \underline{Eliminar}.
\begin{figure}[!htb]\centering
   \begin {minipage}{1\textwidth}
     \frame{\includegraphics[width=\linewidth]{usuarios}}
     \caption{Listado de Usuarios}
	 \label{Fig:usuarios}
   \end{minipage}
\end{figure}

\subsubsection{Detalles de Usuario}

El botón Detalles en el listado presenta la información más detallada del usuario. Aquí se puede ver su información y preferencias:

\begin{figure}[!htb]\centering
   \begin {minipage}{1\textwidth}
     \frame{\includegraphics[width=\linewidth]{detalle-usuario}}
     \caption{Vista del detalle de usuario}
	 \label{Fig:Detalle-usuario}
   \end{minipage}
\end{figure}

\begin{itemize}
	\item \textbf{Historial.} Indica las últimas ganancias y pérdidas por jornada
	\item \textbf{Perfil de riesgo.} Despendiendo de la encuesta realizada por usuario se tiene su adversidad al riesgo.
	\item \textbf{Casas de apuesta.} En el sistema se tienen varias Casas que proporcionan distintos momios para los partidos.
	\item \textbf{Favoritos.} Los equipos favoritos del usuario
	\item \textbf{Transacciones realizadas.} Los últimos pagos realizados.
	\item \textbf{Usuario desde.} Tiempo que lleva como usuario de Egobets.
	\item \textbf{Estatus de actividad.} Al ser un sistema de paga los usuarios pagan por jornada para recibir la asesoría de apuestas.
		\begin{itemize}
			\item Activo: el usuarios está recibiendo recomendaciones.
			\item Inactivo: el usuario no está recibiendo recomendaciones.
		\end{itemize}
	\item \textbf{Jornadas restantes.} La cantidad de Jornadas que el usuario va seguir recibiendo asesorías.
	\item \textbf{Idioma.} En que lenguaje lee el portal el usuario (Inglés o Español)
\end{itemize}

\textbf{Acciones Administrativas}
\begin{itemize}
\item \textbf{Aumentar/Restar una Jornada.}
Una jornada le permite al usuario recibir la asesoría de los siguientes partidos.
Los administradores del sistema le pueden otorgar o quitar a los usuarios jornadas con tan solo click en el botón.

\item \textbf{Contactar.}
Permite al administrador enviar un correo desde su progama predeterminado de correo al usuario.

\item \textbf{Eliminar.} Todos los datos del usuario son eliminados del sistema.
\end{itemize}

\begin{figure}[!htb]\centering
   \begin {minipage}{0.5\textwidth}
     \frame{\includegraphics[width=\linewidth]{eliminar-usuario}}
     \caption[Confirmar la eliminación de un usuario]{Confirmar la eliminación de un usuario\footnotemark}
	 \label{Fig:Eliminar-usuario}
   \end{minipage}
\end{figure}

\footnotetext{La información de los usuarios eliminados no podrá ser rescatada.}

\subsection{Pagos}
Los pagos de los usuarios se cambian por la sugerencia de apuestas de una jornada. En esta sección se muestra un listado de las transacciones monetarias más recientes y su información general.

\begin{figure}[!htb]\centering
   \begin {minipage}{1\textwidth}
     \frame{\includegraphics[width=\linewidth]{Transacciones}}
     \caption{Listado con los últimos pagos realizados}
	 \label{Fig:Transacciones}
   \end{minipage}
\end{figure}

Los detalles de las transacciones son:
\begin{itemize}
	\item Fecha en la que la transacción se inicio.
	\item Número de transacción en la cuenta de PayPal de Egobets.
	\item Nombre y el correo del usuario que realiza la transacción, al dar clic sobre su nombre seremos dirigidos a la información detallada de dicho usuario
	\item Cantidad de jornadas por las que se realiza la transacción
	\item Cantidad monetaria por la que se realiza la transacción
	\item Estatus de la transacción:
	\begin{itemize}
		\item Pendiente: se ha iniciado la transacción para la compra de jornadas, sin embargo aun no ha concluido.
		\item Pagada: se realizó exitosamente y las jornadas han sido agregadas al usuario.
	\end{itemize}
\end{itemize}

\subsection{Estadísticas}

Esta sección muestra las estadísticas y gráficas a los usuarios administrativos con información relevante de: ganancias y pérdidas de los usuarios, resultados de las predicciones, preferencias de los usuarios, pagos y partidos.

\subsubsection{Resultados Netos}

Indica el promedio de las ganancias y pérdidas de todos los usuarios en las últimas cinco jornadas, esta información se puede ver de manera porcentual o en cantidad neta.
Véase figura~\ref{Fig:mayor-ganancia}
\begin{figure}[!htb]\centering
   \begin {minipage}{0.4\textwidth}
     \frame{\includegraphics[width=\linewidth]{mayor-ganancia}}
     \caption{Ganancias y pérdidas de los usuarios}
	 \label{Fig:mayor-ganancia}
   \end{minipage}
\end{figure}


\textbf{Mayor Pérdida.}
Indica la mayor pérdida porcentual que se ha dado en la última jornada y al dar click presenta el perfil de dicho usuario.

\textbf{Mayor Ganancia.}
Indica la ganancia porcentual mayor que se ha dado en la última jornada y al dar click presenta el perfil de dicho usuario.


\subsubsection{Usuarios y sus Datos}

\textbf{Total.} Número total de usuarios registrados y al dar click presenta la sección de \underline{Usuarios}.

\textbf{Nuevos.} Número de usuarios registrados recientemente y al dar click presenta la sección de \underline{Usuarios}.

Véase figura~\ref{Fig:nuevos-usuarios}
\begin{figure}[!htb]\centering
   \begin {minipage}{0.4\textwidth}
     \frame{\includegraphics[width=\linewidth]{nuevos-usuarios}}
     \caption{Usuarios recién inscritos}
	 \label{Fig:nuevos-usuarios}
   \end{minipage}
\end{figure}

Además, el sistema muestra la siguiente información general, porcentaje de usuarios que:
\begin{itemize}
	\item Usan Facebook para conectarse a Egobets
	\item Usan apuestas dobles
	\item Usan reserva
	\item Ven Egobets en inglés
	\item Se encuentran activos.
\end{itemize}
Véase figura~\ref{Fig:graficas-usuarios}
\begin{figure}[!htb]\centering
   \begin {minipage}{1\textwidth}
     \frame{\includegraphics[width=\linewidth]{graficas-usuarios}}
     \caption{Datos estadísticos de los usuarios}
	 \label{Fig:graficas-usuarios}
   \end{minipage}
\end{figure}
 
\subsubsection{Pagos Recibidos}

\begin{figure}[!htb]\centering
   \begin {minipage}{0.5\textwidth}
     \frame{\includegraphics[width=\linewidth]{ultimos-pagos}}
     \caption{Pagos más recientes}
	 \label{Fig:ultimos-pagos}
   \end{minipage}
\end{figure}

\textbf{Última Semana.}
Se representan las ganancias monetarias que obtenidas durante la última semana. Al dar click se muestra la sección de \textbf{Pagos}.

\textbf{Transacciones en total.}
Indica el número de transacciones que se han realizado durante todo el tiempo del sistema.

\subsubsection{Partidos y Predicciones}

\begin{figure}[!htb]\centering
   \begin {minipage}{0.5\textwidth}
     \frame{\includegraphics[width=\linewidth]{partidos-acertados}}
     \caption{Partidos acertados}
	 \label{Fig:partidos-acertados}
   \end{minipage}
\end{figure}

\textbf{Aciertos}
Presenta la cantidad de partidos de esta semana y al dar clic nos lleva a la sección de \underline{Ingesta}.

\textbf{Aciertos}
Representan con una gráfica la cantidad de aciertos obtenidos en las predicciones hechas en partidos pasados. Al presionarla se dirige el navegador a los \underline{Resultados Anteriores} dentro de la sección de \underline{Ingesta}.

\subsection{Correos}

En esta sección se puede enviar correos a un subconjunto de usuarios registrados en Egobets. Los mensajes deberán ser escritos en Español y en Inglés para que el correo recibido dependa del lenguaje elegido por el usuario al crear su cuenta. Los grupos de usuarios con los que nos se puede comunicar son:

\begin{itemize}
	\item Todos los usuarios
	\item Usuarios activos: aquellos que tienen jornadas pagadas
	\item Usuarios inactivos: aquellos que ya no tienen jornadas pagadas
	\item Usuarios registrados: aquellos que se registraron pero no han confirmado su correo
\end{itemize}
Véase figura~\ref{Fig:correos}

\begin{figure}[!htb]\centering
   \begin {minipage}{0.8\textwidth}
     \frame{\includegraphics[width=\linewidth]{correos}}
     \caption{Comunicación con los usuarios}
	 \label{Fig:correos}
   \end{minipage}
\end{figure}

Para redactar los textos, se debe usar la sintaxis de Markdown\footnote{El hipervínculo de \underline{Markdown} redirige a una página dónde se puede aprender sobre el uso de esta sintaxis.}.
Se pueden usar textos de reemplazo cuándo se quieran personalizar los mensajes, para esto basta con utilizar las palabras clave: \{nombre\}, \{correo\} y \{id\}, las cuales el sistema sustituirá, al momento de mandar el correo, por los valores correspondientes para cada usuario.




 %Egobets
\chapter{Sistema de Egobets.com}
\label{chap:software}
\graphicspath{{/Users/brunomedina/Dropbox/Tesis-Egobets/egobets-notas/resources/diagramas/}}

Egobets.com proporciona al cliente los servicios de asesoría de apuestas personalizada a través de un portal web usable, práctico y profesional. En este capítulo se presenta el sistema desarrollado con los fundamentos teóricos descritos en los capítulos anteriores, una verdadera aplicación computacional de las matemáticas

% Se describen las cuatro piezas de software desarrolladas que conforman en su totalidad el sistema de Egobets. Las tres primeras permiten al usuario adminstrativo echar a andar toda la maquinaria. Y la última pieza, conocida como e
% A todo este conjunto de herramientas y programas que el usuario necesita para esta tarea se le conocerá como \emph{Back Office}.




\section{Diseño y arquitectura}
\label{sec:design}
Egobets.com es un sistema con arquitectura cliente servidor montado sobre una máquina virtual con sistema operativo Linux en la nube de ``Amazon Web Services (AWS)''. En esta sección se presentan los diagramas que describen la arquitectura del sistema, se muestran las ventajas de utilizar el computo en la nube, se presentan las tecnologías más relevantes involucradas en el sistema y también se expone el patrón de diseño Modelo Vista Controlador junto con su respectiva representación gráfica de base de datos.
\subsection{Ventajas de correr Egobets en la nube}
\begin{figure}[!htb]\centering
   \begin {minipage}{1\textwidth}
     \includegraphics[width=\linewidth]{diagrama-portal-publico}
     \caption{Arquitectura Cliente Servidor sobre Nube de AWS}\label{Fig:diagrama-portal-publico}
   \end{minipage}
\end{figure}

El sistema usa la nube de AWS para ofrecer sus servicio a los usuarios (Vease la imagen~\ref{Fig:diagrama-portal-publico}), más aún se puede decir que el software corre en un esquema tipo ``SaaS''\footnote{Software as a Service. ``Es el más conocido de los niveles de cómputo en la nube. El SaaS es un modelo de distribución de software que proporciona a los clientes el acceso a éste a través de la red (generalmente Internet). De esta forma, ellos no tienen que preocuparse de la configuración, implementación o mantenimiento de las aplicaciones, ya que todas estas labores se vuelven responsabilidad del proveedor. Las aplicaciones distribuidas a través de un modelo de Software como Servicio pueden llegar a cualquier empresa sin importar su tamaño o ubicación geográfica.'' \cite{godinez2010nube}.}, esto implica que el usuario simplemente ingresa a su cuenta en un navegador de internet y puede ver las asesorías para sus apuestas.
Del artículo ``Cómputo en Nube: Ventajas y Desventajas'' de Martínez y Gutiérrez \cite{godinez2010nube}  se retoman las siguentes ventajas de este paradigma:
\begin{itemize}
	\item \textbf{Costos.} Podría ser la ventaja más atractiva que presenta el cómputo en la nube, y si no lo es, al menos es la más evidente de todas las que ofrece esta tecnología. Al dejar la responsabilidad de la implementación de la infraestructura al proveedor, el cliente no tiene que preocuparse por comprar equipos de cómputo, capacitar personal para la configuración y mantenimiento de éstos, y en algunos casos, por el desarrollo del software. Además el usuario de estos servicios únicamente paga por los recursos que utiliza, permitiéndole diseñar un plan de pago normalmente a partir del tiempo en que éste se utiliza (memoria, procesamiento, almacenamiento). Para Egobets, esta cualidad es vital, ya que en el comienzo después de haber implantado el software, la cantidad de usuarios es mínima y los ingresos también. Conforme crece la bolsa de clientes también irá creciendo la potencia del servidor.

	\item \textbf{Competitividad.} Al no tener que adquirir equipos costosos, las pequeñas empresas pueden tener acceso a las más nuevas tecnologías a precios a su alcance pagando únicamente por consumo. De este modo las organizaciones de cualquier tipo podrían competir en igualdad de condiciones en áreas de TI con empresas de cualquier tamaño. La ventaja competitiva no está en aquel que tiene los recursos de cómputo sino en quien los emplea mejor. En particular a Egobets le permite utilizar este tipo de tecnología a la par de otros sistemas gigantes que tienen mucho más tiempo en el mercado y mucho mayores ingresos.

	\item \textbf{Disponibilidad.} El proveedor está obligado a garantizar que el servicio siempre esté disponible para el cliente. En este sentido, la virtualización juega un papel fundamental, ya que el proveedor puede hacer uso de esta tecnología para diseñar una infraestructura redundante que le permita ofrecer un servicio constante de acuerdo a las especificaciones del cliente. A manera anecdótica, en Egobets se tuvo un problema con uno de los discos duros de un servidor, bastó con crear una nueva máquina virtual de la imagen que ya se poseía y clonar el código necesario, en menos de cinco minutos el servicio estaba de vuelta en línea.


	\item \textbf{Abstracción de la parte técnica.} Como se mencionó al hablar de costos, el cómputo en la nube permite al cliente la posibilidad de olvidarse de la implementación, configuración y mantenimiento de equipos; transfiriendo esta responsabilidad al proveedor del servicio. En Egobets, jamas se ha tenido que realizar ningún tipo de mantenimiento a ningún equipo de hardware de los servicios en la nube.

	\item \textbf{Acceso desde cualquier punto geográfico.} El uso de las aplicaciones diseñadas sobre el paradigma del cómputo en la nube puede ser accesible desde cualquier equipo de cómputo en el mundo que esté conectado a Internet. El acceso normalmente se hace desde un navegador web, lo que permite a la aplicación ser utilizada no únicamente desde una computadora de escritorio o una computadora portátil, sino que va más allá, permitiendo al usuario hacer uso de la aplicación incluso desde dispositivos móviles como smartphones. El sistema montado en Egobets.com, por ejemplo, no tiene ninguna restricción hacia ningún país del mundo.

	\item \textbf{Escalabilidad.} El cliente no tiene que preocuparse por actualizar el equipo de cómputo sobre el que se está corriendo la aplicación que utiliza, ni tampoco por la actualización de sistemas operativos o instalación de parches de seguridad, ya que es obligación del proveedor del servicio realizar este tipo de actualizaciones. Además, éstas son transparentes para el cliente, por lo que la aplicación debe de continuar disponible para el usuario en todo momento aún cuando se esté realizando el proceso de actualización del lado del proveedor. Las actualizaciones y nuevas funcionalidades son instaladas prácticamente de manera inmediata. Si se quisiera ampliar el poder de cómputo de los servidores de Egobets.com, bastaría con realizar un par de clicks y esperar unos minutos a que el servidor se auto-escale.

Es por estas razones que Egobets funciona tan bien en el paradigma del computo en la nube.

\end{itemize}

\subsection{Servidor LNNP}

Este acrónimo respresenta un sistema de infraestructura de internet\footnote{LNNP viene a retomar el acrónimo LAMP (Linux, Apache, MySQL y PHP) que será descrito en la sección \ref{sec:inserting-data}.} que utiliza las siguientes tecnologías:
\begin{itemize}
	\item \textbf{L}inux, el sistema operativo.
	\item \textbf{N}ginx, el servidor Web.
	\item \textbf{N}oSQL, la Base de datos.
	\item \textbf{P}HP, el lenguaje de programación.
\end{itemize}

La máquina virtual de AWS corre Ubuntu LTS 12.04, una versión del sistema operativo creado por Linux Torvalds, \textbf{Linux} \cite{torvalds2001just}. Algunas de las ventajas de usar Linux son:
		\begin{itemize}
 			\item El costo de licencia es gratuito y su uso no tiene algún otro costo monetario.
 			\item Hay miles de aplicaciones libres para hacer más robusto el servidor.
 			\item Tener las aplicaciones en sus últimas versiones, bien configuradas y aplicar los parches de manera inteligente, garantizan que el servidor se encuentre seguro y sea funcional.
 		\end{itemize}
		
Encima del sistema operativo, tenemos \textbf{Nginx}. Un servidor Web de alto performance creado por Igor Sysoev para el sitio \emph{www.rambler.ru}, el segundo sitio más grande de Rusia \cite{reese2008nginx}. Nginx es capaz de servir más peticiones por segundo y con menos recursos que sus competidores, gracias a su arquitectura. Grosso modo, consiste en un proceso maestro que delega a sus procesos \emph{trabajadores} toda al carga de trabajo. Cada trabajador maneja varias solicitudes de manera asíncrona utilizando una funcionalidad especial del kernel de Linux\footnote{Se puede leer más al respecto de como funciona internamente Nginx en Zhu \cite{zhu2010nginx}}. Esto permite a Nginx, manejar un gran número de solicitudes simultáneas  con muy poca sobrecarga.

A un lado, se tiene la base de datos No-SQL. A \textbf{MongoDB}, derivado de la palabra \emph{humongous}, Membrey \cite{membrey2010definitive} le describe como una nueva especie de base de datos carente de conceptos de tablas, esquemas, SQL, o renglones. No tiene transacciones, joins, llaves externas, o cualquier otra de las características que suelen causar dolores de cabeza matutinos. En pocas palabras, MongoDB es una base de datos orientada a documentos, optimizada para ser veloz, escalable y fácil de integrar con cualquier lenguaje.

Finalmente el lenguaje en el que está programado el portal es \textbf{PHP}, cuyo acrónimo recursivo significa: \emph{PHP Hypertext Preprocessor}. Según su página Web \cite{phpWeb} es un lenguaje de scripting, el cual puede ser embebido dentro de páginas HTML. Gran parte de su sintaxis fue tomada de C, Java y Perl con un par de características específicas propias de PHP. El objetivo del lenguaje es permitir a desarrolladores Web escribir páginas generadas dinámicamente con agilidad.
PHP está enfocado principalmente a la programación de scripts del lado del servidor, por lo que se puede hacer cualquier cosa como recopilar datos de formularios, generar páginas con contenidos dinámicos, o enviar y recibir cookies. Aunque PHP puede hacer mucho más.
				 Ventajas:
	\begin{itemize}
		\item No depende del sistema operativo. Esto se debe a que corre al ser llamado por el servidor web y PHP corre sobre la mayoría de Servidores web, incluyendo Apache, IIS, lighthttpd, nginx y muchos otros.
		\item No se limita a genera HTML. Puede crear por ejemplo: imágenes, ficheros PDF e incluso películas Flash al vuelo. También puede generar archivos de texto y guardarlos en el sistema de archiso del sistema operativo.
		\item Permite la conexión casi cualquier Base de datos como: MySQL, SQLite, PostgreSQL, Mongo, Mssql, IBM DB2, entre muchas otras.
		\item PHP también cuenta con soporte para comunicarse con otros servicios usando protocolos tales como LDAP, IMAP, SNMP, NNTP, POP3, HTTP, COM (en Windows) y muchos otros.
		\item Existen muchas otras extensiones interesantes, las cuales están organizadas alfabéticamente y por categoría.
		\item Corre bajo la licencia Pública General de GNU. Por lo que se puede ser modificado y ser utilizado por cualquiera.
	\end{itemize}


\subsection{Patrón de diseño MVC}

El portal público de Egobets.com está desarrollado en una de las arquitecturas más utilizadas en los sistemas de información, Modelo Vista Controlador (MVC) \cite{alfredo2005ingenieria}. Esta arquitectura se basa en tres dimensiones principales: \emph{Modelo} correspondiente a la información, \emph{Vista} correspondiente a la presentación o interacción con el usuario y \emph{Control} correspondiente al comportamiento. El sistema utiliza esta arquitectura a través de: \emph{CodeIgniter}.

\begin{figure}[!htb]\centering
   \begin {minipage}{1\textwidth}
     \includegraphics[width=\linewidth]{mvc-architecture-mongo}
     \caption{Patrón de diseño Modelo Vista Controlador (MVC)}\label{Fig:mvc}
   \end{minipage}
\end{figure}

CodeIgniter es un framework\footnote{Es una estructura de software compuesta de componentes personalizables e intercambiables para el desarrollo de una aplicación. En otras palabras, un framework se puede considerar como una aplicación genérica incompleta y configurable a la que se le puede añadir las últimas piezas para construir una aplicación concreta.} de PHP que ahorra tiempo en le programación, robustece tu sistema y permite al programador alcanzar un grado mayor de sofisticación en su código. Uno de los puntos interesantes de este framework es que utiliza el patrón de diseño MVC, este patrón fue descrito por el noruego Trygve Reenskaug en 1979.
	
	De su página Web \cite{codeigniterWeb} se pueden destacar las siguientes propiedades:
	\begin{itemize}
		\item Tamaño pequeño. CodeIgniter 2.2 tiene una descarga 2.2MB, incluyendo la guía del usuario.
		\item Documentación clara. La guía que se incluye cuenta con guía y tutoriales para empezar a trabajar de manera muy práctica.
		\item Compatibildad con casi cualquier servicio de alojamiento. Sólo necesita PHP 5.1.6 y tiene soporte con las bases de datos más comunes incluído MySQL.

		\item Casi no necesita configuración. Todas las variables y opciones de configuración vienen predefinidas a los estandares convenidos en internet.

	\end{itemize}
	
	
	Sobre el libro de Upton \cite{upton2007codeigniter} se tiene una aproximación a este patrón de diseño en CodeIgniter:
	\begin{itemize}
		\item Modelos, son objetos que representan los datos. Estos objetos reflejan las tabla de la base de datos y pueden modificarla conforme sea requerido. Los modelos también realizan operaciones a los datos según sea necesario.
		\item Vistas, reflejan el estado del modelo. Son las responsables de desplegar la información al usuario final. En este caso específico, todas las vistas son representaciones HTML del contenido.
		\item Controladores, ofrecen opciones para cambiar el estado del modelo. Son los encargados de consultar los modelos. Proveen a las vistas los datos dinámicos a mostrar.
	\end{itemize}
	


\begin{figure}[!htb]\centering
   \begin {minipage}{1\textwidth}
     \includegraphics[width=\linewidth]{mongoDB/schema/bd-egobets}
     \caption{Representación UML de las colecciones de la base de datos de Egobets}\label{Fig:db-egobets}
   \end{minipage}
\end{figure}


\section{Servicios}
\label{sec:services}

\begin{enumerate}
	\item \emph{Recomendación personalizada de apuestas en fútbol:} Cada persona es diferente y debe ser tratada de forma única, en Egobets se determina el perfil de riesgo de cada usuario mediante una encuesta y se le recomienda apuestas a su medida, de tal forma que pueda obtener ganancias y sentirse cómodo al mismo tiempo.
	\item \emph{Pronósticos:} Para cada partido se proporcionan, entre otros: marcador final más probable, resultado más probable y el nivel de posesión de balón de cada equipo.
	\item \emph{Estadísticas:} A través de éstas se pueden analizar las fortalezas y debilidades de los equipos favoritos del usuario.
\end{enumerate}

\subsection{Encuesta}
\label{subsec:encuesta}


\textbf{Perfil de riesgo}


El perfil de riesgo sirve para poder personalizar la asesoría de apuestas y se calcula a través de las respuestas proporcionadas en la encuesta de perfil de riesgo.

De forma genérica hay tres perfiles de riesgo:
\begin{enumerate}
	\item Agresivo: Toma riesgos altos para poder obtener la mayor cantidad de ganancias posibles en el corto plazo. 
	\item Conservador: Apuesta a lo más seguro para proteger su dinero lo más posible, busca ganancias al largo plazo.
	\item Moderado: Término medio entre agresivo y conservador.
\end{enumerate}

\textbf{Sistema de reservas}

En \emph{Egobets.com} se entiende que cada persona es diferente, que cada apuesta es diferente y debe ser analizada de forma individual, por eso se ha desarrollado el sistema de reservas que determina cuánto dinero apostar en la recomendación de la semana.

La reserva es la cantidad de dinero que no se apostará, sirve para poder seguir apostando en semanas posteriores en el caso en que se lleguen a tener pérdidas.


Se toman en cuenta tres factores: la volatilidad de la apuesta, la ganancia esperada de ésta y el nivel de riesgo deseado del cliente. Se combina esta información en un modelo probabilístico que proporciona la cantidad a apostar. Mediante este sistema se busca de proteger al cliente de pérdidas potenciales.

Beneficios:
\begin{enumerate}

	\item Protege su dinero de pérdidas potenciales.
	\item Permite recuperarse con mayor velocidad de semanas con pérdidas.
	\item Permite dar una estructura de fondo de inversión a las apuestas al obtener un sistema de interés compuesto.

\end{enumerate}

Costos:
\begin{enumerate}
	\item Se restringe la cantidad de ganancias a corto plazo.
\end{enumerate}

Es un sistema a largo plazo, no se recomienda a personas que desean incurrir en riesgos elevados en beneficio de la posibilidad de obtener mayores ganancias.


\subsection{Tablero de apuestas}

El tablero es la página principal para el usuario de \emph{Egobets.com}. En el tablero se presenta la recomendación de la semana:
\begin{enumerate}

\item Barra de ingreso.
\item Los partidos en que se apostará: el partido, el resultado a apostar, la cantidad de dinero a apostar, el momio ofrecido y la casa de apuestas que ofrece tal momio.
\item La gráfica de valor esperado.
\end{enumerate}

En \emph{Egobets.com} se le da seguimiento a cada uno de nuestros clientes. Con esta información se puede monitorear la evolución del ingreso y así dar las recomendaciones de acuerdo al nivel de ganancias o pérdidas. Es importante que esta información sea verdadera para poder brindar el mejor servicio posible.
\begin{enumerate}

	\item \textbf{¿Se pueden hacer recomendaciones de resultados que no sean los más probables?}
\end{enumerate}


Sí, depende del perfil de riesgo y de lo que pague la casa de apuestas en tal partido. En algunas ocasiones es recomendable apostar en contra del favorito si el pago es suficientemente grande. Si se tiene activado el sistema en contra de favoritos (en el menú perfil) o si el perfil es muy agresivo se presentarán muchas recomendaciones de este tipo. 

La gráfica de valor esperado es una herramienta visual que permite ver cuáles son los posibles resultados de la recomendación de la semana. En \emph{Egobets.com} se conoce que todas las apuestas tienen un riesgo y mediante esta gráfica se puede cuantificar: Cada barra representa la probabilidad de que se gane o pierda la cantidad indicada debajo de ella, mientras más grande sea la barra mayores probabilidades hay de que tal resultado ocurra.



\subsection{Pronósticos}
Pronósticos: Se pronostica el marcador final, la posesión del balón y el ganador del encuentro. Al poner el apuntador sobre el resultado más probable aparece el grado de confiabilidad del pronóstico.


En \emph{Egobets.com} se presentan los equipos mediante un power ranking. Con base en las estadísticas de los partidos y dados sus resultados, se pronostica cuál será la tabla de posiciones al finalizar dicha liga.


\subsection{Estadísticas}


En orden de aparición: Una estrella indicando si el equipo está o no marcado como uno de los favoritos, la posición en el power ranking, el nombre del equipo, el índice de ataque general del equipo, el índice de defensa general del equipo y por último el cambio dentro de la tabla de power ranking. 

\begin{enumerate}

\item Índice de ataque general: Con calificación de una a cinco estrellas o de uno a diez (abajo). Representa la capacidad general del equipo para atacar.
\item Índices de medio centro, delanteros y definición: Con una calificación de cero a cien. Representan la capacidad de controlar el medio centro, de atacar a portería y de precisión de los tiros, respectivamente.
\item Índice de defensa general: Con calificación de una a cinco estrellas o de uno a diez (abajo). Representa la capacidad general del equipo para defender.
\item Índices de medio centro, defensas y portero: Con una calificación del cero a cien. Representan la capacidad de defender en el centro, de los defensas y del portero, respectivamente.
\item Al hacer click en las variables mencionadas en 2 o 4 se tienen acceso a la evolución de tales variables desde el minuto 0 hasta el 90 de un partido.
\end{enumerate}

\begin{enumerate}

	\item \emph{Resultados de la jornada anterior:} Se presenta el partido, el marcador real, el marcador pronosticado y el resultado pronosticado. Esto con el fin de que los usuarios puedan comparar lo pronosticado y lo que en verdad ocurrió.
	\item \emph{Pronósticos de la jornada actual:} Se presenta el partido, el resultado pronosticado (o favorito), el marcador pronosticado y la fecha del encuentro. Además al poner el apuntador encima de un partido se presenta el grado de confiabilidad del pronóstico.
\end{enumerate}

Además para cada tabla se pueden presentar los resultados de una liga en particular al seleccionarla en el recuadro de arriba.

\section{Alimentando el sistema}
\label{sec:inserting-data}	

\begin{figure}[!htb]\centering
   \begin {minipage}{1\textwidth}
     \includegraphics[width=\linewidth]{sistemas}
     \caption{Diagrama de sistemas y usuarios}\label{Fig:Sistemas}
   \end{minipage}
\end{figure}

\begin{figure}[!htb]\centering
   \begin {minipage}{1\textwidth}
     \includegraphics[width=\linewidth]{erd-pronosticos}
     \caption{Diagrama de entidad-relación de la BD del recopilador}\label{Fig:erd-pronosticos}
   \end{minipage}
\end{figure}

El proceso que se lleva a cabo en el \emph{Back Office} para alimentar el \emph{Portal público} (Ver la figura~\ref{Fig:flujo}), se puede describir de la siguiente manera:
\begin{enumerate}
	\item A través del \emph{Sistema de recopilación} los administradores descargan de la página de Internet de ESPN los resultados de todos los partidos de la temporada junto con la información de los próximos partidos por jugar  de cada una de las ligas Europeas.
	\item Los datos recopilados permiten a los administradores generar un conjunto de archivos de texto con toda la información de los resultados de los últimos partidos y las fechas de los próximos partidos.
	\item Los administradores usan estos archivos para alimentar el \emph{Sistema de estimación} y calcular los pronósticos de los próximos partidos y las probabilidades de los resultados.
	\item Se obtienen los archivos que contienen la información de los próximos partidos así como la información de los equipos por liga y su desempeño en la temporada en curso.
	\item En el \emph{Portal administrativo} se ingestan los archivos obtenidos con la información de los próximos partidos, resultados de partidos anteriores y las estadísticas de los equipos en la temporada en curso.
	\item Finalmente, con la nueva información ingresada, los usuarios podrán disfrutar en el \emph{Portal público} sus recomendaciones peronalizadas de apuestas.
	
\begin{figure}[!htb]\centering
   \begin {minipage}{1\textwidth}
     \includegraphics[width=\linewidth]{flujo}
     \caption{Proceso de alimentación del sistema}\label{Fig:flujo}
   \end{minipage}
\end{figure}
\end{enumerate}




% \section{Descripción General}
% \label{sec:description}
%
% El ecosistema de Egobets consiste principalmente de cuatro piezas de software. Ver la figura~\ref{Fig:Sistemas}.
%
%
%
%
% \begin{figure}[!htb]\centering
%    \begin {minipage}{1\textwidth}
%      \includegraphics[width=\linewidth]{sistemas}
%      \caption{Diagrama de sistemas y usuarios}\label{Fig:Sistemas}
%    \end{minipage}
% \end{figure}
%
% El Sistema de recopilación de información y estadísiticas de los partidos (\emph{Sistema de recopilación}), el \emph{Portal administrativo} y el \emph{Portal público} corren bajo una arquitectura cliente servidor en la nube de Amazon Web Services; mientras que el Sistema de estimación de probabilidades (\emph{Sistema de estimación}) corre en un ordenador personal.
%
%

%
% \subsection{Piezas de software y su interacción}
%
% \begin{figure}[!htb]\centering
%    \begin {minipage}{1\textwidth}
%      \includegraphics[width=\linewidth]{flujo}
%      \caption{Diagrama de flujo de información}\label{Fig:flujo}
%    \end{minipage}
% \end{figure}
%
% El proceso que se lleva a cabo en el \emph{Back Office} para alimentar el \emph{Portal público} (Ver la figura~\ref{Fig:flujo}), se puede describir de la siguiente manera:
% \begin{enumerate}
% 	\item A través del \emph{Sistema de recopilación} los administradores descargan de la página de Internet de ESPN los resultados de todos los partidos de la temporada junto con la información de los próximos partidos por jugar  de cada una de las ligas Europeas.
% 	\item Los datos recopilados permiten a los administradores generar un conjunto de archivos de texto con toda la información de los resultados de los últimos partidos y las fechas de los próximos partidos.
% 	\item Los administradores usan estos archivos para alimentar el \emph{Sistema de estimación} y calcular los pronósticos de los próximos partidos y las probabilidades de los resultados.
% 	\item Se obtienen los archivos que contienen la información de los próximos partidos así como la información de los equipos por liga y su desempeño en la temporada en curso.
% 	\item En el \emph{Portal administrativo} se ingestan los archivos obtenidos con la información de los próximos partidos, resultados de partidos anteriores y las estadísticas de los equipos en la temporada en curso.
% 	\item Finalmente, con la nueva información ingresada, los usuarios podrán disfrutar en el \emph{Portal público} sus recomendaciones peronalizadas de apuestas.
% \end{enumerate}
%
%
% \subsection{Sistema de recopilación de información y estadísticas de los partidos}
% \graphicspath{{/Users/brunomedina/Dropbox/Tesis-Egobets/egobets-notas/resources/recopilador/}}
% Se desarrolló un sistema web que recupera al principio de cada temporada todos los partidos que se hayan jugado en las temporadas pasadas, así como el calendario de próximos partidos por jugar. Para lograr la extracción de esta información el sistema cuenta con un ``Web Scraper'' que consigue la información de la página de ESPN y la transforma en objetos que después son persistidos en la base de datos. Finalmente, esta información se organiza en un documento CSV para ser descargado por los administradores.
%
% \subsubsection{Tecnologías destacadas}
%
% \begin{itemize}
% 	\item \textbf{Servidor LAMP.} Esta es una de las configuraciones más populares para servidores web.
% 		Una de las principales ventajas de utilizar una arquitectura LAMP es que la mayoría del software utilizado corre bajo la ``Licencia Pública General de GNU'', esta licencia de uso permite a los programadores que utilizan este software, ser capaces de ver el código fuente y en muchos casos le permiten modificarlo y compartirlo \cite{lozano2008software}.
% 	El famoso acrónimo representa lo siguiente :
% 	\begin{itemize}
% 		\item Apache. El servidor HTTP, encargado de recibir y procesar las llamadas HTTP, nació hace diecisiete años como un proyecto por desarrollar un software robusto, de grado comercial, modular y gratuito para servir peticiones (Web) HTTP. Algunas de las ventajas con las que cuenta son \cite{apacheWeb}:
% 		\begin{itemize}
% 			\item Gran variedad de módulos. Existen muchos módulos que la comunidad desarrolla para atacar las problemáticas diarias, gracias a que su comunidad es muy activa estos desarrollos nunca acaban.
% 			\item Fácil de administrar. El soporte de la comunidad y su extensivo uso permite que sea muy sencillo encontrar documentación de como realizar las funciones básicas de administración de servidores como la creación de nuevos dominios, implantación de certificados SSL, etc.
% 			\item No importa el sistema operativo, Apache corre en UNIX, Windows, Mac y en la gran mayoría de sistemas operativos.
% 			\item Corre bajo la licencia pública general de GNU.
% 		\end{itemize}
%
% 		\item MySQL. Base de datos relacional que permite al sistema persistir toda la información recuperada del internet. Las principales características de MySQL son \cite{mysqlWeb}:
% 		\begin{itemize}
% 			\item Sistema de administración. El servidor MySQL provee las herramientas para agregar, ingresear, procesar y desplegar la información guardada en la base de datos.
% 			\item Las bases de datos MySQL son relacionales. La información se guarda en tablas separadas en vez de poner todo junto en un solo lugar. Las estructuras de base de datos están organizadas en archivos físicos optimizados para la velcidad. El modelo lógico con objetos como bases de datos, tablas, vistas, tuplas y columnas ofrece un ambiente de programación flexible. Y MySQL se asegura de que se respeten los tipos de relaciones entre tablas como puede ser uno a uno, uno a varios, únicas, u opcionales. Una base de datos bien diseñada nunca permitirá información incosistente, duplicada, huerfana, fuera de fecha o perdida.
% 			\item Corre bajo la licencia Pública General de GNU. Por lo que se puede modificar y ser utilizado por cualquiera.
% 		\end{itemize}
% 				\item PHP: 
%
%
% 	\item Code Igniter. Es un framework\footnote{Es una estructura de software compuesta de componentes personalizables e intercambiables para el desarrollo de una aplicación. En otras palabras, un framework se puede considerar como
% una aplicación genérica incompleta y configurable a la que se le puede añadir las últimas piezas para construir una aplicación concreta.} de PHP que ahorra tiempo en le programación, robustece tu sistema y permite al programador alcanzar un grado mayor de sofisticación en su código. Uno de los puntos interesantes de este framework es que utiliza el patrón de diseño conocido como Modelo Vista Controlador (MVC), este patrón fue descrito por el noruego Trygve Reenskaug en 1979.
% 	Sobre el libro de Upton \cite{upton2007codeigniter} se tiene una aproximación a este patrón de diseño en CodeIgniter:
% 	\begin{itemize}
% 		\item Modelos, son objetos que representan los datos. Estos objetos reflejan las tabla de la base de datos y pueden modificarla conforme sea requerido. Los modelos también realizan operacioens a los datos según sea necesario.
% 		\item Vistas, reflejan el estado del modelo. Son las responsables de desplegar la información al usuario final. En este caso específico, todas las vistas son representaciones HTML del contenido.
% 		\item Controladores, ofrecen opciones para cambiar el estado del modelo. Son los encargados de consultar los modelos. Proveen a las vistas los datos dinámicos a mostrar.
% 	\end{itemize}
% 	De su página Web \cite{codeigniterWeb} se pueden destacar las siguientes propiedades:
% 	\begin{itemize}
% 		\item Tamaño pequeño. CodeIgniter 2.2 tiene una descarga 2.2MB, incluyendo la guía del usuario.
% 		\item Documentación clara. La guía que se incluye cuenta con guía y tutoriales para empezar a trabajar de manera muy práctica.
% 		\item Compatibildad con casi cualquier servicio de alojamiento. Sólo necesita PHP 5.1.6 y tiene soporte con las bases de datos más comunes incluído MySQL.
%
% 		\item Casi no necesita configuración. Todas las variables y opciones de configuración vienen predefinidas a los estandares convenidos en internet.
%
% 	\end{itemize}
%
%
% 	\item PHP Simple HTML DOM Parser
% 	Es un script de PHP que permite interpretar el HTML DOM de una página web y permite manipularlo de manera muy sencilla. Requiere PHP 5 o mayor, soporta archivos HTML mal formados y permite encontrar las etiquetas HTML con selectores como lo haría jQuery. Con una simple línea de código basta para extraer los contenidos de una página HTML \cite{htmlparserWeb}.
% 	\cite{chen2009php} \cite{chowdhury2014intelwiki}
% 	\item Bootstrap. Es un framework elegante, intuitivo y poderoso que agiliza y facilita el desarrollo Web de front-end Su principal objetivo es facilitar el desarrollo de sitios móviles y responsive.
% 	Documentación amplia y detallada, docenas de elementos HTML, componentes CSS e increíbles plugins de jQuery son algunas de sus principales características\cite{bootstrapWeb}.
% 	\cite{otto2010bootstrap}
% 	\cite{cochran2012twitter}
%
% \end{itemize}
%
% \subsubsection{Web Scraper}
% El proceso de ``Web Scraping'' consiste en extraer y crear representaciones estructuradas con la información de un sitio Web en particular. HTML, el lenguaje de marcado usado para darle estructura a la información de las páginas web está sujeta a muchos cambios, especialmente cuando se actualiza su estilo. Ya que las técnicas de extracción se basan en el lenguaje de marcado, un solo cambio puede llevar a extraer datos incorrectos o incompletos \cite{cording2011algorithms}.
%
% Las técnicas de ``Web Scraping'' permiten, por ejemplo, que una compañía monitoree los precios de los productos de sus competidores. De igual manera le permiten a Egobets, conseguir la última información detallada con la información de las fechas de los partidos, marcadores, tiros a gol, posesión del balón y nombres de los equipos.
%
% Si el dueño de la información no provee de una API abierta, el remedio (como en el caso de Egobets) es el de escribir un programa que apunte a la información desplegada en la página Web. Las propiedades buscadas en un scraper son:
% \begin{itemize}
% 	\item Ser tan tolerante como pueda ser posible a cambios en el lenguaje de marcado.
% 	\item Ser lo suficientemente rápido como para ofrecer respuestas en milisegundos y evitar ``time-outs'' durante el consumo del servicio.
% 	\item No tener restricciones en los patrones que conforman la estructura del HTML del sitio. En específico esto implica que el servicio debe ser lo suficientemente bueno como para interpretar de la mejor manera la información aún incluso si el HTML se encuentra mal formado
% \end{itemize}
%
% Cording, P. \cite{cording2011algorithms} presenta en su tesis de maestría un estudio de como realizar un scraper que base su funcionamiento en la coincidencia aproximada de los patrones de un árbol, esta es la base de muchas de las librerías que existen (incluyendo la que se usa en este trabajo). Para realizar esta tarea se utiliza una librería de código abierto que permite manipular DOM\footnote{W3schools \cite{domWeb} lo define como: ``Documento W3C Object Model (DOM) es una interfaz de la plataforma y de lenguaje neutro que permite a los programas y scripts acceder y actualizar dinámicamente el contenido, la estructura y el estilo de un documento''. La especificación formal se puede leer en la recomendación \cite{wood1998document}.} de HTML y conseguir la información necesaria, esta libreria se llama: PHP Simple HTML DOM Parser.
%
% \textbf{Un ejemplo de uso}
%
% Supóngase el siguiente HTML:
%
% \lstset{language=HTML}
% \begin{lstlisting}[frame=single]
% <div class="team-names">
% 	<div class="team-name">
% 		<span>
% 			<img src="http://espncdn.com/teamlogos/soccer/500/102.png" alt="Villarreal" width="22">
% 			Villarreal
% 		</span>
% 	</div>
% 	<div class="team-name">
% 		<span>
% 			<img src="http://espncdn.com/teamlogos/soccer/500/93.png" alt="Athletic Bilbao" width="22">
% 			Athletic Bilbao
% 		</span>
% 	</div>
% </div>
% \end{lstlisting}
%
% Ahora analícese el siguiente código:
%
% \lstset{language=PHP}
% \begin{lstlisting}[frame=single]
% 	$aux = $table->find('.team-name',0);
% 	$nomEquipoLocal = $aux;
% 	$partido['prints']['local_nom']= $nomEquipoLocal;
% 	$idEspnEquipoLocal = trim(strstr(substr(strstr($aux->innertext, 'soccer/500/'), strlen('soccer/500/')),'.png',true));
% \end{lstlisting}
%
%
% Con este código se pueden obtener los nombres del equipo local y el identificador que le asigna ESPN a sus equipos. Esto es gracias a que \emph{``find('.team-name',0)''} recorre todo el árbol DOM y encuentra todas las ocurrencias de la clase con nombre \emph{``team-name''} el cero que se le proporciona a la función \emph{find} regresa el primer elemento encontrado. De ahí, la propiedad \emph{``->innertext''} muestra la cadena de caracteres plana que contiene ese elemento, por lo que basta con hacer un manejo de cadenas de caracteres para obtener el identificador del equipo que se nombra el archivo PNG del logo del equipo. Para más ejemplos y documentación de como utilizar la librería dirigirse a \cite{sourceparserWeb}.
% % Es un script que
% % \cite{hogue2005thresher}
% % Definir lo que es un scraper
% %
% %
% % Definir las páginas que se buscan y como se recorren
% %
% % Definir los objetos finales de la base de datos que se consumen.
% %
% %
%
%
% \subsubsection{Descripción de su funcionamiento}
%
% \begin{enumerate}
% 	\item \textbf{Equipos de la temporada.}
% 	Para poder comenzar la recuperación de información, es importante contar con los equipos que estén jugando esta temporada. Dependiendo de los resultados de la temporada anterior, los equipos que hayan quedado hasta abajo en la tabla de posición descienden a ligas menores y a su vez suben los mejores de estas ligas. Ver figura~\ref{Fig:los-equipos}
% 	\begin{figure}[!htb]\centering
% 	   \begin {minipage}{1\textwidth}
% 	     \frame{\includegraphics[width=\linewidth]{los-equipos}}
% 	     \caption[Ejemplo de equipos de la liga española]{Ejemplo de equipos de la liga española\footnotemark }\label{Fig:los-equipos}
% 	   \end{minipage}
% 	\end{figure}
%
%
% 	\item \textbf{Calendario de próximos partidos.}
% 	\begin{figure}[!htb]\centering
% 	   \begin {minipage}{1\textwidth}
% 	     \frame{\includegraphics[width=\linewidth]{proximos-partidos}}
% 	     \caption{Recuperación de próximos partidos}\label{Fig:proximos-partidos}
% 	   \end{minipage}
% 	\end{figure}
%
% 	\item \textbf{Estadísticas e información de partidos jugados.}
% 	\begin{figure}[!htb]\centering
% 	   \begin {minipage}{1\textwidth}
% 	     \frame{\includegraphics[width=\linewidth]{pasados-partidos}}
% 	     \caption{Recuperación de resultados de partidos ya jugados}\label{Fig:pasados-partidos}
% 	   \end{minipage}
% 	\end{figure}
% 	\item \textbf{Generación de archivos con resultados.}
% 	Es importante mencionar que se usan aproximadamente los últimos quinientos partidos para la generación de los archivos, esto implica que se deben tener en base de datos los equipos que participaron en las pasadas dos temporadas de juegos. Por este motivo de pueden encontrar equipos que se encuentran en el sistema con la bandera de inactivos.
%
% \end{enumerate}
%
%
%
%
%
%
%
%
%
%
%
%
%
% \subsection{Sistema de estimación de probabilidades}
% Adicionalmente en esta sección se describirá el funcionamiento del Sistema de Estimación. Sin embargo, al ser un conjunto de programas en \emph{Fortran} que son ajenos al autor, no se profundizará en los detalles del desarrollo del mismo. Sin embargo, se dará la pauta para entender como se podrían generar probabilidades y pronósticos de los partidos.
%
% \subsubsection{Predicciones}
%
%
%
% \cite{rue2000prediction}
%
% \cite{baio2010bayesian}
%
% \cite{dixon2004value}
%
% \cite{koopman2013dynamic}
%
%  \subsubsection{Tecnologías destacadas}
%
% \begin{itemize}
% 	\item \textbf{Fortran}
% 	\cite{robison1996c++}
% 	\cite{veldhuizen1997will}
% \end{itemize}
%
% \subsubsection{Descripción de su funcionamiento}
%
% Montecarlo
% Poisson
% Nonormal
% Markov
%
%
% \subsection{Portal administrativo}
%
% \subsubsection{Sitio web para los administradores}
%
%  \cite{alfredo2005ingenieria}
%
%  \subsubsection{Tecnologías destacadas}
%
%  \begin{itemize}
%  	\item LNNP
%  	\item Code Igniter
%  	\item Raphael
%  \end{itemize}
%
%  \subsubsection{Diagrama de base de datos}
%
%  \subsubsection{Descripciónde su funcionamiento}
%
%  \begin{itemize}
%  	\item CRUD
%  	\item Ingesta de archivos
%
%  \end{itemize}
%
%
% \section{Portal público Egobets.com}
%
% \subsection{Características principales}
% \subsubsection{Sitio web para los jugadores}
% \subsubsection{Tecnologías destacadas}
%
%  \begin{itemize}
%  	\item LNNP
%  	\item Code Igniter
%  	\item Parallax
%  \end{itemize}
%
% \subsubsection{Módulos}
%
%  \begin{itemize}
%  	\item Tablero
%  	\item Mis Equipos
%  	\item Ligas
%  	\item Partidos
%  	\item Perfil
%  	\item Pagos
%  	\item Perfil de riesgo
%  	\item Sistemas de reserva
%  \end{itemize}
%
%
%
 %Conclusiones
% Reestructurando a sólo 4 capítulos
%% \chapter{Portal público}
% \section{Perfil de usuario}
% \section{Encuesta de aversión al riesgo}
% \section{Ahorro precaucional}
% \section{Sugerencia de apuestas}
% \section{Pagos en línea}
% \section{Power ranking}
%
\chapter{Conclusiones}

En esta tesis se describió el sistema de recomendación de apuestas personalizadas para las ligas europeas de fútbol: \textbf{``Egobets''}. En conclusión, el sistema se puede pensar como un compendio de técnicas, teoremas y resultados ya conocidos en las ramas de optimización, estadística, finanzas, probabilidad y economía; aplicados en la creación de un sistema computacional robusto.

De este trabajo se puede concluir lo siguiente:



\begin{itemize}


	\item Existe más de un modelo que permita predecir los resultados de los partidos de las ligas europeas de fútbol. Más aún, el modelo de Maher (1982)\cite{maher1982modelling} proporciona predicciones lo suficientemente confiables como para considerar su uso en una estrategia de apuestas de hoy en día.

	\item Conociendo las probabilidades de los partidos, aún siendo estimadas, es posible encontrar una estrategia de apuestas que genere rendimientos positivos. Por ejemplo, Koopman en su artículo \cite{koopman2013dynamic} expone un ejemplo donde, mediante una estrategia de apuestas simple, por cada $75$ unidades apostadas recibe $25$ unidades de ganancia.

	\item La aversión al riesgo de un apostador se puede modelar con funciones de utilidad, esto permite que las recomendaciones de apuestas sean personalizadas. 
	
	\item Considerando que las temporadas del fútbol tienen varias jornadas, se integra un sistema de reservas cuya función principal consiste en maximizar la tasa de crecimiento de las ganancias del jugador. Vancura \cite{vancura2000finding} muestra varias estrategias de reservas que buscan conseguir estos resultados.

\item En el sistema de Egobets.com se provee de una asesoría de apuestas integral para cada semana de la temporada donde se utilizan las predicciones de los resultados de los partidos, el perfil de riesgo del usuario y el sistema de reservas.

\item Los momios que publican las casas de apuestas se enfocan a predecir el mercado, por lo tanto se pueden mejorar las probabilidades propuestas.

\item  El porcentaje de aciertos en los pronósticos es del $70\%$ para la liga española, $61\%$ en la italiana, $52\%$ en la inglesa, $52\%$ en la alemana y $48\%$ en la francesa.

\item En los resultados observados, Egobets reporta un rendimiento desde el $29\%$ hasta el $83\%$ dependiendo del nivel de riesgo por apuesta del usuario.

\item El cómputo en la nube es esencial para los sistemas como Egobets, sus ventajas como escalabilidad permiten mantener costeables y funcionales los servicios.

\item El enfoque de un desarrollo al diseño y la usabilidad permiten que el usuario se enfoque en realizar únicamente lo que debe realizar.

\item MongoDB proporciona el beneficio de guardar información sin una estructura definida. Esto permite mucha mayor flexibilidad en los documentos que se persisten y la información que contienen\cite{puniaimplementing}.

\end{itemize}

% - Los juegos de azar de los casinos contemplan un margen para la casa que les garantiza ganancias a futuro.
% - Es claro que, más allá del azar, la habilidad y destreza de los equipos dominan los resultados de una temporada de fútbol.
% - Los momios que publican las casas de apuestas se enfocan a predecir el mercado, por lo tanto se pueden mejorar las probabilidades propuestas.
% - Existe más de un modelo que permita predecir los resultados de las ligas de fútbol, desde Maher \cite{maher1982modelling} estas predicciones han sido lo suficientemente confiables como para considerar su uso en apuestas.
% - Conociendo las probabilidades de los partidos, aún siendo estimadas, es posible encontrar una estrategia de apuestas que genere rendimientos positivos como demostró Koopman en su artículo \cite{koopman2013dynamic} mediante una simple estrategia de apuestas considerando valores esperados positivos.
%
%
% - Además de tener las probabilidades y la estrategia de apuestas, Vancura \cite{vancura2000finding} señala que el criterio de Kelly \cite{kelly1956new} maximiza asintóticamente la tasa de crecimiento de las ganancias, por lo que se integra esquema de reservas a la solución propuesta.
%
% - Con el fin de buscar una asesoría personalizada se propone un esquema de recomendaciones basadas en el perfil de riesgo del apostador.
%
% - En el sistema de Egobets.com se provee de una asesoría de apuestas integral para cada semana de la temporada donde se utilizan las predicciones de los resultados de los partidos, el perfil de riesgo del usuario y el sistema de reservas.
%
%
% - El porcentaje de aciertos en los pronósticos es del $70\%$ para la liga española, $61\%$ en la italiana, $52\%$ en la inglesa, $52\%$ en la alemana y $48\%$ en la francesa.
%
%
% - En los resultados observados, Egobets reporta un rendimiento desde el $29\%$ hasta el $83\%$ dependiendo del nivel de riesgo por apuesta del usuario.
%
% - Las ganancias dependen del acierto de las predicciones, sin embargo tener estas ganancias con la probabilidad de aciertos que se tiene verifica las sospechas de que se puede tener una estrategia redituable de apuestas a pesar del pronóstico de los partidos.
%
% - El cómputo en la nube es esencial para los sistemas como Egobets, sus ventajas como escalabilidad permiten mantener costeables y funcionales los servicios.
%
% - El enfoque de un desarrollo al diseño y la usabilidad permiten que el usuario se enfoque en realizar únicamente lo que debe realizar
%
% - MongoDB proporciona el beneficio de guardar información sin una estructura definida. Esto permite mucha mayor flexibilidad en los documentos que se persisten y la información que contienen\cite{puniaimplementing}. Mientras que las bases de datos relacionales resultan mucho más eficientes en consultas complejas. En general, un esquema combinado de ambos tipos de bases de datos pueden resultar en un sistema de información eficaz y completo \cite{faraj2014comparative}.
%
%
% Beneficios:
%
% - Egobets ofreca recomendaciones personalizadas de apuestas, cada persona es diferente y es tratada de forma única. Gracias a la encuesta se determina el perfil de riesgo del usuario y se le asesora de tal manera que obtenga ganancias y se sienta cómodo  al mismo tiempo.
%
%
% Acciones a futuro
% - Mejorar el sistema de predicción de resultados.
% - En un futuro este sistema podría modificarse para abarcar más mercado.
% 	- Otras ligas, por ejemplo la mexicana
% 	- Otros deportes, por ejemplo fútbol americano.
% - Incluso se varios de los autores hablan de que modelos parecidos a estos podrían servir para predecir elecciones.
%



\textbf{Acciones a futuro}
Un sistema tan complejo como éste genera muchas áreas de oportunidad, una de ellas sería implementar las nuevas investigaciones para mejorar el sistema de predicción de resultados, otro detalle sería encontrar las funciones de utilidad más prolíficas del sistema y centrar a los usuarios sobre esas. Adicionionalmente, se podría trabajar en hacer más autónomo el sistema para generar las recomendaciones. Finalmente, sobre este sistema se podría incluso podrían modificar la lógica y adaptar el modelo para abarcar más mercados, como:
\begin{itemize}

	\item Muchas más ligas de fútbol. La traslación del modelo a la liga de fútbol mexicana sería la primera opción y después una apertura a todas las ligas. Considerando ajustes necesarios como los aumentos en la varianza de los resultados, verificación de la significancia de las variables utilizadas en este modelo junto con las correcciones necesarias al modelo de predicción. De igual manera se tendría que buscar la integración de los momios de estas ligas y verificar las fuentes.

	\item Otros deportes. En una primera instancia se piensa utilizar este mismo modelo para fútbol americano, considerando que este deporte contiene mucho mayor información estadística de cada partido por lo que podría ayudar a tener una mejor predicción.

\end{itemize}

	
Para concluir, \textbf{``Egobets''} es un vivo ejemplo de como las Matemáticas y la Computación, conviviendo de manera simbiótica, aportan un sistema tangible que facilita a cualquier persona el acceso a información detallada, procesada y enfocada a mejorar su entendimiento acerca de un tema que, a la distancia, podría parecer complejo. Más aun, \textbf{``Egobets''} pone a la disposición de sus clientes un conjunto de poderosas herramientas computacionales basadas en estudiados modelos matemáticos que les proporcionan un servicio disponible a cualquier hora del día a través de un práctico portal Web. Finalmente, retomando la famosa frase de Bernardo de Chartres \cite{john1962metalogicon}: ``Somos enanos parados sobre hombros de gigantes''. Se invita a todos los lectores interesados, en usar este trabajo como excusa para desarrollar sistemas y plataformas innovadoras con usos prácticos de sus modelos matemáticos favoritos.

























%
Partiendo de los siguientes dos supuestos: a) un jugador promedio busca maximizar las ganancias de sus apuestas en función de su adversidad al riesgo y, b) apostar siempre conviene más que no apostar. Se plantea una manera de encontrar la apuesta óptima para un partido. Con base en este planteamiento se sigue el análisis a una jornada: ¿A qué partidos de la jornada el usuario le debería apostar? Finalmente, considerando que el usuario busca apostar en todas las jornadas de la temporada, se ataca el problema de la evolución del dinero a apostar durante toda la temporada.


\section{Ahorro precaucional}



\section{Evolución de la cantidad a Apostar}


%----------------------------------------------------------------------------------------
%	APÉNDICES
%----------------------------------------------------------------------------------------

\addtocontents{toc}{\vspace{2em}} % Agrega espacios en la toc

\appendix % Los siguientes capítulos son apéndices
%  Incluye los apéndices en el folder de apéndices
\chapter{La ruina del jugador}\label{chap:manual}

\section{Demostración de la ruina del jugador}
http://www.columbia.edu/~ks20/stochastic-I/stochastic-I-GRP.pdf


http://www.ifp.illinois.edu/~sgorant2/gambler.html


Two gamblers Alice and Bob play the following game: Alice repeatedly tosses a fair coin. 
After each toss that comes up H, Bob pays Alice one dollar. After each toss that comes up T, 
Alice pays Bob one dollar. The game continues until either one or the other gambler runs out o
f money. If Alice starts with \$A and Bob starts with \$B,
. What is the probability that, when the game ends, Alice has all the cash?
. What is the expected duration of the game?
 
== Solution
Lets solve the problem using the Doob's Optional stopping Theorem for martingales.
Let $X_1,X_2,\cdots$ be the increments of Alice's wealth. Hence $X_i= 1$, depending on H or T. 
Hence, change in Alice's cash is
\( 
S_n = \sum_{i=1}^n X_i 
\)
Define  
\(
\tau = \min\{t: S_t = +B \mbox{ or } S_t = -A \}
\)
Clearly $\tau$ is a stopping time relative to the natural filtration
$\mathcal{F}_n = \sigma(X_1,X_2,X_3,\cdots,X_n)$. 
$\tau' = \tau \wedge n$ is also a stopping time.

=== (1) Probability of Alice winning
The sequence $S_n$ is a martingale relative to the natural filtration $\mathcal{F}_n$. 
Hence, using Optional Stopping Theorem for $n < \infty$,
\( 
 0 = {E}[S_0] = {E}[S_{\tau \wedge n}] = -A P(\tau \leq n \mbox{ and } S_{\tau} = -A) + B P(\tau \leq n \mbox{ and } S_{\tau} = +B) + E[S_n \chi_{\tau > n}]
 \)                                               
 As $n \to \infty$, the probability that  $\tau > n$ converges to zero. 
 The last term in the above equation is the expectation of a bounded martingale $S_n$
 bounded between $A$ and $B$ and converges to 0. Thus,
 \(
  0 = - A P(S_{\tau} = -A) + B P(S_{\tau} = +B) 
 \)
 Hence, probability that Alice has all the cash $S_{\tau} = +B$ is
 \(
 P(S_{\tau} = +B) = \frac{A}{A+B}.
 \)
 
=== (2) Expected duration of game.
The sequence $(S_n^2-n)$ is a martingale relative to the natural filtration $\mathcal{F}_n$.
Hence, using Optional Stopping Theorem for $n < \infty$,
\( 
 0 = {E}[S^2_0] = {E}[S_{\tau \wedge n}^2 - (\tau \wedge n)]
 \)                                                                                                          
 As $n \to \infty$, the probability that  $\tau > n$ converges to zero. 
Hence,
\(
E[\tau] = E[S_{\tau}^2] = A^2  P(S_{\tau} = -A) + B^2  P(S_{\tau} = +B) = AB 
\)                               
Hence, the expected duration of the game is AB.


 \section{Decidir a favor de quien apostar}
\label{apostar-a-quien}
 \begin{enumerate}[(a)]
  \item Sean $p_L$, $p_z$, $p_v$ las probabilidades de que gane local, empaten o gane visitante, respectivamente. Sean $\mu_L$, $\mu_z$ y $\mu_v$ los momios respectivos. El problema de decisión de apostar \$\,1 en esta situación es:\\
 
 % Set the overall layout of the tree
 \tikzstyle{level 1}=[level distance=3.5cm, sibling distance=2.5cm]
 \tikzstyle{level 2}=[level distance=3.5cm, sibling distance=2cm]
 \tikzstyle{level 3}=[level distance=3.5cm, sibling distance=2cm]


 \begin{figure}[ht]
 \begin{tikzpicture}[grow=right, sloped]
 \node[text width=4em, text centered] {$\square$}
 %%%%%%%%%%%%%%%%%%%%%%%%%%%%%%%%%%%%%%%%%%%%%%%%%
 %cuarto
 %%%%%%%%%%%%%%%%%%%%%%%%%%%%%%%%%%%%%%%%%%%%%%%%
 child {       
     node[text width=4em, text centered] {1}
              edge from parent         
  node[above] {$\delta_{NA}$}
     }
 %%%%%%%%%%%%%%%%%%%%%%%%%%%%%%%%%%%%%%%%%%%%%%%%%
 %tercero
 %%%%%%%%%%%%%%%%%%%%%%%%%%%%%%%%%%%%%%%%%%%%%%%%
 child {       
     node[text width=4em, text centered] {\textbigcircle}
     child {
                 node[circle, minimum width=1pt,fill, inner sep=0pt, label=right:
                     {$0$}] {}
                 edge from parent
                 node[above] {$1-p_v$}             
             }
             child {
                 node[circle, minimum width=1pt,fill, inner sep=0pt, label=right:
                     {$\mu_v$}] {}
                 edge from parent
                 node[above] {$p_v$}              
             }
              edge from parent         
  node[above] {$\delta_v$}
     }
 %%%%%%%%%%%%%%%%%%%%%%%%%%%%%%%%%%%%%%%%%%%%%%%
 %segundo
 %%%%%%%%%%%%%%%%%%%%%%%%%%%%%%%%%%%%%%%%%%%%%%%%
     child {       
     node[text width=4em, text centered] {\textbigcircle}
     child {
                 node[circle, minimum width=1pt,fill, inner sep=0pt, label=right:
                     {$0$}] {}
                 edge from parent
                 node[above] {$1-p_z$}             
             }
             child {
                 node[circle, minimum width=1pt,fill, inner sep=0pt, label=right:
                     {$\mu_z$}] {}
                 edge from parent
                 node[above] {$p_z$}              
             }
              edge from parent         
  node[above] {$\delta_E$}
     }
 %%%%%%%%%%%%%%%%%%%%%%%%%%%%%%%%%%%%%%%%%%%%%%%
 %primero
 %%%%%%%%%%%%%%%%%%%%%%%%%%%%%%%%%%%%%%%%%%%%%%%%
     child{
     node[text width=4em, text centered] {\textbigcircle}        
             child {
                 node[circle, minimum width=1pt,fill, inner sep=0pt, label=right:
                     {$0$}] {}
                 edge from parent
                 node[above] {$1-p_L$}             
             }
             child {
                 node[circle, minimum width=1pt,fill, inner sep=0pt, label=right:
                     {$\mu_L$}] {}
                 edge from parent
                 node[above] {$p_L$}              
             }    
  edge from parent         
  node[above] {$\delta_L$}
     };  
 \end{tikzpicture}
 \caption{Decidir por quién apostar}
 \end{figure}

   $E_p[U(\delta_i)]=p_i\mu_i;\quad i=L,Z,V$\\
  
   Sol: Se escoge $\rho_i \,\, \cdot \ni \cdot \,\, E_p[U(\delta_i)]=max\{p_L\mu_L,p_z\mu_z,p_v\mu_v,1\}$
  
   \item Se quiere decidir si apostar o no en la ocurrencia de un evento: Sea $p=p(E)$ y $f_p$ densidad de $p$. Sea $\mu$ el momio en el caso de ocurrencia. El problema de decisión asociado es el siguiente:\\
 
 \begin{figure}[!ht]
  \begin{tikzpicture}[grow=right, sloped]
 \node[text width=4em, text centered] {$\square$}
 %%%%%%%%%%%%%%%%%%%%%%%%%%%%%%%%%%%%%%%%%%%%%%%%%
 %segundo
 %%%%%%%%%%%%%%%%%%%%%%%%%%%%%%%%%%%%%%%%%%%%%%%%
 child {       
     node[text width=4em, text centered] {0}
              edge from parent         
  node[above] {$\delta_{NA}$}
     }
 %%%%%%%%%%%%%%%%%%%%%%%%%%%%%%%%%%%%%%%%%%%%%%%
 %primero
 %%%%%%%%%%%%%%%%%%%%%%%%%%%%%%%%%%%%%%%%%%%%%%%%
     child{
     node[text width=4em, text centered]{\textbigcircle}
    	    child {
    	        node[]{\textbigcircle}   	                 
                 %node[above] {$f_p$}
                 child{node[circle, minimum width=1pt,fill, inner sep=0pt, label=right:
                     {$-1$}] {}                    
                 edge from parent
                 node[above] {$1-p$}             
             }
             child {
                 node[circle, minimum width=1pt,fill, inner sep=0pt, label=right:
                     {$\mu-1$}] {}
                 edge from parent
                 node[above] {$p$}              
             }
             edge from parent
             node[above] {$f_p$}}    
  edge from parent         
  node[above] {$\delta_A$}
     };
 \end{tikzpicture}  
 \caption{Decidir si apostar o no apostar}
 \end{figure}
 
 \newpage
  
   $\rightarrow E_p[U(\delta_A)]=E_{f_p}[p(\mu-1)-(1-p)]\\
   =E_{f_p}[p(\mu)-1]\\
   =E_{f_p}(p)\mu-1$\\

   Apuestas si $E_{f_p}(P)\cdot \mu \ge 1$
  
   \item Mismo problema que el caso anterior, sólo que la utilidad depende de $p$ y $\mu$: $U: \Re\times[0,1]\rightarrow\Re$\\
   $(U(0,p)=0\quad\forall p)$.\\
 
 \begin{figure}[ht]
 \begin{tikzpicture}[grow=right, sloped]
 \node[text width=4em, text centered] {$\square$}
 %%%%%%%%%%%%%%%%%%%%%%%%%%%%%%%%%%%%%%%%%%%%%%%%%
 %segundo
 %%%%%%%%%%%%%%%%%%%%%%%%%%%%%%%%%%%%%%%%%%%%%%%%
 child {       
     node[text width=4em, text centered] {0}
              edge from parent         
  node[above] {$\delta_{NA}$}
     }
 %%%%%%%%%%%%%%%%%%%%%%%%%%%%%%%%%%%%%%%%%%%%%%%
 %primero
 %%%%%%%%%%%%%%%%%%%%%%%%%%%%%%%%%%%%%%%%%%%%%%%%
     child{
     node[text width=4em, text centered]{\textbigcircle}
    	    child {
    	        node[]{\textbigcircle}   	                 
                 %node[above] {$f_p$}
                 child{node[circle, minimum width=1pt,fill, inner sep=0pt, label=right:
                     {$U(-1,p)$}] {}                    
                 edge from parent
                 node[above] {$1-p$}             
             }
             child {
                 node[circle, minimum width=1pt,fill, inner sep=0pt, label=right:
                     {$U(\mu-1,p)$}] {}
                 edge from parent
                 node[above] {$p$}              
             }
             edge from parent
             node[above] {$f_p$}}    
  edge from parent         
  node[above] {$\delta_A$}
     };
 \end{tikzpicture}
 \caption{Decidir si apostar en función de una utilidad}
 \end{figure}

  
  
   Se apuesta si: \\
   $E_p(U(\delta_A))=E_p[p\,U(\mu-1,p)+(1-p)U(-1,p)]\ge0$
 \end{enumerate}

 Algunas funciones de utilidad posibles:
 \begin{itemize}
  \item $U_\mu(x,p)=x(\frac{1}{\mu}-p)^2$\\
 
  Notese que: $p\,U_\mu(\mu-1,p)+(1-p)U_\mu(-1,p)$\\
 
  $(\hat p=\frac{1}{\mu})=(\hat p-p)^2(p\mu-1)$\\
 
  Me duele más mientras más alejado esté de un trato beneficioso y me produce mayor placer mientras mayor sea el beneficio del trato.
 
  \item $U_{\mu,a}(x,p)= \left\{ \begin{array}{lcc}
              ax(\hat p-p)^2 &   si  & p \le \hat p \\
              & &\\
              x (\hat p-p)^2 &  si & p>\hat p\\             
              \end{array}
  \right.$
 
  Notese que: \\
  \[U_\mu=U_{\mu,1}\]\\
  \[p\,U_{\mu,a}(\mu-1,p)+(1-p)U_\mu(-1,p)= \left\{ \begin{array}{lcc}
              a(\hat p-p)^2 (p\mu-1)&   si  & p \le \hat p \\
              & &\\
              (\hat p-p)^2(p\mu-1) &  si & p>\hat p\\             
              \end{array}
  \right.\]

  Me duele ``a'' veces más un trato perjudicial  que un trato beneficioso si me encuentro a la mis ma distancia que $\hat p$.
 
  \item $U_{\mu,a,b}=U_{\frac{\mu}{1+\mu b},a}$\\
 
  y considerar el problema de decisión con $\mu'=\frac{\mu}{1+\mu b}$.\\
 
  Si $\mu'=\frac{\mu}{1+\mu b}\rightarrow \hat p'=\hat p+b$.\\
 
  Los tratos empiezan a ser beneficiosos hasta que el menos sea $b\%$ más probable que ocurra el evento de lo que sería justo.\\
 
  {\bf Nota:} En un problema de decisión sin aversión a la distribución de probabilidades (o con probabilidades fijas) si se desea apostar en apuestas con un mínimo de ganancias esperadas igual a $b\%$ se debe comparar $\mu_p$ con $1+b$ (i.e. apostar $\leftrightarrow \mu_p \ge 1+b$).
 
 \end{itemize}

 \section{Decidir la cantidad de dinero a apostar}
 \label{sec:cantidad-apostar}

 Supongamos que $\mu_p \ge 1$ y que existen 2 funciones de utilidad:
 \[U_1:\Re^+ \rightarrow \Re^+\]
 \[U_2:\Re^+ \rightarrow \Re^+\]

 La primera es la función de utilidad del dinero para las ganancias y la segunda es la utilidad del dinero para las pérdidas monetarias.\\

 Se harán las siguientes supuestos:

 \begin{enumerate}[(i)]
  \item $U_1(0)=U_2(0)=0$. $U_1$, $U_2$ no decrecientes, una vez cont. dif.
  \item $U'_1(0)>U'_2(0)$ (por lo tanto convendrá apostar).
  \item $\forall M>0$ fija $\displaystyle \lim_{x\rightarrow \infty} \frac{U_1(\mu x)}{U_2(x)}=0$.\\
  (Perder duele muchisimo más que ganar).
 \end{enumerate}
 El problema de decisión asociado a  determinar la cantidad óptima a postar es:(con $0<p<1$ fija y $\mu$ momio)\\

 \begin{figure}[ht]
  \begin{center}
 \begin{tikzpicture}[grow=right, sloped]
 \node[text width=4em, text centered] {$\square$}
 %%%%%%%%%%%%%%%%%%%%%%%%%%%%%%%%%%%%%%%%%%%%%%%
 %primero
 %%%%%%%%%%%%%%%%%%%%%%%%%%%%%%%%%%%%%%%%%%%%%%%%
 child{
     node[text width=4em, text centered] {\textbigcircle}        
             child {
                 node[circle, minimum width=1pt,fill, inner sep=0pt, label=right:
                     {$-U_2(x)$}] {}
                 edge from parent
                 node[above] {$1-p$}             
             }
             child {
                 node[circle, minimum width=1pt,fill, inner sep=0pt, label=right:
                     {$U_1((\mu-1)x)$}] {}
                 edge from parent
                 node[above] {$p$}              
             }    
  edge from parent         
  node[above] {$\delta_x$}
     };
 \end{tikzpicture} 
 \end{center}
 \caption{Árbol de probabilidad 4}
 \end{figure}



 $\rightarrow E_p[U(\delta x)]=pU_1((\mu-1)x)-(1-p)U_2(x)$\\

 Sea $f(x)=E_p[U(\delta x)]$\\

 Encontrar el óptimo es encontrar $x \ge 0$ que resuelva el problema: $\displaystyle \max_{x\ge0}f(x)$\\

 \[f'(x)=p(\mu-1)U'_1((\mu-1)x)-(1-p)U'_2(x)=0\]
 \[\frac{p(\mu-1)}{(1-p)}=\frac{U'_2(x)}{U'_1((\mu-1)x)}\]
 P.d.$$\exists \quad x^* \quad \cdot \ni \cdot \quad \frac{p(\mu-1)}{1-p}=\frac{U'_2(x)}{U'_1(\mu x)}$$

 \begin{enumerate}[(i)]

  \item $f'(0)=p(\mu-1)U'_1(0)-(1-p)U'_2(0)>p(\mu-1)U'_2(0)-(1-p)U'_2(0)$\\
 
  $\,\,\,\quad\quad=U'_2(0)(p\,\mu-1)\ge 0$\\
 
  Con $U_2'(0)\ge 0$ y $p\mu\ge0$\\
  Por tanto $f'(0)>0$
 
  \item $f(0)=0$
  \item $\displaystyle\frac{f(x)}{U_2(x)}=p\displaystyle\frac{U_1((\mu-1)x)}{U_2(x)}-(1-p)$\\
  $\rightarrow \displaystyle \lim_{x\rightarrow\infty}\frac{f(x)}{U_2(x)}=-(1-p)$\\
 
  $\rightarrow \exists \, x\,\,\cdot \ni \cdot \,\, \displaystyle\frac{f(x)}{U_2(x)}=-(1+p)+\varepsilon<0$\\
 
  $\rightarrow \exists\,x\,\,\cdot \ni \cdot \,\,f(x)<0$
  \begin{itemize}
   \item Por $T.V.M.\,\,\,\exists\,\, x'\in(0,x)\,\,\cdot \ni \cdot \,\,xf'(x')=f(x)-f(0)=f(x)<0$\\
   $\rightarrow f'(x')<0$
   \item T.V.I. $\exists\,\, x^*\in(0,x')\,\,\cdot \ni \cdot \,\,f'(x^*)=0$. i.e. $\displaystyle\frac{p(\mu-1)}{1-p}=\displaystyle\frac{U'_2(x)}{U'_1(\mu x)}$\\
  
   Como $f$ es primero creciente y en algún punto decreciente:\\
   $\rightarrow x\,\,\cdot \ni\cdot\,\,f'(x)=0$ es un maximizador.
  \end{itemize}
 \end{enumerate}

 Algunas funciones a considerar:
 \begin{itemize}
  \item $U_{1,\alpha}(x)=x^{\alpha}\qquad\qquad 0<\alpha<1$\\ 
  $U_2(x)=x$\\
 
  Compruébense los supuestos:
  \begin{enumerate}[(i)]
   \item $U_{1,\alpha}(0)=0=U_2(0)$, son crecientes y una vez dif.
   \item $U'_{1,\alpha}(0)=+\infty$, $U'_2(0)=1\qquad{\therefore \,\, U'_{1,\alpha}(0)>U'_2(0)}$
   \item $\forall \,\, \mu>0$\\
  
   $\displaystyle\lim_{x\rightarrow +\infty}\displaystyle\frac{U_{1,\alpha}(\mu x)}{U_2(x)}=\mu^{\alpha}\displaystyle\lim_{x\rightarrow +\infty}\displaystyle\frac{x^{\alpha}}{x}=\mu^{\alpha}\displaystyle\lim_{x\rightarrow +\infty}\displaystyle\frac{1}{x^{1-\alpha}}=0$\\
  
   Para una apuesta con probabilidad $p$ y momio $\mu$ el óptimo se da en:\\
  
   $\displaystyle{\frac{p(\mu-1)}{(1-p)}=\frac{U'_2(x)}{U'_{1,\alpha}((\mu-1)x)}=\frac{1}{\alpha((\mu-1)x)^{\alpha-1}}=\frac{1}{\alpha}(\mu-1)^{1-\alpha}x^{1-\alpha}}$\\\\
  
   $\rightarrow \left(\displaystyle\frac{\alpha p}{(1-p)}\right)(\mu-1)^{\alpha}=x^{1-\alpha}\rightarrow x^*=\left(\displaystyle\frac{\alpha p}{1-p}\right)^{\frac{1}{1-\alpha}}(\mu-1)^{\alpha/1-\alpha}$\\
  \end{enumerate}

  \item $U_{1,\alpha}(x)=x^{\alpha}\qquad\qquad 0<\alpha<1$\\
  $U_{2,\beta}(x)=x^{\beta}\qquad\qquad \beta \le1$\\
 
  Es fácil revisar los supuestos. Para una apuesta con probabilidad $p$ y momio $\mu$ el óptimo se da en:\\
 
  ${\displaystyle\frac{p(\mu-1)}{(1-p)}=\frac{\beta x^{\beta-1}}{\alpha(\mu-1)^{\alpha-1}x^{\alpha-1}}=\frac{\beta}{\alpha}(\mu-1)^{1-\alpha}x^{\beta-\alpha}}$\\
 
  $\rightarrow{\displaystyle\left(\frac{\alpha p}{\beta(1-p)}\right)(\mu-1)^{\alpha}=x^{\beta-\alpha}\rightarrow x^*=\left(\frac{\alpha p}{\beta(1-p)}\right)^{1/\beta-\alpha}(\mu-1)^{\alpha/\beta-\alpha}}$
 
  \item $U_1(x)=\ln (x)$\\
  $U_2(x)=x$\\
 
  Es fácil revisar los supuestos. Para una apuesta con probabilidad $p$ y momio $\mu$ el óptimo se da en:\\
 
  ${\displaystyle \frac{p(\mu-1)}{(1-p)}=\frac{1}{(\frac{1}{(\mu-1) x})}=(\mu-1)x\rightarrow x^*=\frac{p}{1-p}}$\\
 
  \item $U_{1,\alpha}(x)=1-e^{-\alpha x}\qquad\qquad \alpha \ge 1$\\
  $U_2(x)=x$\\
 
  Es fácil revisar los supuestos. Para una apuesta con probabilidad $p$ y momio $\mu$ el óptimo se da en:\\
 
 \[{\displaystyle\frac{p(\mu-1)}{1-p}=\frac{1}{\alpha e^{-\alpha(\mu-1)x}}\,\,\rightarrow \,\,\ln \left(\frac{\alpha p(\mu-1)}{(1-p)}\right)=\alpha(\mu-1)x}\]
 \[\qquad\qquad\qquad\qquad\qquad\qquad\rightarrow\,\, x^*={\displaystyle\frac{1}{\alpha (\mu-1)}\ln \left(\frac{\alpha p(\mu-1)}{(1-p)}\right)}\]
 \end{itemize}

 Otras tres funciones de utilidad a considerar:

 \begin{itemize}
  \item $U_{1,\alpha}(x)=\alpha x \qquad\qquad \alpha \ge 1$\\
  $U_2(x)=e^x-1$\\
 
  $\rightarrow {\displaystyle\frac{p(\mu-1)}{1-p}=\frac{e^x}{\alpha}}$\\
 
  $\rightarrow x^*=\ln \left(\displaystyle\frac{p(\mu-1)}{1-p}\right)+\ln (\alpha)$
 
  \item $U_1(x)=\ln (x)\qquad\qquad \alpha \ge 1$\\
  $U_2(x)=x^{\alpha}$\\
 
  $\rightarrow {\displaystyle\frac{p(\mu-1)}{1-p}=\frac{\alpha x^{\alpha-1}}{\frac{1}{(\mu-1)x}}=\alpha(\mu-1)x^{\alpha}}$\\
 
  $\rightarrow x^*=\left(\displaystyle\frac{p}{\alpha(1-p)}\right)^{1/\alpha}$
 
  \item $U_{1,\alpha}(x)=\tan^{-1}(x)$\\
  $U_{2,\alpha}(x)=\alpha x\qquad\qquad 0<\alpha\le 1$\\
 
  $\rightarrow \displaystyle\frac{p(\mu-1)}{1-p}=\alpha(1+(\mu-1)^2x^2)$\\
 
  $\rightarrow \displaystyle\frac{p\mu-p-\alpha(1-p)}{1-p}=\alpha(\mu-1)^2x^2$\\
 
  $\rightarrow \displaystyle\frac{p\mu-(1-\alpha)p-\alpha}{1-p}=\alpha(\mu-1)^2x^2$\\

  $\rightarrow x^*={\displaystyle\frac{1}{\sqrt{\alpha}(\mu-1)}\left(\frac{p\mu-(1-\alpha)p-\alpha}{1-p}\right)^{1/2}}$\\
 
  equivalentemente:  $x^*={\displaystyle\frac{1}{\mu-1}\left(\frac{p\mu-(1-\alpha)p-\alpha}{1-p}\right)^{1/2}}$\\
 
  Basta probar que $p\mu-(1-\alpha)p-\alpha \ge 0$\\
 
  $p\mu-(1-\alpha)p-\alpha \ge p\mu-(1-\alpha)-\alpha=p\mu-1 >0$\\
 
  $x^*$ está bien definido.
 \end{itemize}
 
 \subsection{Ahorro precaucional}
\label{sec:ahorro-precaucional}
 Supongamos $F_1,...,F_n$ distribuciones y la siguiente sucesión de Variables aleatorias $(x_1^t)_{t=1}^{\infty},..., (x_n^t)_{t=n}^{\infty}$ independientes $x_j^{t} \sim F_j \,\forall\, t\, \in\, \mathbb{N}$.\\

 Sean $\alpha_1,..., \alpha_n\,\in\,\Re^+\,\, \cdot \ni \cdot \,\,\displaystyle \sum_{j=1}^n\alpha_j=1$, definimos:

 \begin{itemize}
  \item $z_1=\displaystyle \sum_{j=1}^n\alpha_jx_j'$
  \item $z_{t+1}=\displaystyle \sum_{j=1}^n\alpha_jx_j^{t+1}+z_t$
 \end{itemize}
 Supongamos que $E[x_j^t]>1\,\,\forall\,t \,\in\,\mathbb{N}\rightarrow E[z_t]=tE[z_1]=t\mu>1$\\

 {\bf Problema:}\\
 Encontrar $y \,\,\cdot \ni\cdot\,\,(1-ty)+yz_t \ge y\,\,\forall\,t\,\in \mathbb{N}$ con probabilidad $(1-\alpha)\times 100\%$. $(y\in[0,1])$.

 $$\rightarrow y z_t\ge (t+1)y-1\quad\rightarrow\quad z_t\ge(t+1)-\frac{1}{y}$$
 $$\qquad\qquad\qquad\qquad\qquad\qquad\qquad\,\,\rightarrow z_t\ge t+k (con\,\,\,k=1-\frac{1}{y})$$

 equivalentemente: Encontrar $k\le 0\,\,\cdot\ni\cdot\,\,z_t\ge t+k\,\,\forall\,t\in\mathbb{N}$ con probabilidad $(1-\alpha)\times 100\%$.\\

 Sol: \\
 Sea $\mu=E[z_1]$, $\sigma^2=Var[z_1]$\\

 \[\rightarrow (1-\alpha)=p(z_t)\ge t+k\,\,\forall\,\in\mathbb{N}\qquad\qquad\qquad\qquad\qquad\qquad\qquad\qquad\quad\quad\]
 \[=p(z_1\ge 1+k)\cdot p(z_2\ge 2+k\,\,|z_1\ge 1+k)\cdots\qquad\quad\]
 \[=p(z_1\ge 1+k)\cdot\displaystyle\prod_{t=1}^{\infty}p(z_{t+1}\ge (t+1)+k\,|z_t\ge t+k)\]\\

 Usando el T.C.L.: $z_t \rightarrow N(t\mu,\,t\sigma^2)\,\,\forall\,t\in\,\mathbb{N}$

 \begin{enumerate}[(i)]
  \item $p(z_1\ge 1+k)=p{\displaystyle\left(\frac{z_1-\mu}{\sigma}\ge\frac{(1+k)-\mu}{\sigma}\right)=1-\Phi\left(\frac{k-(\mu-1)}{\sqrt{t}\sigma}\right)}$\\

  \item $p(z_{t+1}\ge (t+1)+k\,|z_t\ge t+k)=\displaystyle\frac{p(z_{t+1}\ge (t+1)+k,\,z_t\ge t+k)}{p(z_t\ge t+k)}$\\

     \begin{itemize}
      \item $p(z_t\ge t+k)=p{\displaystyle\left(\frac{z_t-t\mu}{\sqrt{t}\sigma}\ge\frac{k-t(\mu-1)}{\sqrt{t}\sigma}\right)}$\\

      $\qquad\qquad\qquad={\displaystyle 1-\Phi\left(\frac{k-t(\mu-1)}{\sqrt{t}\sigma}\right)}$\\

      \item $z_{t+1}=y_t+z_t$ con $y_t \sim (\mu,\sigma^2),\,\,y_t,z_t$ independientes\\ $z_t\sim(t\mu,t\sigma^2)$
     \end{itemize}
 $\rightarrow f(y_t,z_t)\simeq \displaystyle\frac{1}{2\pi\sqrt{t}\sigma^2}\exp\{-\frac{1}{2\sigma^2}[(y_t-\mu)^2+\frac{1}{t}(z_t-t\mu)^2]\}$
 \end{enumerate}

 Sea \[\omega_t=y_t+z_t\qquad\qquad y_t=\omega_t-v_t\]
 %\[\rightarrow\]
 \[v_t=z_t\qquad\qquad z_t=v_t\]

 \[\rightarrow J=\left( {1\atop 0} {-1\atop {1}} \right)\rightarrow |det(J)|=1\]

 \[f(z_{t+1}z_t)=\displaystyle\frac{1}{2\pi\sqrt{t}\sigma^2}\exp\{\displaystyle -\frac{1}{2\sigma^2}[(z_{t+1}-z_t-\mu)^2+\frac{1}{t}(z_t-t\mu)^2]\}\]\\

 $\rightarrow p(z_{t+1})\ge (t+1)+k,\,z_t\ge t+k$\\

 \[={\displaystyle\int_{t+k}^{\infty}\int_{t+1+k}^{\infty}\frac{1}{2\pi\sqrt{t}\sigma^2}\exp\{\displaystyle -\frac{1}{2\sigma^2}[(z_{t+1}-z_t-\mu)^2+\frac{1}{t}(z_t-t\mu)^2]\}}dz_{t+1}dz_t\]\\

 \[{=\displaystyle\frac{1}{2\pi\sqrt{t}\sigma^2}\int_{t+k}^{\infty}\exp\{-\frac{1}{2\sigma^2 t}(z_t-t\mu)^2\}\int_{t+1+k}^{\infty}\frac{1}{\sqrt{2\pi}\sigma}\exp\{\displaystyle -\frac{1}{2\sigma^2}[(z_{t+1}-z_t-\mu)^2\}dz_{t+1}dz_t}\]\\

 \newpage

 \[\footnote{Ver Apéndice A}= {\displaystyle\frac{1}{2\pi\sqrt{t}\sigma^2}\int_{t+k}^{\infty}\exp\{-\frac{1}{2\sigma^2 t}(z_t-t\mu)^2\}\left[1-\Phi\left(\frac{k+t-z_t-(\mu-1)}{\sigma}\right)\right]}\]

  \rule{14cm}{0.1mm}
 \[\overline{z_t}=\frac{1}{t}z_t,\,\,\,d\overline{z_t}=\frac{1}{t}dz_t,\,\,\,(\overline{z_t})_0=1+\frac{k}{t},\,\,\,(\overline{z_t})_1=\infty\]
  \rule{14cm}{0.1mm}

 \[={\displaystyle\frac{\sqrt{t}}{2\pi\sqrt{t}\sigma^2}\int_{1+k/t}^{\infty}\exp\{-\frac{t}{2\sigma^2 }(\overline{z}_t-\mu)^2\}\left[1-\Phi\left(\frac{k+t(1-\overline{z}_t)-(\mu-1)}{\sigma}\right)\right]d\overline{z}_t}\]

 Por tanto:\\

 $p(z_{t+1})\ge (t+1)+k,\,z_t\ge t+k$\\

 \[\simeq {\displaystyle \frac{\sqrt{t}\int_{1+k/t}^{\infty}\exp\{-\frac{t}{2\sigma^2 }(\overline{z}_t-\mu)^2\}\left[1-\Phi\left(\frac{k+t(1-\overline{z}_t)-(\mu-1)}{\sigma}\right)\right]d\overline{z}_t}{\sqrt{2\pi}\sigma\left(1-\Phi\left(\frac{k-t(\mu-1)}{\sqrt{t}\sigma}\right)\right)}}\]

 Para calcular $k$ se resuelve la siguiente ecuación:

 \[\log (1-\alpha)=\log(p|z_1\ge 1+k)+\displaystyle\sum_{t=1}^{\infty}\log(p(z_{t+1})\ge (t+1)+k,\,z_t\ge t+k)\]

 \[y=\frac{1}{1-k}\]

 Se realizó una muestra $y_1,..., y_n$, donde:

 \[y_1=CA(p_i,\mu_i,\sigma_i)\]

 Donde:
 \begin{itemize}
  \item $p_i$: Un valor de probabilidad deseado.
  \item $M_i$: Un valor de $E[z_1]$ dado.
  \item $\delta_i$: Un valorde $Var(z_i)^{1/2}$.
  \item $CA$: La función que se define implícitamente de resolver las ecuaciones para calcular la cantidad de apostar.
 \end{itemize}

 A tales datos se les ajustó el siguiente modelo lineal:

 \[y_i=\beta_0+\beta_1p_i+\beta_2\mu_i+\beta_3\sigma_i+\varepsilon_i\]

 \newpage

 El ajuste es el siguiente:

 \begin{itemize}
  \item $\beta_0=0.2925$
   \item $\beta_1=-0.9975$
  \item $\beta_2=1.3772$
  \item $\beta_3=-1.1127$
 \end{itemize}

 Con $R^2=0.95$.\\

 En adelante, se tomará como aproximación lo siguiente:

 \[CA(p,\mu,\sigma)\simeq0.2925-0.9975p+1.3772\mu-1.1127\sigma\]

 \subsection{Evolución del dinero en el tiempo}
 \label{sec:evolucion-dinero}
 {\bf Problema:} Decidir $p$ de manera óptima.\\

 Sea $x$ la cantidad de ingresos restantes ($o\ge x\ge1$, en porcentaje), y $\mu$, $\sigma$ la media y la desviación estandar de apostar en un periodo dados.\\

 Supongamos $U_1,U_2: \Re^+\rightarrow \Re^+$ funciones de utilidad del dinero. ($U_1$ ganancias, $-U_2$ pérdidas) $\cdot\ni\cdot$ son no decrecientes y una vez continuamente diferenciables. Considerese la siguiente función:\\

 \[f(p;x,\mu,\sigma)=[beneficio]-[costo]\]
 \[f(p;x,\mu,\sigma)=[pU_1(y(p,\mu,\sigma)\mu x)]-[(1-p)U_2(x)]\]\\

 Suponiendo $y(p,\mu,\sigma)=a_0-a_1p+a_2\mu-a_3\sigma$ se obtiene:\\

 $a_i\ge 0,\,\,i=0,...,3$
 \[f=pU_1((a_0-a_1p+a_2\mu-a_3\sigma)\mu x)-(1-p)U_2(x)\]

 El problema es:\\

 $max\,\,f$\\

 Sol:\\

 \[f'(p)=-pU_1'((a_0-a_1p+a_2\mu-a_3\sigma)\mu x)a_1\mu x\]
 \[\qquad\qquad\qquad+U_1((a_0-a_1p+a_2\mu-a_3\sigma)\mu x)+U_2(x)=0\]

 Si $U_1$ es cóncava $\rightarrow$ $p^*$ es un maximizador.\\

  Forma aproximada de obtener $p$ :

  \[y=a_0-a_1p+a_2\mu-a_3\sigma\]

  $\rightarrow$ Sea $b=a_1\mu x$, $p_0$ una aproximación de $p$. Definimos:

  \[\omega=y\mu x,\quad \omega_0=y(p_0,\mu,\sigma)\mu x\]

  \[\rightarrow U_1(\omega)=U_1(\omega_0)+U'_1(\omega_0)(\omega-\omega_0)+O((\omega-\omega_0))\]
  \[=U_1(\omega_0)+bU'_1(\omega_0)(\omega_0)(p-p_0)+O((\omega-\omega_0))\]

  Se puede aproximar $f$ por:

  \[f(p)\simeq p[U_1(\omega_0)+bU'_1(\omega_0)(\omega_0)(p-p_0)]-(1-p)U_2(x)\]

  \[\Rightarrow f'(p)\simeq U_1(\omega_0)-2bU'_1(\omega_0)p+bU'_1(\omega_0)p_0+U_2(x)=0\]

  \[\Rightarrow p\simeq \frac{1}{2bU'_1(\omega_0)}[U_1(\omega_0)+U_2(x)]+\frac{1}{2}p_0\]

 Supongamos $U_1:\Re^+ \rightarrow \Re^+$ dada por $U(\omega)=\omega^\alpha\qquad(0<\alpha\le1)$ y $U_2(x)=\beta x$\\

 Notese que:
 \begin{itemize}
  \item ${\displaystyle\frac{U_1(\omega_0)}{U'_1(\omega_0)}=\frac{\omega_0}{\alpha}}$
  \item ${\displaystyle\frac{\omega_0}{b}=\frac{a_0-a_1p+a_2\mu-a_3\sigma}{a_1}}$
 \end{itemize}

 \[\Rightarrow p\simeq \frac{1}{2\alpha a_1}[a_0+a_2\mu-a_3\sigma+\frac{\beta}{\mu}[(a_0-a_1p_0+a_2\mu-a_3\sigma)\mu x]^{1-\alpha}]+\frac{1}{2}(1-\frac{1}{\alpha})p_0\]

 Supongamos ahora que $f$ es de la siguiente forma:

 \[f(p)=pU_1(y(p,\mu,\sigma)\cdot \mu x)-(1-p)U_2(x)-pU_3(\theta(k\sigma-\mu))\]

 i.e. Hay pérdidas potenciales por el riesgo de la inversión considerar
 \[U_3(\theta(k\sigma-\mu))U_2(\theta(k\sigma-\mu)x)I(\theta(k\sigma-\mu)\ge 0)\]

 $\Rightarrow$ De manera análoga se obtiene:

 \[p\simeq\frac{1}{2bU'(\omega_0)}[U_1(\omega_0)+U_2(x)+U_2(\theta(k\sigma-\mu)xI_{\{m\ge0\}}]+\frac{1}{2}p_0\]

 \newpage
 Si tomamos $U_1(\omega)0\omega^\alpha$, $U_2(x)=\beta x$ \\\\

 $p\simeq \frac{1}{2\alpha a_1}\{(a_0+a_2\mu-a_3\sigma)$
 \[+\frac{\beta_1}{\mu}[1+(\beta_2\sigma-\beta_3 \mu)I_{\{m\ge0\}}][(a_0-a_1p+a_2\mu-a_3\sigma)\mu x]^{1-\alpha}\}+\frac{1}{2}(1-\frac{1}{\alpha})p_0\]

 
\chapter{Formatos de archivos para la ingesta en el Portal de Administrativo}\label{chap:archivos}

Los tipos de archivos que el sistema procesa son archivos de texto plano, en codificación UTF-8 con ó sin BOM, en formato CSV, en el que la separación de valores se logra mediante tabulaciones, ó el caracter ``\textbackslash t'', y cada línea se termina con el caracter de nueva línea de Unix, es decir ``\textbackslash n''; el archivo debe contener todos los campos y estar en el orden indicado.

Con esto en mente, todas las variables presentadas aquí deberán ser escritas en el mismo renglón separadas por comas. Se ordenan en dos columnas únicamente para el ahorro de espacio.

\section{Formato de archivo para Partidos}

\begin{multicols}{2}
	\begin{enumerate}
	    \setlength{\itemsep}{1pt}
	    \setlength{\parskip}{0pt}
	    \setlength{\parsep}{0pt}
		\item Identificador único del equipo local,
		\item Identificador único del equipo visitante,
		\item Marcador local,
		\item Marcador visitante,
		\item Probabilidad local,
		\item Probabilidad empate,
		\item Probabilidad visitante,
		\item Fecha del partido, expresada en segundos desde la época Unix (1 de enero
		de 1970) en UTC.
	\end{enumerate}
\end{multicols}

\section{Formato de archivo para Equipos}
\begin{multicols}{2}
	\begin{enumerate}
	    \setlength{\itemsep}{1pt}
	    \setlength{\parskip}{0pt}
	    \setlength{\parsep}{0pt}
		\item Identificador único del equipo, 
		\item Variable local 1,
		\item Variable local 2,
		\item Variable local 3,
		\item Variable local 4,
		\item Variable local 5,
		\item Variable local 6,
		\item Variable local 7,
		\item Variable de ataque 1-1, 
		\item Variable de ataque 1-2,
		\item Variable de ataque 1-3, 
		\item Variable de ataque 1-4, 
		\item Variable de ataque 1-5, 
		\item Variable de ataque 1-6, 
		\item Variable de ataque 1-1, 
		\item Variable de ataque 1-2, 
		\item Variable de ataque 1-3, 
		\item Variable de ataque 1-4, 
		\item Variable de ataque 1-5, 
		\item Variable de ataque 1-6,
		\item Variable de ataque 2-1,
		\item Variable de ataque 2-2,
		\item Variable de ataque 2-3,
		\item Variable de ataque 2-4,
		\item Variable de ataque 2-5, 
		\item Variable de ataque 2-6, 
		\item Variable de ataque 3-1, 
		\item Variable de ataque 3-2, 
		\item Variable de ataque 3-3, 
		\item Variable de ataque 3-4, 
		\item Variable de ataque 3-5, 
		\item Variable de ataque 3-6, 
		\item Variable de defensa 1-1, 
		\item Variable de defensa 1-2, 
		\item Variable de defensa 1-3, 
		\item Variable de defensa 1-4, 
		\item Variable de defensa 1-5, 
		\item Variable de defensa 1-6, 
		\item Variable de defensa 1-1,
		\item Variable de defensa 1-2, 
		\item Variable de defensa 1-3, 
		\item Variable de defensa 1-4, 
		\item Variable de defensa 1-5, 
		\item Variable de defensa 1-6, 
		\item Variable de defensa 2-1, 
		\item Variable de defensa 2-2, 
		\item Variable de defensa 2-3, 
		\item Variable de defensa 2-4, 
		\item Variable de defensa 2-5, 
		\item Variable de defensa 2-6, 
		\item Variable de defensa 3-1, 
		\item Variable de defensa 3-2, 
		\item Variable de defensa 3-3, 
		\item Variable de defensa 3-4, 
		\item Variable de defensa 3-5, 
		\item Variable de defensa 3-6, 
		\item Variable de posesión
	\end{enumerate}
\end{multicols}

\graphicspath{{/Users/brunomedina/Dropbox/Tesis-Egobets/egobets-notas/resources/marco/}}

\chapter{Ligas europeas de futbol}\label{chap:equipos}
En esta sección se presenta información relevante de cada una de las ligas europeas de futbol. Esta información hace más comprensible los procesos de calificació y eliminación de los equipos. También se da un listado de los equipos participantes de las ligas.


\section{Bundesliga (Alemania)}


Fundada el 28 de Julio de 1962 en la convención anual de la \emph{DFL Deutsche Fußball Liga GmbH}, la primer temporada se jugó en 1963. La liga evolucionó en función de la reunificación de Alemania y la integración de la liga del Este \cite{hesse2003tor} Hoy en día la \emph{Bundesliga} es conocida como una de las ligas con mayor afluencia en sus partidos, en la temporada 2011/12 hubo un promedio de 44,293 espectadores por partido. Se vendieron 18.8 millones de entradas en total.

\begin{figure}[!htb]\centering
   \begin {minipage}{0.4\textwidth}
     \includegraphics[width=\linewidth]{logo-bundesliga}
   \end{minipage}
\end{figure}
\begin{chapquote}{Gary Lineker, \textit{ex-futbolista inglés.}}
	``El fútbol es un deporte que inventaron los ingleses, que lo saben jugar los brasileños y en el que siempre ganan los alemanes.''
\end{chapquote}


La \emph{DFL} se encarga de la operación de las ligas de futbol: \emph{Bundesliga} y \emph{2. Bundesliga}; que son las más importantes de Alemania. Cuenta con treinta y seis clubs de futbol los cuales juegan se dividen en ambas divisiones (Ver~\ref{sec:equipos-ger}) Todos miembros de la Asociación  de la Liga cuentan con una licencia\footnote{Cada temportada todos los clubs deben cumplir los criterios deportivos, legales, administrativos, financieros y de infraestructura del Lizenzierungsordnung (LO) y sus respectivos apéndices} para poder jugar y deben seguir los sistemas de entrenamiento y procedimientos disciplinarios.

Dieciocho equipos juegan en cada división, cada equipo juega una vez de local y otra de visitante contra cada uno de los otros diecisiete equipos de la liga. Esto significa que al ser $n=18$ equipos se tienen $\sum\limits_{i=1}^{n-1} i= \sum\limits_{i=1}^{17} i= 153 $ partidos en una temporada. Al término de estos partidos se calculan los puntos que cada equipo tiene y se hace la tabla de posiciones, los dos peores equipos de la \emph{Bundesliga} son intercambiados con los dos mejores de la \emph{2. Bundesliga}. Mientras que el tercer mejor equipo de la \emph{2. Bundesliga} disputa un partido con el tercer peor equipo de la \emph{Bundesliga} para decidir quien se queda en la primera división. Análogamente, el equipo con más puntos se vuelve el campeón de la liga.

Los puntos de la tabla son dados por las victorias de cada equipo, una victoria suma tres puntos a la tabla; las derrotas o empates no suman nada. Si en la tabla hay equipos con la misma cantidad de puntos, para el desempate se deben consideran criterios como: diferencias de goles, cantidad de goles anotados en la temporada,  diferencia de goles que resulten de los partidos jugados entre ellos y la cantidad de goles como visitantes. Si todos estos criterios no deciden el desempate, se deberá jugar un partido en una cancha neutral para decidir su posición en la tabla.

La regulación de la cantidad de jugadores extranjeros en los equipos sigue la regulación d ela UEFA desde el 21 de Diciembre del 2005. Actualmente hay 977 jugadores con un contrato profesional, 503 en la \emph{Bundesliga} y 474 en la \emph{2. Bundesliga}. El cuarenta y siete por ciento de la primera división son extranjeros (234 jugadores) y el treinta y seis por ciento  en la segunda liga (171 jugadores)

En total, 43 clubs han ganado la Bundesliga desde su fundación. Los tres equipos con más campeonatos son: \emph{FC Bayern Munich} con 23 títulos, \emph{BFC Dynamo Berlin} con 10 y \emph{1. FC Nürnberg} con 9. Los tres máximos goleadores de la liga son: \emph{Ger Müller} (1965-1979) con 365 goles, \emph{Klaus Fischer} (1968-1988) con 268 y \emph{Jupp Heyncke}s con 220. \cite{bundesliga}


\subsection{Equipos Alemanes}
\label{sec:equipos-ger}
\begin{multicols}{2}
	\begin{itemize}
	    \setlength{\itemsep}{1pt}
	    \setlength{\parskip}{0pt}
	    \setlength{\parsep}{0pt}

		\item 1. FC Kaiserslautern

		\item 1. FC Köln GmbH \& Co.KGaA

		\item 1. FC Nürnberg

		\item 1. FSV Mainz 05

		\item Bayer 04 Leverkusen Fußball GmbH

		\item Borussia Dortmund GmbH \& Co. KGaA

		\item Borussia VfL 1900 Mönchengladbach GmbH

		\item DSC Arminia Bielefeld GmbH \& Co. KGaA

		\item Eintracht Frankfurt Fußball AG

		\item FC Augsburg 07

		\item FC Bayern München AG

		\item FC Carl Zeiss Jena e.V.

		\item FC Energie Cottbus

		\item FC Erzgebirge Aue

		\item FC Hansa Rostock

		\item FC Schalke 04

		\item FC St. Pauli

		\item Hamburger SV

		\item Hannover 96 GmbH \& Co. KGaA

		\item Hertha BSC Berlin KGmbH aA

		\item Karlsruher SC

		\item MSV Duisburg GmbH \& Co.KGaA

		\item Offenbacher Fußballclubs Kickers 1901 e.V.

		\item SC Freiburg

		\item SC Paderborn 07 e.V.

		\item SpVgg. Greuther Fürth GmbH \& Co. KG

		\item SV Wehen 1926 Wiesbaden

		\item TSG Hoffenheim

		\item TSV Alemannia Aachen

		\item TSV München von 1860 GmbH \& Co. KGaA

		\item TuS Koblenz 1911 e.V.

		\item VfB Stuttgart 1893 e.V.

		\item VfL Bochum

		\item VfL Osnabrück

		\item VfL Wolfsburg-Fußball GmbH

		\item Werder Bremen GmbH \& Co. KGaA

	\end{itemize}
\end{multicols}


\section{Liga BBVA (España)}

La Primera División de España comenzó a disputarse en la temporada 1928-29, siendo el FC. Barcelona el primer equipo que se proclamó Campeón. Hasta ese momento, el fútbol español se organizaba en torno al Campeonato de España. Las primeras temporadas se disputaron con los primeros campeones y subcampeones del Campeonato de España. Conocida hoy en día como la \emph{Liga BBVA}\footnote{Nombre proveniente del patrocinio del Banco Bilbao Vizcaya Argentaria. Segunda División ahora se conoce como la \emph{Liga Adelante}. Curiosamente la Segunda División solía tener el nombre de \emph{Liga BBVA} } (por motivos de patrocinio, es considerada hoy en día como la liga de más fuerte del mundo y de mayor importancia. \cite{strongest-league}

\begin{figure}[!htb]\centering
   \begin {minipage}{0.5\textwidth}
     \includegraphics[width=\linewidth]{logo-bbva}
   \end{minipage}
\end{figure}

La Liga de Fútbol Profesional (LFP) se fundó el 26 de julio de 1984. Es una asociación deportiva integrada por todas las sociedades anónimas deportivas y clubes de fútbol de Primera y Segunda División que participan en competiciones oficiales profesionales de España. La LFP forma parte de la Real Federación Española de Fútbol pero tiene autonomía jurídica en su organización y funcionamiento. 

En la actualidad, la Liga de Fútbol Profesional está formada por un total de 42 equipos: 20 en Primera División y 22 en Segunda División (Ver~\ref{sec:equipos-esp}). Igual que la liga Alemana, cada equipo juega una vez de local y otra de visitante contra cada uno de los otros diecinueve equipos de la liga. Esto significa que al ser $n=19$ equipos se tienen 190 partidos en una temporada. Con estas 38 jornadas los equipos suman puntos en la tabla de posiciones, los primeros 3 entran a la fase de grupos de la \emph{Liga de Campeones de la UEFA}. Los últimos tres equipos en la tabla de posiciones descienden a la \emph{Liga Adelante}, mientras que los mejoes 2 de la Segunda División suben a Primera. El tercer ascenso a Liga BBVA es el ganador de un mini torneo entre el tercer vs quinto y cuarto vs sexto mejor clasificados.

Cada victoria suma tres puntos al Club vencedor, en caso de empate ambos equipos se llevan un punto. Las reglas de desempate son las siguientes: 
\begin{itemize}

	\item El que tenga una mayor diferencia entre goles a favor y en contra según el resultado de los partidos jugados entre ellos.

	\item El que tenga la mayor diferencia de goles a favor teniendo en cuenta todos los obtenidos y recibidos en el transcurso de la competición.

	\item El club que haya marcado más goles.

\end{itemize}

En caso de que haya tres equipos o más empatados se siguen los siguientes criterios para el desmpate:

\begin{itemize}

	\item La mejor puntuación de la que a cada uno corresponda a tenor de los resultados de los partidos jugados entre sí por los clubes implicados.

	\item La mayor diferencia de goles a favor y en contra, considerando únicamente los partidos jugados entre sí por los clubes implicados.

	\item La mayor diferencia de goles a favor y en contra teniendo en cuenta todos los encuentros del campeonato.

	\item El mayor número de goles a favor teniendo en cuenta todos los encuentros del campeonato.

	\item El club mejor clasificado en función de las regulaciones de fair play.

\end{itemize}

Se inscriben 25 jugadores cada temporada a cada Club, de los que 3 pueden ser ajenos a la Unión Europea. Sin embargo, todos aquellos que se puedan nacionalizar por sus lazos familiares pueden jugar en el equipo sin ocupar una plaza de extracomunitaria.

59 equipos han jugado en esta ligas desde su comienzo. Los únicos 3 que nunca han descendido son: Athletic Club, FC Barcelona y Real Madrid CF. Los campeones máximos son \emph{Real Madrid CF} con 32 títulos, \emph{FC Barcelona} con 22 y \emph{Club Atlético Madrid} con 10. Loa goleadores más prolíficos son: \emph{Telmo Zarra} (1921-2006) con 251 goles, \emph{Lionel Messi} (1987) con 250 y \emph{Hugo Sánchez} (1958) con 234. \cite{primera}


\subsection{Equipos Españoles}\label{sec:equipos-esp}
\begin{multicols}{2}
	\begin{itemize}
	    \setlength{\itemsep}{1pt}
	    \setlength{\parskip}{0pt}
	    \setlength{\parsep}{0pt}
		\item Alavés
		\item Albacete
		\item Alcorcón
		\item Almería
		\item Athletic
		\item Atlético
		\item Celta
		\item Córdoba
		\item Deportivo
		\item Eibar
		\item Elche
		\item Espanyol
		\item FC Barcelona
		\item FC Barcelona B
		\item Getafe
		\item Girona
		\item Granada
		\item Las Palmas
		\item Leganés
		\item Levante
		\item Llagostera
		\item Lugo
		\item Málaga
		\item Mallorca
		\item Mirandés
		\item Numancia
		\item Osasuna
		\item Ponferradina
		\item R. Betis
		\item R. Madrid
		\item R. Sociedad
		\item Racing
		\item Rayo
		\item Recreativo
		\item Sabadell
		\item Sevilla
		\item Sporting
		\item Tenerife
		\item Valencia
		\item Valladolid
		\item Villarreal
		\item Zaragoza
		
	\end{itemize}
\end{multicols}

\section{Ligue 1 (Francia)}

Fundada el 11 de septiembre de 1932 bajo el nombre de \emph{National} que después cambió a \emph{Division 1}

\begin{figure}[!htb]\centering
   \begin {minipage}{0.5\textwidth}
     \includegraphics[width=\linewidth]{logo-ligue1}
   \end{minipage}
\end{figure}

\subsection{Equipos Franceses}\label{sec:equipos-fra}
\begin{multicols}{2}
	\begin{itemize}
	    \setlength{\itemsep}{1pt}
	    \setlength{\parskip}{0pt}
	    \setlength{\parsep}{0pt}
		\item AC Ajaccio
		\item AC Arles Avignon
		\item AJ Auxerre
		\item Amiens SC
		\item Angers SCO
		\item AS Beauvais
		\item AS Cannes
		\item AS Monaco
		\item AS Nancy-Lorraine
		\item AS Red Star 93
		\item AS Saint-Etienne
		\item ASOA Valence
		\item Besançon RC
		\item CA Bastia
		\item Chamois Niortais
		\item Châteauroux
		\item Clermont Foot
		\item CS Louhans-Cuiseaux
		\item CS Sedan
		\item Dijon FCO
		\item EA Guingamp
		\item ES Wasquehal
		\item ESTAC Troyes
		\item Evian TG FC
		\item FC Gueugnon
		\item FC Libourne Saint Seurin
		\item FC Lorient
		\item FC Martigues
		\item FC Metz
		\item FC Mulhouse
		\item FC Nantes
		\item FC Rouen 1899
		\item FC Sète 34
		\item FC Sochaux-Montbéliard
		\item GF38
		\item GFC Ajaccio
		\item Girondins de Bordeaux
		\item Havre AC
		\item Le Mans FC
		\item LOSC Lille
		\item Montpellier Hérault SC
		\item Nîmes Olympique
		\item OGC Nice
		\item Olympique d'Alès
		\item Olympique de Charleville
		\item Olympique de Marseille
		\item Olympique Lyonnais
		\item Paris Saint-Germain
		\item Perpignan FC
		\item RC Lens
		\item RC Strasbourg
		\item SA Epinal
		\item SC Toulon
		\item SM Caen
		\item Stade Brestois 29
		\item Stade Briochin
		\item Stade de Reims
		\item Stade Lavallois
		\item Stade Poitevin
		\item Stade Rennais FC
		\item Toulouse FC
		\item Tours FC
		\item US Boulogne CO
		\item US Créteil-Lusitanos
		\item US Dunkerque
		\item US Orléans
		\item Valenciennes FC
		\item Vannes OC
	\end{itemize}
\end{multicols}


\section{Premier (Inglaterra)}

La \emph{DFL Deutsche Fußball Liga GmbH} se encarga de la operación de la \emph{Bundesliga} y \emph{2. Bundesliga}, las más importantes ligas de futbol de alemania. Cuenta con treinta y seis clubs 

\subsection{Equipos Ingleses}\label{sec:equipos-eng}
\begin{multicols}{2}
	\begin{itemize}
	    \setlength{\itemsep}{1pt}
	    \setlength{\parskip}{0pt}
	    \setlength{\parsep}{0pt}

	\item Arsenal

		\item Aston Villa

		\item Barnsley

		\item Birmingham City

		\item Blackburn Rovers

		\item Blackpool

		\item Bolton Wanderers

		\item Bradford City

		\item Burnley

		\item Cardiff City

		\item Charlton Athletic

		\item Chelsea

		\item Coventry City

		\item Crystal Palace

		\item Derby County

		\item Everton

		\item Fulham

		\item Hull City

		\item Ipswich Town

		\item Leeds United

		\item Leicester City

		\item Liverpool

		\item Manchester City

		\item Manchester United

		\item Middlesbrough

		\item Newcastle United

		\item Norwich City

		\item Nottingham Forest

		\item Oldham Athletic

		\item Portsmouth

		\item Queens Park Rangers

		\item Reading

		\item Sheffield United

		\item Sheffield Wednesday

		\item Southampton
		
	\end{itemize}
\end{multicols}


\section{Serie A (Italia)}

La \emph{DFL Deutsche Fußball Liga GmbH} se encarga de la operación de la \emph{Bundesliga} y \emph{2. Bundesliga}, las más importantes ligas de futbol de alemania. Cuenta con treinta y seis clubs 


\subsection{Equipos Italianos}\label{sec:equipos-ita}
\begin{multicols}{2}
	\begin{itemize}
	    \setlength{\itemsep}{1pt}
	    \setlength{\parskip}{0pt}
	    \setlength{\parsep}{0pt}
		\item AC Milan
		\item AS Roma
		\item Atalanta
		\item Cagliari
		\item Cesena
		\item Chievo Verona
		\item Empoli
		\item Fiorentina
		\item Genoa
		\item Verona
		\item Inter Milan
		\item Juventus
		\item Lazio
		\item Napoli
		\item Palermo
		\item Parma
		\item Sampdoria
		\item Sassuolo
		\item Torino
		\item Udinese
	\end{itemize}
\end{multicols}


\chapter{Bases de datos} 
\label{chap:dbs} 


\section{Dump BD Egobets.com} 

En este anexo se presenta un dump BSON de la base de datos de MongoDB del sistema de Egobets. Cada apartado contiene el ejemplo de un documento de esa colección.



\subsection{Casas de apuestas}
\lstset{language=javascript}
\begin{lstlisting} 
{
	"_id" : ObjectId("4e4953dac67d3e1d5b000000"),
	"nombre" : "Bwin",
	"url" : "http://www.bwin.com"
} 
\end{lstlisting}

\subsection{Administradores}
\begin{lstlisting} 
{
	"_id" : ObjectId("544450209273f5e31edb66b7"),
	"correo" : "bruno@egobets.com",
	"nombre" : "Brus Medina",
	"password" : "9f9c1136c11828b14e6ff2c057e693d6eaeb9485",
	"titulo" : "Arquitecto de Software"
} 
\end{lstlisting}


\subsection{Equipos}
\begin{lstlisting} 
{
	"_id" : ObjectId("4e012952c67d3ebc5e000000"),
	"diff" : -1,
	"indicadores" : {
		"local" : {
			"ataques" : {
				"1" : [
					77,
					74,
					76,
					77,
					64,
					78
				],
				"2" : [
					100,
					100,
					76,
					85,
					100,
					91
				],
				"3" : [
					83,
					100,
					82,
					78,
					84,
					93
				]
			},
			"defensas" : {
				"1" : [
					71,
					62,
					58,
					56,
					48,
					79
				],
				"2" : [
					67,
					68,
					68,
					59,
					64,
					60
				],
				"3" : [
					95,
					100,
					87,
					77,
					100,
					91
				]
			},
			"ataque_total" : 100,
			"defensa_total" : 72,
			"posesion" : 42.55401
		},
		"visitante" : {
			"ataques" : {
				"1" : [
					70,
					67,
					69,
					70,
					58,
					71
				],
				"2" : [
					74,
					74,
					56,
					63,
					74,
					67
				],
				"3" : [
					65,
					78,
					64,
					61,
					66,
					72
				]
			},
			"defensas" : {
				"1" : [
					64,
					56,
					52,
					51,
					43,
					71
				],
				"2" : [
					84,
					85,
					85,
					73,
					79,
					75
				],
				"3" : [
					92,
					97,
					84,
					74,
					97,
					88
				]
			},
			"ataque_total" : 53,
			"defensa_total" : 79,
			"posesion" : "43.75194"
		}
	},
	"liga" : "alemana",
	"nombre" : "Bayern Munich",
	"posicion" : 1,
	"status" : 1
}
\end{lstlisting}


\subsection{Usuario}

\begin{lstlisting} 
{
	"_id" : ObjectId("4e810d6992c6e97059000002"),
	"usuario" : ObjectId("4e680ec592c6e98b2a000001"),
	"jornada" : 3,
	"inicio" : 0,
	"fin" : 0
} 
\end{lstlisting}

\subsection{Mailing}
\begin{lstlisting} 
{
	"_id" : ObjectId("4e0e264292c6e94939000000"),
	"correo" : "rob@surealista.com",
	"desde" : "2011-07-01T19:55:46.648Z",
	"idioma" : "en",
	"invitado" : 1,
	"nombre" : "Roberto Hidalgo"
} 
\end{lstlisting}

\subsection{partidos}
\begin{lstlisting} 
{
	"_id" : ObjectId("4e23a78092c6e90b15000000"),
	"fecha" : "2011-07-24T05:00:00Z",
	"liga" : "alemana",
	"local" : "4e012952c67d3ebc5e00000f",
	"marcl" : "0.93522",
	"marcv" : "1.0974",
	"momios" : {
		"local" : 3.36516,
		"empate" : 3.20177,
		"visitante" : 2.56075
	},
	"probe" : "0.30131",
	"probl" : "0.30991",
	"probv" : "0.38878",
	"visitante" : "4e012952c67d3ebc5e00000d"
}

\end{lstlisting}

\subsection{Resultados}

\begin{lstlisting} 
{
	"_id" : ObjectId("4e810d6992c6e97059000002"),
	"usuario" : ObjectId("4e680ec592c6e98b2a000001"),
	"jornada" : 3,
	"inicio" : 0,
	"fin" : 0
} 
\end{lstlisting}

\subsection{system.indexes}
\begin{lstlisting} 
{
	"v" : 1,
	"name" : "_id_",
	"key" : {
		"_id" : 1
	},
	"ns" : "egobets.usuarios"
}
\end{lstlisting}

\subsection{usuarios}

\begin{lstlisting} 
{
	"_id" : ObjectId("5457db71c0622ba170a37d30"),
	"nombre" : "Manuel Colin",
	"correo" : "machingon15@hotmail.com",
	"idioma" : true,
	"facebook" : "555255439",
	"status" : NumberLong(1),
	"usuarioDesde" : "2014-11-03T19:45:53.241Z",
	"jornadas" : NumberLong(0)
} 
\end{lstlisting}





\include{Apendices/manual-admin}
\chapter{Preguntas frecuentes}\label{chap:faq}

En este apéndice se encuentra el FAQ\footnote{El acrónimo significa: \textbf{Preguntas Frecuentes}. En inglés la abreviación corresponde a: \textbf{``Frequently Asked Questions''}.} del portal público \emph{``\emph{Egobets.com}''}, esta información busca cubrir las dudas que pudieran tener los usuarios con respecto a la operación del portal. Es importante recalcar que también tienen un proposito promocional.

\section{Preguntas del portal en general}
\begin{itemize}
\item \textbf{¿Qué es Egobets.com?}


Egobets es el mejor sistema de recomendación personalizada de apuestas deportivas para fútbol europeo en Internet.

\item \textbf{¿Qué servicios ofrece Egobets.com?}


Algunos de los servicios son los siguientes:
\begin{enumerate}
	\item \emph{Recomendación personalizada de apuestas en fútbol:} Cada persona es diferente y debe ser tratada de forma única, en Egobets se determina el perfil de riesgo de cada usuario mediante una encuesta y se le recomienda apuestas a su medida, de tal forma que pueda obtener ganancias y sentirse cómodo al mismo tiempo.
	\item \emph{Pronósticos:} Para cada partido se proporcionan, entre otros: marcador final más probable, resultado más probable y el nivel de posesión de balón de cada equipo.
	\item \emph{Estadísticas:} A través de éstas se pueden analizar las fortalezas y debilidades de los equipos favoritos del usuario.
\end{enumerate}

\item \textbf{¿Por qué usar el servicio de Egobets.com?}


Algunas de las ventajas son:
\begin{enumerate}
	\item Es el único servicio de apuestas personalizadas del mercado.
	\item Al mostrar los momios de las casas de apuestas más populares del mercado, se presenta en las recomendaciones el rendimiento más grande del mercado.
	\item Los precios por el servicio son los más competitivos y se ofrece un servicio de mayor calidad que cualquier otro portal de asesoría de apuestas.
\end{enumerate}

\end{itemize}



\section{Preguntas del tablero}
\begin{itemize}

\item \textbf{¿Qué es el tablero y para qué sirve?}


El tablero es la página principal para el usuario de \emph{Egobets.com}. En el tablero se presenta la recomendación de la semana:
1.	Barra de ingreso.
2.	Los partidos en que se apostará: el partido, el resultado a apostar, la cantidad de dinero a apostar, el momio ofrecido y la casa de apuestas que ofrece tal momio.
3.	La gráfica de valor esperado.
\item \textbf{¿Por qué se pregunta la cantidad de dinero a apostar?}


Para poder indicar en la recomendación la cantidad de dinero a apostar, si el recuadro se deja en blanco entonces la cantidad de dinero a apostar se presentará como porcentaje.

\item \textbf{¿Qué representa la primera barra?}


Representa la cantidad de dinero total del cliente: cada cuadro es la cantidad de dinero que se recomienda apostar en un partido, al poner el apuntador arriba de un cuadro se iluminará el partido correspondiente. La parte gris representa el dinero que no se apostará en la semana, es decir, la reserva.

\item \textbf{¿Por qué al principio de cada jornada se preguntn la cantidad de dinero antes de apostar y después de apostar?}


En \emph{Egobets.com} se le da seguimiento a cada uno de nuestros clientes. Con esta información se puede monitorear la evolución del ingreso y así dar las recomendaciones de acuerdo al nivel de ganancias o pérdidas. Es importante que esta información sea verdadera para poder brindar el mejor servicio posible.

\item \textbf{¿Qué es la gráfica de últimos resultados que se muestra en el menú de arriba?}


Es una representación gráfica del porcentaje de ganancias que el cliente ha tenido en las últimas semanas. Un click la agranda.

\item \textbf{¿Se pueden hacer recomendaciones de resultados que no sean los más probables?}


Sí, depende del perfil de riesgo y de lo que pague la casa de apuestas en tal partido. En algunas ocasiones es recomendable apostar en contra del favorito si el pago es suficientemente grande. Si se tiene activado el sistema en contra de favoritos (en el menú perfil) o si el perfil es muy agresivo se presentarán muchas recomendaciones de este tipo. 
\end{itemize}

\section{Gráfica de valor esperado}
\begin{itemize}
\item \textbf{¿Qué es la gráfica de valor esperado?}


La gráfica de valor esperado es una herramienta visual que permite ver cuáles son los posibles resultados de la recomendación de la semana. En \emph{Egobets.com} se conoce que todas las apuestas tienen un riesgo y mediante esta gráfica se puede cuantificar: Cada barra representa la probabilidad de que se gane o pierda la cantidad indicada debajo de ella, mientras más grande sea la barra mayores probabilidades hay de que tal resultado ocurra.

\item \textbf{¿Por qué aparecen barras en números negativos?}


La parte negativa de la gráfica representa el riesgo de la apuesta (o probabilidad de perder dinero). Sin embargo, las recomendaciones del sistema están calculadas de tal forma que la parte positiva de la gráfica sea mayor que la parte negativa, es decir, que en el mediano plazo se puedan obtener ganancias aunque haya habido algunas semanas con pérdidas.

\end{itemize}

\section{Preguntas del menú mis equipos}
\begin{itemize}

\item \textbf{¿Qué equipos aparecen en la sección de mis equipos?}


En esta sección aparecen todos los equipos que hayas marcado como favoritos. Para marcar a un equipo como favorito debes acceder al menú de ligas, seleccionar la liga correspondiente al equipo y marcar a tal equipo como favorito haciendo click en la estrella al lado de su nombre.

\item \textbf{¿Qué ventajas tiene marcar un equipo como favorito?}


Poder tener acceso a las estadísticas de ese equipo de manera más rápida.
\end{itemize}

\section{Preguntas del menú de ligas}
\begin{itemize}

\item \textbf{¿Cuáles son las ligas que presenta \emph{Egobets.com}?}


Inglesa, española, francesa, italiana y alemana.

\item \textbf{¿Por qué el orden de la tabla en cada liga no es el orden oficial en la tabla de posiciones?}


En \emph{Egobets.com} se presentan los equipos mediante un power ranking. Con base en las estadísticas de los partidos y dados sus resultados, se pronostica cuál será la tabla de posiciones al finalizar dicha liga.

\item \textbf{¿Por qué las primeras cinco semanas de cada liga la tabla está ordenada de acuerdo a la tabla de posiciones del año pasado?}


Para las primeras cinco semanas no se tiene información suficiente para poder realizar un power ranking, sin embargo, a partir de la sexta semana la información presentada será de acuerdo al power ranking calculado.

\item \textbf{¿Qué información se presenta en la tabla de las ligas?}


En orden de aparición: Una estrella indicando si el equipo está o no marcado como uno de los favoritos, la posición en el power ranking, el nombre del equipo, el índice de ataque general del equipo, el índice de defensa general del equipo y por último el cambio dentro de la tabla de power ranking.

\item \textbf{¿Qué información se ofrece en la sección de cada equipo y cómo interpretarla?}


Se presentan las estadísticas del equipo, del lado izquierdo como local y del lado derecho como visitante:
\begin{enumerate}

\item Índice de ataque general: Con calificación de una a cinco estrellas o de uno a diez (abajo). Representa la capacidad general del equipo para atacar.
\item Índices de medio centro, delanteros y definición: Con una calificación de cero a cien. Representan la capacidad de controlar el medio centro, de atacar a portería y de precisión de los tiros, respectivamente.
\item Índice de defensa general: Con calificación de una a cinco estrellas o de uno a diez (abajo). Representa la capacidad general del equipo para defender.
\item Índices de medio centro, defensas y portero: Con una calificación del cero a cien. Representan la capacidad de defender en el centro, de los defensas y del portero, respectivamente.
\item Al hacer click en las variables mencionadas en 2 o 4 se tienen acceso a la evolución de tales variables desde el minuto 0 hasta el 90 de un partido.
\end{enumerate}

Por último, se presentan los partidos que tiene tal equipo en la semana.

\end{itemize}

\section{Preguntas del menú de partidos}
\begin{itemize}

\item \textbf{¿A qué información tengo acceso a través del menú de partidos de la semana?}


A resultados de la jornada anterior y a pronósticos de la jornada actual:
\begin{enumerate}

	\item \emph{Resultados de la jornada anterior:} Se presenta el partido, el marcador real, el marcador pronosticado y el resultado pronosticado. Esto con el fin de que los usuarios puedan comparar lo pronosticado y lo que en verdad ocurrió.
	\item \emph{Pronósticos de la jornada actual:} Se presenta el partido, el resultado pronosticado (o favorito), el marcador pronosticado y la fecha del encuentro. Además al poner el apuntador encima de un partido se presenta el grado de confiabilidad del pronóstico.
\end{enumerate}

Además para cada tabla se pueden presentar los resultados de una liga en particular al seleccionarla en el recuadro de arriba.

\item \textbf{¿Por qué algunos partidos aparecen de otro color en la tabla?}


Esos son los partidos en los cuáles el sistema te ha recomendado apostar. Para ver la recomendación debes acceder al menú tablero.

\item \textbf{¿Qué es el grado de confiabilidad?}


Es el nivel de certeza que se tiene del pronóstico del resultado del partido (local, empate o visitante), se mide de cero a cinco estrellas. A mayor cantidad de estrellas es más posible que ocurra el resultado pronosticado.

\item \textbf{¿Qué información se presenta en la sección de un partido?}


El pronóstico del partido y las estadísticas de cada equipo:
\begin{enumerate}

\item Pronósticos: Se pronostica el marcador final, la posesión del balón y el ganador del encuentro. Al poner el apuntador sobre el resultado más probable aparece el grado de confiabilidad del pronóstico.
\item Estadísticas: Se presentan las mismas estadísticas que en la sección de cada equipo, del lado derecho las del equipo local y del lado izquierdo las del visitante, para hacerlo fácil de comparar.
\end{enumerate}

\end{itemize}

\section{Preguntas del menú del perfil}
\begin{itemize}

\item \textbf{¿Qué puede hacer el usuario a través del menú del perfil?}


Se pueden realizar las siguientes acciones:
\begin{enumerate}

\item Cambiar el idioma: Inglés o español.
\item Indicar si se desea o no el sistema de apuestas en contra de favoritos: Explicación en las siguientes preguntas.
\item Cambiar las casas de apuestas: El usuario puede indicar en qué casas de apuestas estás inscrito y las recomendaciones se harán considerando los momios que éstas ofrezcan. Si no se está inscrito a alguna de las listadas se recomienda dejar la opción de todas activa.
\item Hacer pagos: Se puede incrementar la cantidad de jornadas de recomendaciones.
\item Cambiar contraseña.
\item Cambiar perfil de riesgo: El usuario puede volver a tomar la encuesta de riesgo para generar recomendaciones a su medida.
\item Conectar con Facebook\footnote{Facebook es la red social (al menos en estos años) más popular. Facebook cuenta con un ``API'' abierto que permite a los desarrolladores consumir sus servicios como: Obtener la información personal del usuario, conocer e invitar a usuarios al servicio, compartir con todos información y muchas otras cosas. Se puede revisar la documentación de Facebook \cite{facebookDocuWeb} para conocer todas las opciones permitidas junto con sus métodos de conexión y consumo.}: El usuario ligar su cuenta de Facebook con la de \emph{Egobets.com}.
\end{enumerate}

Los cambios en preferencias, casas de apuesta o el perfil de riesgo; se verán reflejados hasta un día después.
\end{itemize}

\section{Sistema en contra de favoritos}
\begin{itemize}

\item \textbf{¿Qué es el sistema en contra de favoritos?}


Es un sistema para jugadores que buscan riesgo moderado o alto en el cual se determina en cuáles partidos apostar en contra del equipo favorito.

\item \textbf{¿Cómo funciona?}


Se analiza cada partido por separado mediante un modelo probabilístico que encuentra los equipos favoritos que no son tan fuertes como lo creen las casas de apuesta o la opinión popular.

\item \textbf{¿Qué son las apuestas dobles y por qué se usan en este sistema?}


Una apuesta doble es cuando se apuesta en uno de los siguientes resultados: local-empate, empate-visitante o local-visitante. El sistema en contra de favoritos recomienda apostar en los resultados contrarios al favorito del partido, por ejemplo, si el favorito de un partido es local se le podrá recomendar la apuesta empate-visitante para que la probabilidad de acertar no sea tan baja.

\item \textbf{¿Por qué apostar en contra del favorito?}


Apostar en contra de un equipo favorito paga más que apostar a lo “seguro”. En fútbol no es raro que equipos ordinarios le ganen a equipos extraordinarios y es en esas ocasiones donde se puede ganar mucho dinero.

\end{itemize}

\section{Con respecto a los pagos}
\begin{itemize}

\item \textbf{¿Cómo se pueden comprar más jornadas?}


A través del menú de arriba donde se indican la jornadas restantes o a través del menú de pagos en perfil.

\item \textbf{¿Qué información se presenta en la sección de pagos?}


Se presentan los paquetes disponibles y el historial de pagos realizados por el usuario. ¡Contrata hasta 10 jornadas y recibe el mejor descuento!

\item \textbf{¿Qué formas de pago tiene \emph{Egobets.com}?}


Pago con tarjetas de crédito, con cuenta paypal o por depósito bancario. Para cualquier reclamación con el sistema de pagos favor de contactar a paypal. Para cualquier devolución del dinero favor de contactar a contacto@\emph{Egobets.com}.com indicando el motivo y se te contestará a la mayor brevedad posible.
\end{itemize}

\section{Con respecto a los perfiles de riesgo}
\begin{itemize}

\item \textbf{¿De qué sirve el perfil de riesgo y cómo lo calculan?}


El perfil de riesgo sirve para poder personalizar la asesoría de apuestas y se calcula a través de las respuestas proporcionadas en la encuesta de perfil de riesgo.

\item \textbf{¿Se puede cambiar el perfil de riesgo?}


Lo puedes cambiar tantas veces como desees desde el menú de perfil, sin embargo, el cambio tarda 24 horas en aparecer en el sistema.
\item \textbf{¿Es obligatorio contestar la encuesta de riesgo?}


 Sí pues es la única forma en el que el sistema puede otorgarte un perfil.

\item \textbf{¿Cuál es la diferencia entre los diferentes perfiles de riesgo?}


De forma genérica hay tres perfiles de riesgo:
\begin{enumerate}
	\item Agresivo: Toma riesgos altos para poder obtener la mayor cantidad de ganancias posibles en el corto plazo. 
	\item Conservador: Apuesta a lo más seguro para proteger su dinero lo más posible, busca ganancias al largo plazo.
	\item Moderado: Término medio entre agresivo y conservador.
\end{enumerate}

\textbf{¿Cuál perfil de riesgo conviene más?}


Eso depende de los gustos personales del cliente. Para cualquier perfil de riesgo se harán recomendaciones que le permitan tener la mayor cantidad de ganancias posibles y al mismo tiempo que le hagan sentir cómodo al apostar.
\end{itemize}

\section{Con respecto a la reserva}
\begin{itemize}

\item \textbf{¿Cuánto dinero debería apostar el cliente esta semana?}


En \emph{Egobets.com} se entiende que cada persona es diferente, que cada apuesta es diferente y debe ser analizada de forma individual, por eso se ha desarrollado el sistema de reservas que determina cuánto dinero apostar en la recomendación de la semana.

\item \textbf{¿Qué es la reserva y para qué sirve?}


La reserva es la cantidad de dinero que no se apostará, sirve para poder seguir apostando en semanas posteriores en el caso en que se lleguen a tener pérdidas.

\item \textbf{¿Cómo funciona el sistema de reservas?}


Se toman en cuenta tres factores: la volatilidad de la apuesta, la ganancia esperada de ésta y el nivel de riesgo deseado del cliente. Se combina esta información en un modelo probabilístico que proporciona la cantidad a apostar. Mediante este sistema se busca de proteger al cliente de pérdidas potenciales.

Beneficios:
\begin{enumerate}

	\item Protege su dinero de pérdidas potenciales.
	\item Permite recuperarse con mayor velocidad de semanas con pérdidas.
	\item Permite dar una estructura de fondo de inversión a las apuestas al obtener un sistema de interés compuesto.

\end{enumerate}

Costos:
\begin{enumerate}
	\item Se restringe la cantidad de ganancias a corto plazo.
\end{enumerate}

Es un sistema a largo plazo, no se recomienda a personas que desean incurrir en riesgos elevados en beneficio de la posibilidad de obtener mayores ganancias.

\end{itemize}




\thispagestyle{empty}

\addtocontents{toc}{\vspace{2em}} % Agrega espacio en la toc

%----------------------------------------------------------------------------------------
%	Bibliografía
%----------------------------------------------------------------------------------------

\backmatter
\printbibliography

\end{document}  