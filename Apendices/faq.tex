\chapter{Preguntas frecuentes}\label{chap:faq}

En este apéndice se encuentra el FAQ\footnote{El acrónimo significa: \textbf{Preguntas Frecuentes}. En inglés la abreviación corresponde a: \textbf{``Frequently Asked Questions''}.} del portal público \emph{``\emph{Egobets.com}''}, esta información busca cubrir las dudas que pudieran tener los usuarios con respecto a la operación del portal. Es importante recalcar que también tienen un proposito promocional.

\section{Preguntas del portal en general}
\begin{itemize}
\item \textbf{¿Qué es Egobets.com?}


Egobets es el mejor sistema de recomendación personalizada de apuestas deportivas para fútbol europeo en Internet.

\item \textbf{¿Qué servicios ofrece Egobets.com?}


Algunos de los servicios son los siguientes:
\begin{enumerate}
	\item \emph{Recomendación personalizada de apuestas en fútbol:} Cada persona es diferente y debe ser tratada de forma única, en Egobets se determina el perfil de riesgo de cada usuario mediante una encuesta y se le recomienda apuestas a su medida, de tal forma que pueda obtener ganancias y sentirse cómodo al mismo tiempo.
	\item \emph{Pronósticos:} Para cada partido se proporcionan, entre otros: marcador final más probable, resultado más probable y el nivel de posesión de balón de cada equipo.
	\item \emph{Estadísticas:} A través de éstas se pueden analizar las fortalezas y debilidades de los equipos favoritos del usuario.
\end{enumerate}

\item \textbf{¿Por qué usar el servicio de Egobets.com?}


Algunas de las ventajas son:
\begin{enumerate}
	\item Es el único servicio de apuestas personalizadas del mercado.
	\item Al mostrar los momios de las casas de apuestas más populares del mercado, se presenta en las recomendaciones el rendimiento más grande del mercado.
	\item Los precios por el servicio son los más competitivos y se ofrece un servicio de mayor calidad que cualquier otro portal de asesoría de apuestas.
\end{enumerate}

\end{itemize}



\section{Preguntas del tablero}
\begin{itemize}

\item \textbf{¿Qué es el tablero y para qué sirve?}


El tablero es la página principal para el usuario de \emph{Egobets.com}. En el tablero se presenta la recomendación de la semana:
1.	Barra de ingreso.
2.	Los partidos en que se apostará: el partido, el resultado a apostar, la cantidad de dinero a apostar, el momio ofrecido y la casa de apuestas que ofrece tal momio.
3.	La gráfica de valor esperado.
\item \textbf{¿Por qué se pregunta la cantidad de dinero a apostar?}


Para poder indicar en la recomendación la cantidad de dinero a apostar, si el recuadro se deja en blanco entonces la cantidad de dinero a apostar se presentará como porcentaje.

\item \textbf{¿Qué representa la primera barra?}


Representa la cantidad de dinero total del cliente: cada cuadro es la cantidad de dinero que se recomienda apostar en un partido, al poner el apuntador arriba de un cuadro se iluminará el partido correspondiente. La parte gris representa el dinero que no se apostará en la semana, es decir, la reserva.

\item \textbf{¿Por qué al principio de cada jornada se preguntn la cantidad de dinero antes de apostar y después de apostar?}


En \emph{Egobets.com} se le da seguimiento a cada uno de nuestros clientes. Con esta información se puede monitorear la evolución del ingreso y así dar las recomendaciones de acuerdo al nivel de ganancias o pérdidas. Es importante que esta información sea verdadera para poder brindar el mejor servicio posible.

\item \textbf{¿Qué es la gráfica de últimos resultados que se muestra en el menú de arriba?}


Es una representación gráfica del porcentaje de ganancias que el cliente ha tenido en las últimas semanas. Un click la agranda.

\item \textbf{¿Se pueden hacer recomendaciones de resultados que no sean los más probables?}


Sí, depende del perfil de riesgo y de lo que pague la casa de apuestas en tal partido. En algunas ocasiones es recomendable apostar en contra del favorito si el pago es suficientemente grande. Si se tiene activado el sistema en contra de favoritos (en el menú perfil) o si el perfil es muy agresivo se presentarán muchas recomendaciones de este tipo. 
\end{itemize}

\section{Gráfica de valor esperado}
\begin{itemize}
\item \textbf{¿Qué es la gráfica de valor esperado?}


La gráfica de valor esperado es una herramienta visual que permite ver cuáles son los posibles resultados de la recomendación de la semana. En \emph{Egobets.com} se conoce que todas las apuestas tienen un riesgo y mediante esta gráfica se puede cuantificar: Cada barra representa la probabilidad de que se gane o pierda la cantidad indicada debajo de ella, mientras más grande sea la barra mayores probabilidades hay de que tal resultado ocurra.

\item \textbf{¿Por qué aparecen barras en números negativos?}


La parte negativa de la gráfica representa el riesgo de la apuesta (o probabilidad de perder dinero). Sin embargo, las recomendaciones del sistema están calculadas de tal forma que la parte positiva de la gráfica sea mayor que la parte negativa, es decir, que en el mediano plazo se puedan obtener ganancias aunque haya habido algunas semanas con pérdidas.

\end{itemize}

\section{Preguntas del menú mis equipos}
\begin{itemize}

\item \textbf{¿Qué equipos aparecen en la sección de mis equipos?}


En esta sección aparecen todos los equipos que hayas marcado como favoritos. Para marcar a un equipo como favorito debes acceder al menú de ligas, seleccionar la liga correspondiente al equipo y marcar a tal equipo como favorito haciendo click en la estrella al lado de su nombre.

\item \textbf{¿Qué ventajas tiene marcar un equipo como favorito?}


Poder tener acceso a las estadísticas de ese equipo de manera más rápida.
\end{itemize}

\section{Preguntas del menú de ligas}
\begin{itemize}

\item \textbf{¿Cuáles son las ligas que presenta \emph{Egobets.com}?}


Inglesa, española, francesa, italiana y alemana.

\item \textbf{¿Por qué el orden de la tabla en cada liga no es el orden oficial en la tabla de posiciones?}


En \emph{Egobets.com} se presentan los equipos mediante un power ranking. Con base en las estadísticas de los partidos y dados sus resultados, se pronostica cuál será la tabla de posiciones al finalizar dicha liga.

\item \textbf{¿Por qué las primeras cinco semanas de cada liga la tabla está ordenada de acuerdo a la tabla de posiciones del año pasado?}


Para las primeras cinco semanas no se tiene información suficiente para poder realizar un power ranking, sin embargo, a partir de la sexta semana la información presentada será de acuerdo al power ranking calculado.

\item \textbf{¿Qué información se presenta en la tabla de las ligas?}


En orden de aparición: Una estrella indicando si el equipo está o no marcado como uno de los favoritos, la posición en el power ranking, el nombre del equipo, el índice de ataque general del equipo, el índice de defensa general del equipo y por último el cambio dentro de la tabla de power ranking.

\item \textbf{¿Qué información se ofrece en la sección de cada equipo y cómo interpretarla?}


Se presentan las estadísticas del equipo, del lado izquierdo como local y del lado derecho como visitante:
\begin{enumerate}

\item Índice de ataque general: Con calificación de una a cinco estrellas o de uno a diez (abajo). Representa la capacidad general del equipo para atacar.
\item Índices de medio centro, delanteros y definición: Con una calificación de cero a cien. Representan la capacidad de controlar el medio centro, de atacar a portería y de precisión de los tiros, respectivamente.
\item Índice de defensa general: Con calificación de una a cinco estrellas o de uno a diez (abajo). Representa la capacidad general del equipo para defender.
\item Índices de medio centro, defensas y portero: Con una calificación del cero a cien. Representan la capacidad de defender en el centro, de los defensas y del portero, respectivamente.
\item Al hacer click en las variables mencionadas en 2 o 4 se tienen acceso a la evolución de tales variables desde el minuto 0 hasta el 90 de un partido.
\end{enumerate}

Por último, se presentan los partidos que tiene tal equipo en la semana.

\end{itemize}

\section{Preguntas del menú de partidos}
\begin{itemize}

\item \textbf{¿A qué información tengo acceso a través del menú de partidos de la semana?}


A resultados de la jornada anterior y a pronósticos de la jornada actual:
\begin{enumerate}

	\item \emph{Resultados de la jornada anterior:} Se presenta el partido, el marcador real, el marcador pronosticado y el resultado pronosticado. Esto con el fin de que los usuarios puedan comparar lo pronosticado y lo que en verdad ocurrió.
	\item \emph{Pronósticos de la jornada actual:} Se presenta el partido, el resultado pronosticado (o favorito), el marcador pronosticado y la fecha del encuentro. Además al poner el apuntador encima de un partido se presenta el grado de confiabilidad del pronóstico.
\end{enumerate}

Además para cada tabla se pueden presentar los resultados de una liga en particular al seleccionarla en el recuadro de arriba.

\item \textbf{¿Por qué algunos partidos aparecen de otro color en la tabla?}


Esos son los partidos en los cuáles el sistema te ha recomendado apostar. Para ver la recomendación debes acceder al menú tablero.

\item \textbf{¿Qué es el grado de confiabilidad?}


Es el nivel de certeza que se tiene del pronóstico del resultado del partido (local, empate o visitante), se mide de cero a cinco estrellas. A mayor cantidad de estrellas es más posible que ocurra el resultado pronosticado.

\item \textbf{¿Qué información se presenta en la sección de un partido?}


El pronóstico del partido y las estadísticas de cada equipo:
\begin{enumerate}

\item Pronósticos: Se pronostica el marcador final, la posesión del balón y el ganador del encuentro. Al poner el apuntador sobre el resultado más probable aparece el grado de confiabilidad del pronóstico.
\item Estadísticas: Se presentan las mismas estadísticas que en la sección de cada equipo, del lado derecho las del equipo local y del lado izquierdo las del visitante, para hacerlo fácil de comparar.
\end{enumerate}

\end{itemize}

\section{Preguntas del menú del perfil}
\begin{itemize}

\item \textbf{¿Qué puede hacer el usuario a través del menú del perfil?}


Se pueden realizar las siguientes acciones:
\begin{enumerate}

\item Cambiar el idioma: Inglés o español.
\item Indicar si se desea o no el sistema de apuestas en contra de favoritos: Explicación en las siguientes preguntas.
\item Cambiar las casas de apuestas: El usuario puede indicar en qué casas de apuestas estás inscrito y las recomendaciones se harán considerando los momios que éstas ofrezcan. Si no se está inscrito a alguna de las listadas se recomienda dejar la opción de todas activa.
\item Hacer pagos: Se puede incrementar la cantidad de jornadas de recomendaciones.
\item Cambiar contraseña.
\item Cambiar perfil de riesgo: El usuario puede volver a tomar la encuesta de riesgo para generar recomendaciones a su medida.
\item Conectar con Facebook: El usuario ligar su cuenta de Facebook con la de \emph{Egobets.com}.
\end{enumerate}

Los cambios en preferencias, casas de apuesta o el perfil de riesgo; se verán reflejados hasta un día después.
\end{itemize}

\section{Sistema en contra de favoritos}
\begin{itemize}

\item \textbf{¿Qué es el sistema en contra de favoritos?}


Es un sistema para jugadores que buscan riesgo moderado o alto en el cual se determina en cuáles partidos apostar en contra del equipo favorito.

\item \textbf{¿Cómo funciona?}


Se analiza cada partido por separado mediante un modelo probabilístico que encuentra los equipos favoritos que no son tan fuertes como lo creen las casas de apuesta o la opinión popular.

\item \textbf{¿Qué son las apuestas dobles y por qué se usan en este sistema?}


Una apuesta doble es cuando se apuesta en uno de los siguientes resultados: local-empate, empate-visitante o local-visitante. El sistema en contra de favoritos recomienda apostar en los resultados contrarios al favorito del partido, por ejemplo, si el favorito de un partido es local se le podrá recomendar la apuesta empate-visitante para que la probabilidad de acertar no sea tan baja.

\item \textbf{¿Por qué apostar en contra del favorito?}


Apostar en contra de un equipo favorito paga más que apostar a lo “seguro”. En fútbol no es raro que equipos ordinarios le ganen a equipos extraordinarios y es en esas ocasiones donde se puede ganar mucho dinero.

\end{itemize}

\section{Con respecto a los pagos}
\begin{itemize}

\item \textbf{¿Cómo se pueden comprar más jornadas?}


A través del menú de arriba donde se indican la jornadas restantes o a través del menú de pagos en perfil.

\item \textbf{¿Qué información se presenta en la sección de pagos?}


Se presentan los paquetes disponibles y el historial de pagos realizados por el usuario. ¡Contrata hasta 10 jornadas y recibe el mejor descuento!

\item \textbf{¿Qué formas de pago tiene \emph{Egobets.com}?}


Pago con tarjetas de crédito, con cuenta paypal o por depósito bancario. Para cualquier reclamación con el sistema de pagos favor de contactar a paypal. Para cualquier devolución del dinero favor de contactar a contacto@\emph{Egobets.com}.com indicando el motivo y se te contestará a la mayor brevedad posible.
\end{itemize}

\section{Con respecto a los perfiles de riesgo}
\begin{itemize}

\item \textbf{¿De qué sirve el perfil de riesgo y cómo lo calculan?}


El perfil de riesgo sirve para poder personalizar la asesoría de apuestas y se calcula a través de las respuestas proporcionadas en la encuesta de perfil de riesgo.

\item \textbf{¿Se puede cambiar el perfil de riesgo?}


Lo puedes cambiar tantas veces como desees desde el menú de perfil, sin embargo, el cambio tarda 24 horas en aparecer en el sistema.
\item \textbf{¿Es obligatorio contestar la encuesta de riesgo?}


 Sí pues es la única forma en el que el sistema puede otorgarte un perfil.

\item \textbf{¿Cuál es la diferencia entre los diferentes perfiles de riesgo?}


De forma genérica hay tres perfiles de riesgo:
\begin{enumerate}
	\item Agresivo: Toma riesgos altos para poder obtener la mayor cantidad de ganancias posibles en el corto plazo. 
	\item Conservador: Apuesta a lo más seguro para proteger su dinero lo más posible, busca ganancias al largo plazo.
	\item Moderado: Término medio entre agresivo y conservador.
\end{enumerate}

\textbf{¿Cuál perfil de riesgo conviene más?}


Eso depende de los gustos personales del cliente. Para cualquier perfil de riesgo se harán recomendaciones que le permitan tener la mayor cantidad de ganancias posibles y al mismo tiempo que le hagan sentir cómodo al apostar.
\end{itemize}

\section{Con respecto a la reserva}
\begin{itemize}

\item \textbf{¿Cuánto dinero debería apostar el cliente esta semana?}


En \emph{Egobets.com} se entiende que cada persona es diferente, que cada apuesta es diferente y debe ser analizada de forma individual, por eso se ha desarrollado el sistema de reservas que determina cuánto dinero apostar en la recomendación de la semana.

\item \textbf{¿Qué es la reserva y para qué sirve?}


La reserva es la cantidad de dinero que no se apostará, sirve para poder seguir apostando en semanas posteriores en el caso en que se lleguen a tener pérdidas.

\item \textbf{¿Cómo funciona el sistema de reservas?}


Se toman en cuenta tres factores: la volatilidad de la apuesta, la ganancia esperada de ésta y el nivel de riesgo deseado del cliente. Se combina esta información en un modelo probabilístico que proporciona la cantidad a apostar. Mediante este sistema se busca de proteger al cliente de pérdidas potenciales.

Beneficios:
\begin{enumerate}

	\item Protege su dinero de pérdidas potenciales.
	\item Permite recuperarse con mayor velocidad de semanas con pérdidas.
	\item Permite dar una estructura de fondo de inversión a las apuestas al obtener un sistema de interés compuesto.

\end{enumerate}

Costos:
\begin{enumerate}
	\item Se restringe la cantidad de ganancias a corto plazo.
\end{enumerate}

Es un sistema a largo plazo, no se recomienda a personas que desean incurrir en riesgos elevados en beneficio de la posibilidad de obtener mayores ganancias.

\end{itemize}

