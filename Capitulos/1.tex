\chapter{Introducción}\label{chap:introduccion}

Desde sus orígenes, las apuestas en los partidos de futbol han sido un controversial tema de interés \cite{udovicic1998special}. Predecir los resultados de los partidos y vencer a las casas de apuestas se ha vuelto una fascinación Hollywoodense\footnote{En la película \emph{``Moneyball''} \cite{moneyball} (basado en \cite{lewis2004moneyball}) un equipo de béisbol logra resultados sorprendentes al resolver un problema de optimización con fuertes restricciones monetarias. Mientras que en el filme \emph{``21''} \cite{21Movie}, basado en el libro ``Bringing Down the House'' \cite{patrick2008bringing}, un grupo de estudiantes del MIT (Massachusetts Institute of Technology) utiliza una estrategia de conteo infalible para ganar cientos de miles de dólares en el juego de cartas ``Black Jack''.}. Muchos supuestos ``oráculos'' han utilizado los métodos menos ortodoxos para la predicción de los marcadores \cite{prevos2010psychic}, e incluso se han llevado a cabo acciones fraudulentas para asegurar los marcadores finales de los partidos\footnote{Por ejemplo, en 2006 se suscitó uno de los mayores escándalos en la historia del futbol: \emph{``Calciopoli''}. Se descubrió que varios equipos de la liga italiana conspiraron para influenciar los resultados de los partidos de la temporada 2004/05 \cite{distaso2008corruption}.}. Sin embargo, en la actualidad, las matemáticas y la computación ofrecen un paradigma menos esotérico pero igual de fascinante: la predicción de resultados de partidos de futbol a través de modelos matemáticos.

En este trabajo se describe cómo funciona \textbf{``Egobets''}, una aplicación computacional de las matemáticas al estudio de las apuestas de futbol. Egobets provee asesoría de apuestas personalizadas para partidos de futbol de las siguientes ligas europeas: alemana, española, francesa, inglesa e italiana. Su objetivo es, dado un perfil de riesgo, indicar al usuario la cantidad de dinero y las apuestas que debe realizar para buscar tener ganancias al final de la temporada. Para tal fin, se combinan un conjunto de modelos matemáticos en un sistema robusto computacional.

El sistema Egobets es interesante e innovador ya que no sólo predice el resultado de un partido de futbol, sino que además utiliza la información de todas las ligas europeas para ofrecer una estrategia que maximice la cantidad de dinero a ganar del usuario tomando en cuenta su perfil de riesgo. Adicionalmente, el sistema le sugiere al usuario conservar un porcentaje de su dinero para apostar más agresivamente en caso de perder todas la apuestas de la jornada; garantizando así una mayor cantidad de apuestas durante la temporada y con esto, asegurar una mayor probabilidad de obtener ganancias.

El alcance de este trabajo es el de describir el sistema desarrollado para asesoría de apuestas Egobets. %Gracias a las Matemáticas se pueden desarrollar modelos de fenómenos tan particulares como lo son los partidos de futbol \cite{goddard2005regression}. Además, como se verá en este documento, las Matemáticas proveen las herramientas necesarias para encontrar el conjunto de apuestas a realizar en cada jornada, diversificando el riesgo sobre las apuestas que prometen mayores ganancias. Por otro lado, gracias a los sistemas computacionales y las nuevas tecnologías, se pueden crear las piezas de software de este sistema para ofrecer resultados reales de estas abstracciones matemáticas. Este ecosistema de modelos, aplicaciones y programas funcionan de manera armoniosa presentando resultados al usuario en una interfaz elegante, funcional, simple y fácil de usar. 
Se explicarán los distintos programas y sistemas que conforman el desarrollo, así como las teorías matemáticas que dan sustento al mismo. El documento cuenta con tres capítulos más las conclusiones y esta introducción; el primer capítulo que habla de las apuestas en general, el segundo describe la teoría matemática y el tercero detalla el proyecto realizado.

El primer capítulo, empieza hablando de las ilusiones que mueven a las personas a apostar, después describe como los casinos utilizan los juegos de azar para generar ganancias. En seguida, explica como funcionan los mercados de las apuestas deportivas y sus diferencias con los casinos. Y finalmente, se cierra el capítulo definiendo lo que son los ``momios'' y su papel como regulador en la demanda de la apuesta. 

En este capítulo se presentan, sin entrar al detalle técnico, la existencia de varios modelos matemáticos que describen el comportamiento de los partidos de futbol y como varios autores han logrado encontrar estrategias de apuestas sencillas que garantizan valores esperados positivos de un conjunto de apuestas.
% De igual manera se destaca la importancia de un sistema de reservas para optimizar las tasas de crecimiento de la cantidad de dinero dedicada a las apuestas.
Finalmente se retoman estos conceptos y se detalla el algoritmo que utiliza Egobets para la generación de las recomendaciones de apuestas.

% El segundo capítulo, habla de la existencia de varios modelos matemáticos que describen el comportamiento de los partidos de futbol y como varios autores han logrado encontrar estrategias de apuestas sencillas que garantizan valores esperados positivos de un conjunto de apuestas. Después, se narran los pasos necesarios para el proceso de selección de apuestas y se presenta un ejemplo práctico de como estas teorías generan un conjunto de apuestas apropiado para el perfil de riesgo del usuario.


Egobets.com proporciona al cliente los servicios de asesoría de apuestas personalizada a través de un portal Web usable, práctico y profesional. En el capítulo tercero, se presenta el sistema desarrollado con los fundamentos teóricos descritos en los capítulos anteriores, una verdadera aplicación computacional de las matemáticas. Se define el diseño y la arquitectura del sistema de Egobets en la nube. Además, se habla del conjunto de tecnologías desarrolladas y utilizadas en el sistema. Para terminar, se presentan los diagramas de base de datos y se expone el sistema a través de las tres dimensiones del patrón de diseño Modelo Vista Controlador. 

En el último capítulo, se concluye que se puede llevar un apuesta simple a una estructura de portafolio de inversión. De igual manera se observa que aunque un jugador tuviera en su poder las probabilidades verdaderas de los resultados de los partidos no podría hacer nada con ellas, por lo que es necesario un enfoque de un problema de optimización. Y finalmente que teniendo un sistema metódico que decida las apuestas, remueve la emoción de la apuesta y lo convierte en un riesgo calculado. Para finalizar, se sugieren los distintos campos al que este tipo de sistemas se podría extender: otras ligas, diferentes deportes, elecciones y otros fenómenos parecidos donde se involucre la habilidad humana.
% dan a conocer datos generales de la industria de las apuestas y su participación en la economía mexicana. Después se parte de la descripción de los describen los mercados de apuestas  que ayudan al lector a comprender la relevancia del sistema en la industria de las apuestas y en el ámbito científico; también, se revisan las teorías relacionadas más destacadas y los estudios previos más sobresalientes. Finalmente, se dan a conocer las ligas que se estarán analizando y los motivos por las que fueron elegidas.


% En el tercer capítulo, partiendo de los siguientes dos supuestos: a) un jugador promedio busca maximizar las ganancias de sus apuestas en función de su adversidad al riesgo y, b) apostar siempre conviene más que no apostar. Se plantea una manera de encontrar la apuesta óptima para un partido. Con base en este planteamiento se sigue el análisis a una jornada: ¿A qué partidos de la jornada el usuario le debería apostar? Finalmente, considerando que el usuario busca apostar en todas las jornadas de la temporada, se ataca el problema de la evolución del dinero a apostar durante toda la temporada.
%
%
% En el siguiente apartado del estudio, se habla del conjunto de módulos que conforman el Back Office: sistema de recopilación de información y estadísticas de los partidos, sistema de estimación de probabilidades de los partidos y portal administrativo. Se describe cómo el sistema de recolección de información descarga los datos de las ligas, partidos por jugar y estadísticas de los ya jugados. Se comienza detallando el funcionamiento del sistema recolector de datos, desde la ingestión de los equipos participantes en la temporada vigente, hasta la recolección de los tiros realizados en cada partido por cada jugador. Después, con toda la información obtenida de los desempeños de los equipos en los últimos partidos, se describen las ideas detrás de la predicción de las probabilidades de los resultados de los partidos de futbol. Posteriormente, se exhibe cómo en el portal administrativo se ingresan estas probabilidades junto con los datos de los próximos partidos a jugar. También se detalla cómo este portal, a través de su interfaz gráfica, permite gestionar usuarios, partidos y probabilidades.
%
%
% En la quinta sección de este estudio, se utilizan las teorías y propuestas de los primeros apartados de esta tesis para el diseño y desarrollo del portal público. Se describe cómo este portal también ofrece al usuario revisar y actualizar su perfil, retomar la encuesta de riesgo, revisar los últimos resultados de los partidos, ver la tabla de ``Power Ranking''\footnote{Una tabla que presenta a los equipos del más fuerte al más débil de la temporada.} y obtener la asesoría de apuestas para la jornada en curso.


% En el último capítulo, se concluye que se puede llevar un apuesta simple a una estructura de portafolio de inversión. De igual manera se observa que aunque un jugador tuviera en su poder las probabilidades verdaderas de los resultados de los partidos no podría hacer nada con ellas, por lo que es necesario un enfoque de un problema de optimización. Y finalmente que teniendo un sistema metódico que decida las apuestas, remueve la emoción de la apuesta y lo convierte en un riesgo calculado. Para finalizar, se sugieren los distintos campos al que este tipo de sistemas se podría extender: otras ligas, diferentes deportes, elecciones y otros fenómenos parecidos donde se involucre la habilidad humana.

