\chapter{Tejiendo la teoría Matemática}

\section{Decidir a favor de quien apuestas}

\begin{enumerate}[(a)]
 \item Sean $p_L$, $p_z$, $p_v$ las probabilidades de que gane local, empaten o gane visitante, respectivamente. Sean $\mu_L$, $\mu_z$ y $\mu_v$ los momios respectivos. El problema de decisión de apostar \$\,1 en esta situación es:\\
 
% Set the overall layout of the tree
\tikzstyle{level 1}=[level distance=3.5cm, sibling distance=2.5cm]
\tikzstyle{level 2}=[level distance=3.5cm, sibling distance=2cm]
\tikzstyle{level 3}=[level distance=3.5cm, sibling distance=2cm]


\begin{figure}[ht]
\begin{tikzpicture}[grow=right, sloped]
\node[text width=4em, text centered] {$\square$}
%%%%%%%%%%%%%%%%%%%%%%%%%%%%%%%%%%%%%%%%%%%%%%%%%
%cuarto
%%%%%%%%%%%%%%%%%%%%%%%%%%%%%%%%%%%%%%%%%%%%%%%%
child {       
    node[text width=4em, text centered] {1}
             edge from parent         
 node[above] {$\delta_{NA}$}
    }
%%%%%%%%%%%%%%%%%%%%%%%%%%%%%%%%%%%%%%%%%%%%%%%%%
%tercero
%%%%%%%%%%%%%%%%%%%%%%%%%%%%%%%%%%%%%%%%%%%%%%%%
child {       
    node[text width=4em, text centered] {\textbigcircle}
    child {
                node[circle, minimum width=1pt,fill, inner sep=0pt, label=right:
                    {$0$}] {}
                edge from parent
                node[above] {$1-p_v$}             
            }
            child {
                node[circle, minimum width=1pt,fill, inner sep=0pt, label=right:
                    {$\mu_v$}] {}
                edge from parent
                node[above] {$p_v$}              
            }
             edge from parent         
 node[above] {$\delta_v$}
    }
%%%%%%%%%%%%%%%%%%%%%%%%%%%%%%%%%%%%%%%%%%%%%%%
%segundo
%%%%%%%%%%%%%%%%%%%%%%%%%%%%%%%%%%%%%%%%%%%%%%%%
    child {       
    node[text width=4em, text centered] {\textbigcircle}
    child {
                node[circle, minimum width=1pt,fill, inner sep=0pt, label=right:
                    {$0$}] {}
                edge from parent
                node[above] {$1-p_z$}             
            }
            child {
                node[circle, minimum width=1pt,fill, inner sep=0pt, label=right:
                    {$\mu_z$}] {}
                edge from parent
                node[above] {$p_z$}              
            }
             edge from parent         
 node[above] {$\delta_E$}
    }
%%%%%%%%%%%%%%%%%%%%%%%%%%%%%%%%%%%%%%%%%%%%%%%
%primero
%%%%%%%%%%%%%%%%%%%%%%%%%%%%%%%%%%%%%%%%%%%%%%%%
    child{
    node[text width=4em, text centered] {\textbigcircle}        
            child {
                node[circle, minimum width=1pt,fill, inner sep=0pt, label=right:
                    {$0$}] {}
                edge from parent
                node[above] {$1-p_L$}             
            }
            child {
                node[circle, minimum width=1pt,fill, inner sep=0pt, label=right:
                    {$\mu_L$}] {}
                edge from parent
                node[above] {$p_L$}              
            }    
 edge from parent         
 node[above] {$\delta_L$}
    };  
\end{tikzpicture}
\caption{Decidir por quién apostar}
\end{figure}

  $E_p[U(\delta_i)]=p_i\mu_i;\quad i=L,Z,V$\\
  
  Sol: Se escoge $\rho_i \,\, \cdot \ni \cdot \,\, E_p[U(\delta_i)]=max\{p_L\mu_L,p_z\mu_z,p_v\mu_v,1\}$
  
  \item Se quiere decidir si apostar o no en la ocurrencia de un evento: Sea $p=p(E)$ y $f_p$ densidad de $p$. Sea $\mu$ el momio en el caso de ocurrencia. El problema de decisión asociado es el siguiente:\\
 
\begin{figure}[!ht]
 \begin{tikzpicture}[grow=right, sloped]
\node[text width=4em, text centered] {$\square$}
%%%%%%%%%%%%%%%%%%%%%%%%%%%%%%%%%%%%%%%%%%%%%%%%%
%segundo
%%%%%%%%%%%%%%%%%%%%%%%%%%%%%%%%%%%%%%%%%%%%%%%%
child {       
    node[text width=4em, text centered] {0}
             edge from parent         
 node[above] {$\delta_{NA}$}
    }
%%%%%%%%%%%%%%%%%%%%%%%%%%%%%%%%%%%%%%%%%%%%%%%
%primero
%%%%%%%%%%%%%%%%%%%%%%%%%%%%%%%%%%%%%%%%%%%%%%%%
    child{
    node[text width=4em, text centered]{\textbigcircle}
   	    child {
   	        node[]{\textbigcircle}   	                 
                %node[above] {$f_p$}
                child{node[circle, minimum width=1pt,fill, inner sep=0pt, label=right:
                    {$-1$}] {}                    
                edge from parent
                node[above] {$1-p$}             
            }
            child {
                node[circle, minimum width=1pt,fill, inner sep=0pt, label=right:
                    {$\mu-1$}] {}
                edge from parent
                node[above] {$p$}              
            }
            edge from parent
            node[above] {$f_p$}}    
 edge from parent         
 node[above] {$\delta_A$}
    };
\end{tikzpicture}  
\caption{Decidir si apostar o no apostar}
\end{figure}
 
\newpage
  
  $\rightarrow E_p[U(\delta_A)]=E_{f_p}[p(\mu-1)-(1-p)]\\
  =E_{f_p}[p(\mu)-1]\\
  =E_{f_p}(p)\mu-1$\\

  Apuestas si $E_{f_p}(P)\cdot \mu \ge 1$
  
  \item Mismo problema que el caso anterior, sólo que la utilidad depende de $p$ y $\mu$: $U: \Re\times[0,1]\rightarrow\Re$\\
  $(U(0,p)=0\quad\forall p)$.\\
 
\begin{figure}[ht]
\begin{tikzpicture}[grow=right, sloped]
\node[text width=4em, text centered] {$\square$}
%%%%%%%%%%%%%%%%%%%%%%%%%%%%%%%%%%%%%%%%%%%%%%%%%
%segundo
%%%%%%%%%%%%%%%%%%%%%%%%%%%%%%%%%%%%%%%%%%%%%%%%
child {       
    node[text width=4em, text centered] {0}
             edge from parent         
 node[above] {$\delta_{NA}$}
    }
%%%%%%%%%%%%%%%%%%%%%%%%%%%%%%%%%%%%%%%%%%%%%%%
%primero
%%%%%%%%%%%%%%%%%%%%%%%%%%%%%%%%%%%%%%%%%%%%%%%%
    child{
    node[text width=4em, text centered]{\textbigcircle}
   	    child {
   	        node[]{\textbigcircle}   	                 
                %node[above] {$f_p$}
                child{node[circle, minimum width=1pt,fill, inner sep=0pt, label=right:
                    {$U(-1,p)$}] {}                    
                edge from parent
                node[above] {$1-p$}             
            }
            child {
                node[circle, minimum width=1pt,fill, inner sep=0pt, label=right:
                    {$U(\mu-1,p)$}] {}
                edge from parent
                node[above] {$p$}              
            }
            edge from parent
            node[above] {$f_p$}}    
 edge from parent         
 node[above] {$\delta_A$}
    };
\end{tikzpicture}
\caption{Decidir si apostar en función de una utilidad}
\end{figure}

  
  
  Se apuesta si: \\
  $E_p(U(\delta_A))=E_p[p\,U(\mu-1,p)+(1-p)U(-1,p)]\ge0$
\end{enumerate}

Algunas funciones de utilidad posibles:
\begin{itemize}
 \item $U_\mu(x,p)=x(\frac{1}{\mu}-p)^2$\\
 
 Notese que: $p\,U_\mu(\mu-1,p)+(1-p)U_\mu(-1,p)$\\
 
 $(\hat p=\frac{1}{\mu})=(\hat p-p)^2(p\mu-1)$\\
 
 Me duele más mientras más alejado esté de un trato beneficioso y me produce mayor placer mientras mayor sea el beneficio del trato.
 
 \item $U_{\mu,a}(x,p)= \left\{ \begin{array}{lcc}
             ax(\hat p-p)^2 &   si  & p \le \hat p \\
             & &\\
             x (\hat p-p)^2 &  si & p>\hat p\\             
             \end{array}
 \right.$
 
 Notese que: \\
 \[U_\mu=U_{\mu,1}\]\\
 \[p\,U_{\mu,a}(\mu-1,p)+(1-p)U_\mu(-1,p)= \left\{ \begin{array}{lcc}
             a(\hat p-p)^2 (p\mu-1)&   si  & p \le \hat p \\
             & &\\
             (\hat p-p)^2(p\mu-1) &  si & p>\hat p\\             
             \end{array}
 \right.\]

 Me duele ``a'' veces más un trato perjudicial  que un trato beneficioso si me encuentro a la mis ma distancia que $\hat p$.
 
 \item $U_{\mu,a,b}=U_{\frac{\mu}{1+\mu b},a}$\\
 
 y considerar el problema de decisión con $\mu'=\frac{\mu}{1+\mu b}$.\\
 
 Si $\mu'=\frac{\mu}{1+\mu b}\rightarrow \hat p'=\hat p+b$.\\
 
 Los tratos empiezan a ser beneficiosos hasta que el menos sea $b\%$ más probable que ocurra el evento de lo que sería justo.\\
 
 {\bf Nota:} En un problema de decisión sin aversión a la distribución de probabilidades (o con probabilidades fijas) si se desea apostar en apuestas con un mínimo de ganancias esperadas igual a $b\%$ se debe comparar $\mu_p$ con $1+b$ (i.e. apostar $\leftrightarrow \mu_p \ge 1+b$).
 
\end{itemize}

\section{Decidir la cantidad de dinero a apostar}

Supongamos que $\mu_p \ge 1$ y que existen 2 funciones de utilidad:
\[U_1:\Re^+ \rightarrow \Re^+\]
\[U_2:\Re^+ \rightarrow \Re^+\]

La primera es la función de utilidad del dinero para las ganancias y la segunda es la utilidad del dinero para las pérdidas monetarias.\\

Se harán las siguientes supuestos:

\begin{enumerate}[(i)]
 \item $U_1(0)=U_2(0)=0$. $U_1$, $U_2$ no decrecientes, una vez cont. dif.
 \item $U'_1(0)>U'_2(0)$ (por lo tanto convendrá apostar).
 \item $\forall M>0$ fija $\displaystyle \lim_{x\rightarrow \infty} \frac{U_1(\mu x)}{U_2(x)}=0$.\\
 (Perder duele muchisimo más que ganar).
\end{enumerate}
El problema de decisión asociado a  determinar la cantidad óptima a postar es:(con $0<p<1$ fija y $\mu$ momio)\\

\begin{figure}[ht]
 \begin{center}
\begin{tikzpicture}[grow=right, sloped]
\node[text width=4em, text centered] {$\square$}
%%%%%%%%%%%%%%%%%%%%%%%%%%%%%%%%%%%%%%%%%%%%%%%
%primero
%%%%%%%%%%%%%%%%%%%%%%%%%%%%%%%%%%%%%%%%%%%%%%%%
child{
    node[text width=4em, text centered] {\textbigcircle}        
            child {
                node[circle, minimum width=1pt,fill, inner sep=0pt, label=right:
                    {$-U_2(x)$}] {}
                edge from parent
                node[above] {$1-p$}             
            }
            child {
                node[circle, minimum width=1pt,fill, inner sep=0pt, label=right:
                    {$U_1((\mu-1)x)$}] {}
                edge from parent
                node[above] {$p$}              
            }    
 edge from parent         
 node[above] {$\delta_x$}
    };
\end{tikzpicture} 
\end{center}
\caption{Árbol de probabilidad 4}
\end{figure}



$\rightarrow E_p[U(\delta x)]=pU_1((\mu-1)x)-(1-p)U_2(x)$\\

Sea $f(x)=E_p[U(\delta x)]$\\

Encontrar el óptimo es encontrar $x \ge 0$ que resuelva el problema: $\displaystyle \max_{x\ge0}f(x)$\\

\[f'(x)=p(\mu-1)U'_1((\mu-1)x)-(1-p)U'_2(x)=0\]
\[\frac{p(\mu-1)}{(1-p)}=\frac{U'_2(x)}{U'_1((\mu-1)x)}\]
P.d.$$\exists \quad x^* \quad \cdot \ni \cdot \quad \frac{p(\mu-1)}{1-p}=\frac{U'_2(x)}{U'_1(\mu x)}$$

\begin{enumerate}[(i)]

 \item $f'(0)=p(\mu-1)U'_1(0)-(1-p)U'_2(0)>p(\mu-1)U'_2(0)-(1-p)U'_2(0)$\\
 
 $\,\,\,\quad\quad=U'_2(0)(p\,\mu-1)\ge 0$\\
 
 Con $U_2'(0)\ge 0$ y $p\mu\ge0$\\
 Por tanto $f'(0)>0$
 
 \item $f(0)=0$
 \item $\displaystyle\frac{f(x)}{U_2(x)}=p\displaystyle\frac{U_1((\mu-1)x)}{U_2(x)}-(1-p)$\\
 $\rightarrow \displaystyle \lim_{x\rightarrow\infty}\frac{f(x)}{U_2(x)}=-(1-p)$\\
 
 $\rightarrow \exists \, x\,\,\cdot \ni \cdot \,\, \displaystyle\frac{f(x)}{U_2(x)}=-(1+p)+\varepsilon<0$\\
 
 $\rightarrow \exists\,x\,\,\cdot \ni \cdot \,\,f(x)<0$
 \begin{itemize}
  \item Por $T.V.M.\,\,\,\exists\,\, x'\in(0,x)\,\,\cdot \ni \cdot \,\,xf'(x')=f(x)-f(0)=f(x)<0$\\
  $\rightarrow f'(x')<0$
  \item T.V.I. $\exists\,\, x^*\in(0,x')\,\,\cdot \ni \cdot \,\,f'(x^*)=0$. i.e. $\displaystyle\frac{p(\mu-1)}{1-p}=\displaystyle\frac{U'_2(x)}{U'_1(\mu x)}$\\
  
  Como $f$ es primero creciente y en algún punto decreciente:\\
  $\rightarrow x\,\,\cdot \ni\cdot\,\,f'(x)=0$ es un maximizador.
 \end{itemize}
\end{enumerate}

Algunas funciones a considerar:
\begin{itemize}
 \item $U_{1,\alpha}(x)=x^{\alpha}\qquad\qquad 0<\alpha<1$\\ 
 $U_2(x)=x$\\
 
 Compruébense los supuestos:
 \begin{enumerate}[(i)]
  \item $U_{1,\alpha}(0)=0=U_2(0)$, son crecientes y una vez dif.
  \item $U'_{1,\alpha}(0)=+\infty$, $U'_2(0)=1\qquad{\therefore \,\, U'_{1,\alpha}(0)>U'_2(0)}$
  \item $\forall \,\, \mu>0$\\
  
  $\displaystyle\lim_{x\rightarrow +\infty}\displaystyle\frac{U_{1,\alpha}(\mu x)}{U_2(x)}=\mu^{\alpha}\displaystyle\lim_{x\rightarrow +\infty}\displaystyle\frac{x^{\alpha}}{x}=\mu^{\alpha}\displaystyle\lim_{x\rightarrow +\infty}\displaystyle\frac{1}{x^{1-\alpha}}=0$\\
  
  Para una apuesta con probabilidad $p$ y momio $\mu$ el óptimo se da en:\\
  
  $\displaystyle{\frac{p(\mu-1)}{(1-p)}=\frac{U'_2(x)}{U'_{1,\alpha}((\mu-1)x)}=\frac{1}{\alpha((\mu-1)x)^{\alpha-1}}=\frac{1}{\alpha}(\mu-1)^{1-\alpha}x^{1-\alpha}}$\\\\
  
  $\rightarrow \left(\displaystyle\frac{\alpha p}{(1-p)}\right)(\mu-1)^{\alpha}=x^{1-\alpha}\rightarrow x^*=\left(\displaystyle\frac{\alpha p}{1-p}\right)^{\frac{1}{1-\alpha}}(\mu-1)^{\alpha/1-\alpha}$\\
 \end{enumerate}

 \item $U_{1,\alpha}(x)=x^{\alpha}\qquad\qquad 0<\alpha<1$\\
 $U_{2,\beta}(x)=x^{\beta}\qquad\qquad \beta \le1$\\
 
 Es fácil revisar los supuestos. Para una apuesta con probabilidad $p$ y momio $\mu$ el óptimo se da en:\\
 
 ${\displaystyle\frac{p(\mu-1)}{(1-p)}=\frac{\beta x^{\beta-1}}{\alpha(\mu-1)^{\alpha-1}x^{\alpha-1}}=\frac{\beta}{\alpha}(\mu-1)^{1-\alpha}x^{\beta-\alpha}}$\\
 
 $\rightarrow{\displaystyle\left(\frac{\alpha p}{\beta(1-p)}\right)(\mu-1)^{\alpha}=x^{\beta-\alpha}\rightarrow x^*=\left(\frac{\alpha p}{\beta(1-p)}\right)^{1/\beta-\alpha}(\mu-1)^{\alpha/\beta-\alpha}}$
 
 \item $U_1(x)=\ln (x)$\\
 $U_2(x)=x$\\
 
 Es fácil revisar los supuestos. Para una apuesta con probabilidad $p$ y momio $\mu$ el óptimo se da en:\\
 
 ${\displaystyle \frac{p(\mu-1)}{(1-p)}=\frac{1}{(\frac{1}{(\mu-1) x})}=(\mu-1)x\rightarrow x^*=\frac{p}{1-p}}$\\
 
 \item $U_{1,\alpha}(x)=1-e^{-\alpha x}\qquad\qquad \alpha \ge 1$\\
 $U_2(x)=x$\\
 
 Es fácil revisar los supuestos. Para una apuesta con probabilidad $p$ y momio $\mu$ el óptimo se da en:\\
 
\[{\displaystyle\frac{p(\mu-1)}{1-p}=\frac{1}{\alpha e^{-\alpha(\mu-1)x}}\,\,\rightarrow \,\,\ln \left(\frac{\alpha p(\mu-1)}{(1-p)}\right)=\alpha(\mu-1)x}\]
\[\qquad\qquad\qquad\qquad\qquad\qquad\rightarrow\,\, x^*={\displaystyle\frac{1}{\alpha (\mu-1)}\ln \left(\frac{\alpha p(\mu-1)}{(1-p)}\right)}\]
\end{itemize}

Otras tres funciones de utilidad a considerar:

\begin{itemize}
 \item $U_{1,\alpha}(x)=\alpha x \qquad\qquad \alpha \ge 1$\\
 $U_2(x)=e^x-1$\\
 
 $\rightarrow {\displaystyle\frac{p(\mu-1)}{1-p}=\frac{e^x}{\alpha}}$\\
 
 $\rightarrow x^*=\ln \left(\displaystyle\frac{p(\mu-1)}{1-p}\right)+\ln (\alpha)$
 
 \item $U_1(x)=\ln (x)\qquad\qquad \alpha \ge 1$\\
 $U_2(x)=x^{\alpha}$\\
 
 $\rightarrow {\displaystyle\frac{p(\mu-1)}{1-p}=\frac{\alpha x^{\alpha-1}}{\frac{1}{(\mu-1)x}}=\alpha(\mu-1)x^{\alpha}}$\\
 
 $\rightarrow x^*=\left(\displaystyle\frac{p}{\alpha(1-p)}\right)^{1/\alpha}$
 
 \item $U_{1,\alpha}(x)=\tan^{-1}(x)$\\
 $U_{2,\alpha}(x)=\alpha x\qquad\qquad 0<\alpha\le 1$\\
 
 $\rightarrow \displaystyle\frac{p(\mu-1)}{1-p}=\alpha(1+(\mu-1)^2x^2)$\\
 
 $\rightarrow \displaystyle\frac{p\mu-p-\alpha(1-p)}{1-p}=\alpha(\mu-1)^2x^2$\\
 
 $\rightarrow \displaystyle\frac{p\mu-(1-\alpha)p-\alpha}{1-p}=\alpha(\mu-1)^2x^2$\\

 $\rightarrow x^*={\displaystyle\frac{1}{\sqrt{\alpha}(\mu-1)}\left(\frac{p\mu-(1-\alpha)p-\alpha}{1-p}\right)^{1/2}}$\\
 
 equivalentemente:  $x^*={\displaystyle\frac{1}{\mu-1}\left(\frac{p\mu-(1-\alpha)p-\alpha}{1-p}\right)^{1/2}}$\\
 
 Basta probar que $p\mu-(1-\alpha)p-\alpha \ge 0$\\
 
 $p\mu-(1-\alpha)p-\alpha \ge p\mu-(1-\alpha)-\alpha=p\mu-1 >0$\\
 
 $x^*$ está bien definido.
\end{itemize}

\section{Ahorro precaucional}

Supongamos $F_1,...,F_n$ distribuciones y la siguiente sucesión de Variables aleatorias $(x_1^t)_{t=1}^{\infty},..., (x_n^t)_{t=n}^{\infty}$ independientes $x_j^{t} \sim F_j \,\forall\, t\, \in\, \mathbb{N}$.\\

Sean $\alpha_1,..., \alpha_n\,\in\,\Re^+\,\, \cdot \ni \cdot \,\,\displaystyle \sum_{j=1}^n\alpha_j=1$, definimos:

\begin{itemize}
 \item $z_1=\displaystyle \sum_{j=1}^n\alpha_jx_j'$ 
 \item $z_{t+1}=\displaystyle \sum_{j=1}^n\alpha_jx_j^{t+1}+z_t$ 
\end{itemize}
Supongamos que $E[x_j^t]>1\,\,\forall\,t \,\in\,\mathbb{N}\rightarrow E[z_t]=tE[z_1]=t\mu>1$\\

{\bf Problema:}\\
Encontrar $y \,\,\cdot \ni\cdot\,\,(1-ty)+yz_t \ge y\,\,\forall\,t\,\in \mathbb{N}$ con probabilidad $(1-\alpha)\times 100\%$. $(y\in[0,1])$.

$$\rightarrow y z_t\ge (t+1)y-1\quad\rightarrow\quad z_t\ge(t+1)-\frac{1}{y}$$
$$\qquad\qquad\qquad\qquad\qquad\qquad\qquad\,\,\rightarrow z_t\ge t+k (con\,\,\,k=1-\frac{1}{y})$$

equivalentemente: Encontrar $k\le 0\,\,\cdot\ni\cdot\,\,z_t\ge t+k\,\,\forall\,t\in\mathbb{N}$ con probabilidad $(1-\alpha)\times 100\%$.\\

Sol: \\
Sea $\mu=E[z_1]$, $\sigma^2=Var[z_1]$\\

\[\rightarrow (1-\alpha)=p(z_t)\ge t+k\,\,\forall\,\in\mathbb{N}\qquad\qquad\qquad\qquad\qquad\qquad\qquad\qquad\quad\quad\]
\[=p(z_1\ge 1+k)\cdot p(z_2\ge 2+k\,\,|z_1\ge 1+k)\cdots\qquad\quad\]
\[=p(z_1\ge 1+k)\cdot\displaystyle\prod_{t=1}^{\infty}p(z_{t+1}\ge (t+1)+k\,|z_t\ge t+k)\]\\

Usando el T.C.L.: $z_t \rightarrow N(t\mu,\,t\sigma^2)\,\,\forall\,t\in\,\mathbb{N}$

\begin{enumerate}[(i)]
 \item $p(z_1\ge 1+k)=p{\displaystyle\left(\frac{z_1-\mu}{\sigma}\ge\frac{(1+k)-\mu}{\sigma}\right)=1-\Phi\left(\frac{k-(\mu-1)}{\sqrt{t}\sigma}\right)}$\\
 
 \item $p(z_{t+1}\ge (t+1)+k\,|z_t\ge t+k)=\displaystyle\frac{p(z_{t+1}\ge (t+1)+k,\,z_t\ge t+k)}{p(z_t\ge t+k)}$\\
 
    \begin{itemize}
     \item $p(z_t\ge t+k)=p{\displaystyle\left(\frac{z_t-t\mu}{\sqrt{t}\sigma}\ge\frac{k-t(\mu-1)}{\sqrt{t}\sigma}\right)}$\\
     
     $\qquad\qquad\qquad={\displaystyle 1-\Phi\left(\frac{k-t(\mu-1)}{\sqrt{t}\sigma}\right)}$\\
     
     \item $z_{t+1}=y_t+z_t$ con $y_t \sim (\mu,\sigma^2),\,\,y_t,z_t$ independientes\\ $z_t\sim(t\mu,t\sigma^2)$
    \end{itemize}
$\rightarrow f(y_t,z_t)\simeq \displaystyle\frac{1}{2\pi\sqrt{t}\sigma^2}\exp\{-\frac{1}{2\sigma^2}[(y_t-\mu)^2+\frac{1}{t}(z_t-t\mu)^2]\}$
\end{enumerate}

Sea \[\omega_t=y_t+z_t\qquad\qquad y_t=\omega_t-v_t\]
%\[\rightarrow\]
\[v_t=z_t\qquad\qquad z_t=v_t\]

\[\rightarrow J=\left( {1\atop 0} {-1\atop {1}} \right)\rightarrow |det(J)|=1\]

\[f(z_{t+1}z_t)=\displaystyle\frac{1}{2\pi\sqrt{t}\sigma^2}\exp\{\displaystyle -\frac{1}{2\sigma^2}[(z_{t+1}-z_t-\mu)^2+\frac{1}{t}(z_t-t\mu)^2]\}\]\\

$\rightarrow p(z_{t+1})\ge (t+1)+k,\,z_t\ge t+k$\\

\[={\displaystyle\int_{t+k}^{\infty}\int_{t+1+k}^{\infty}\frac{1}{2\pi\sqrt{t}\sigma^2}\exp\{\displaystyle -\frac{1}{2\sigma^2}[(z_{t+1}-z_t-\mu)^2+\frac{1}{t}(z_t-t\mu)^2]\}}dz_{t+1}dz_t\]\\

\[{=\displaystyle\frac{1}{2\pi\sqrt{t}\sigma^2}\int_{t+k}^{\infty}\exp\{-\frac{1}{2\sigma^2 t}(z_t-t\mu)^2\}\int_{t+1+k}^{\infty}\frac{1}{\sqrt{2\pi}\sigma}\exp\{\displaystyle -\frac{1}{2\sigma^2}[(z_{t+1}-z_t-\mu)^2\}dz_{t+1}dz_t}\]\\

\newpage

\[\footnote{Ver Apéndice A}= {\displaystyle\frac{1}{2\pi\sqrt{t}\sigma^2}\int_{t+k}^{\infty}\exp\{-\frac{1}{2\sigma^2 t}(z_t-t\mu)^2\}\left[1-\Phi\left(\frac{k+t-z_t-(\mu-1)}{\sigma}\right)\right]}\]

 \rule{14cm}{0.1mm}
\[\overline{z_t}=\frac{1}{t}z_t,\,\,\,d\overline{z_t}=\frac{1}{t}dz_t,\,\,\,(\overline{z_t})_0=1+\frac{k}{t},\,\,\,(\overline{z_t})_1=\infty\]
 \rule{14cm}{0.1mm}

\[={\displaystyle\frac{\sqrt{t}}{2\pi\sqrt{t}\sigma^2}\int_{1+k/t}^{\infty}\exp\{-\frac{t}{2\sigma^2 }(\overline{z}_t-\mu)^2\}\left[1-\Phi\left(\frac{k+t(1-\overline{z}_t)-(\mu-1)}{\sigma}\right)\right]d\overline{z}_t}\]

Por tanto:\\

$p(z_{t+1})\ge (t+1)+k,\,z_t\ge t+k$\\

\[\simeq {\displaystyle \frac{\sqrt{t}\int_{1+k/t}^{\infty}\exp\{-\frac{t}{2\sigma^2 }(\overline{z}_t-\mu)^2\}\left[1-\Phi\left(\frac{k+t(1-\overline{z}_t)-(\mu-1)}{\sigma}\right)\right]d\overline{z}_t}{\sqrt{2\pi}\sigma\left(1-\Phi\left(\frac{k-t(\mu-1)}{\sqrt{t}\sigma}\right)\right)}}\]

Para calcular $k$ se resuelve la siguiente ecuación:

\[\log (1-\alpha)=\log(p|z_1\ge 1+k)+\displaystyle\sum_{t=1}^{\infty}\log(p(z_{t+1})\ge (t+1)+k,\,z_t\ge t+k)\]

\[y=\frac{1}{1-k}\]

Se realizó una muestra $y_1,..., y_n$, donde:

\[y_1=CA(p_i,\mu_i,\sigma_i)\]

Donde:
\begin{itemize}
 \item $p_i$: Un valor de probabilidad deseado.
 \item $M_i$: Un valor de $E[z_1]$ dado.
 \item $\delta_i$: Un valorde $Var(z_i)^{1/2}$.
 \item $CA$: La función que se define implícitamente de resolver las ecuaciones para calcular la cantidad de apostar.
\end{itemize}

A tales datos se les ajustó el siguiente modelo lineal:

\[y_i=\beta_0+\beta_1p_i+\beta_2\mu_i+\beta_3\sigma_i+\varepsilon_i\]

\newpage

El ajuste es el siguiente:

\begin{itemize}
 \item $\beta_0=0.2925$
  \item $\beta_1=-0.9975$
 \item $\beta_2=1.3772$
 \item $\beta_3=-1.1127$
\end{itemize}

Con $R^2=0.95$.\\

En adelante, se tomará como aproximación lo siguiente:

\[CA(p,\mu,\sigma)\simeq0.2925-0.9975p+1.3772\mu-1.1127\sigma\]

\section{Evolución de la cantidad a Apostar}

{\bf Problema:} Decidir $p$ de manera óptima.\\

Sea $x$ la cantidad de ingresos restantes ($o\ge x\ge1$, en porcentaje), y $\mu$, $\sigma$ la media y la desviación estandar de apostar en un periodo dados.\\

Supongamos $U_1,U_2: \Re^+\rightarrow \Re^+$ funciones de utilidad del dinero. ($U_1$ ganancias, $-U_2$ pérdidas) $\cdot\ni\cdot$ son no decrecientes y una vez continuamente diferenciables. Considerese la siguiente función:\\

\[f(p;x,\mu,\sigma)=[beneficio]-[costo]\]
\[f(p;x,\mu,\sigma)=[pU_1(y(p,\mu,\sigma)\mu x)]-[(1-p)U_2(x)]\]\\

Suponiendo $y(p,\mu,\sigma)=a_0-a_1p+a_2\mu-a_3\sigma$ se obtiene:\\

$a_i\ge 0,\,\,i=0,...,3$
\[f=pU_1((a_0-a_1p+a_2\mu-a_3\sigma)\mu x)-(1-p)U_2(x)\]

El problema es:\\

$max\,\,f$\\

Sol:\\

\[f'(p)=-pU_1'((a_0-a_1p+a_2\mu-a_3\sigma)\mu x)a_1\mu x\]
\[\qquad\qquad\qquad+U_1((a_0-a_1p+a_2\mu-a_3\sigma)\mu x)+U_2(x)=0\]

Si $U_1$ es cóncava $\rightarrow$ $p^*$ es un maximizador.\\

 Forma aproximada de obtener $p$ :
 
 \[y=a_0-a_1p+a_2\mu-a_3\sigma\]
 
 $\rightarrow$ Sea $b=a_1\mu x$, $p_0$ una aproximación de $p$. Definimos:
 
 \[\omega=y\mu x,\quad \omega_0=y(p_0,\mu,\sigma)\mu x\]
 
 \[\rightarrow U_1(\omega)=U_1(\omega_0)+U'_1(\omega_0)(\omega-\omega_0)+O((\omega-\omega_0))\] 
 \[=U_1(\omega_0)+bU'_1(\omega_0)(\omega_0)(p-p_0)+O((\omega-\omega_0))\]
 
 Se puede aproximar $f$ por:
 
 \[f(p)\simeq p[U_1(\omega_0)+bU'_1(\omega_0)(\omega_0)(p-p_0)]-(1-p)U_2(x)\]
 
 \[\Rightarrow f'(p)\simeq U_1(\omega_0)-2bU'_1(\omega_0)p+bU'_1(\omega_0)p_0+U_2(x)=0\]
 
 \[\Rightarrow p\simeq \frac{1}{2bU'_1(\omega_0)}[U_1(\omega_0)+U_2(x)]+\frac{1}{2}p_0\]
 
Supongamos $U_1:\Re^+ \rightarrow \Re^+$ dada por $U(\omega)=\omega^\alpha\qquad(0<\alpha\le1)$ y $U_2(x)=\beta x$\\

Notese que:
\begin{itemize}
 \item ${\displaystyle\frac{U_1(\omega_0)}{U'_1(\omega_0)}=\frac{\omega_0}{\alpha}}$
 \item ${\displaystyle\frac{\omega_0}{b}=\frac{a_0-a_1p+a_2\mu-a_3\sigma}{a_1}}$
\end{itemize}

\[\Rightarrow p\simeq \frac{1}{2\alpha a_1}[a_0+a_2\mu-a_3\sigma+\frac{\beta}{\mu}[(a_0-a_1p_0+a_2\mu-a_3\sigma)\mu x]^{1-\alpha}]+\frac{1}{2}(1-\frac{1}{\alpha})p_0\]
 
Supongamos ahora que $f$ es de la siguiente forma:

\[f(p)=pU_1(y(p,\mu,\sigma)\cdot \mu x)-(1-p)U_2(x)-pU_3(\theta(k\sigma-\mu))\]
 
i.e. Hay pérdidas potenciales por el riesgo de la inversión considerar
\[U_3(\theta(k\sigma-\mu))U_2(\theta(k\sigma-\mu)x)I(\theta(k\sigma-\mu)\ge 0)\]
 
$\Rightarrow$ De manera análoga se obtiene:

\[p\simeq\frac{1}{2bU'(\omega_0)}[U_1(\omega_0)+U_2(x)+U_2(\theta(k\sigma-\mu)xI_{\{m\ge0\}}]+\frac{1}{2}p_0\]
 
\newpage 
Si tomamos $U_1(\omega)0\omega^\alpha$, $U_2(x)=\beta x$ \\\\
 
$p\simeq \frac{1}{2\alpha a_1}\{(a_0+a_2\mu-a_3\sigma)$
\[+\frac{\beta_1}{\mu}[1+(\beta_2\sigma-\beta_3 \mu)I_{\{m\ge0\}}][(a_0-a_1p+a_2\mu-a_3\sigma)\mu x]^{1-\alpha}\}+\frac{1}{2}(1-\frac{1}{\alpha})p_0\] 
 