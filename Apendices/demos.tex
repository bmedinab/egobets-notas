\chapter{La ruina del jugador}\label{chap:ruina}

% \section{Demostración de la ruina del jugador}
% http://www.columbia.edu/~ks20/stochastic-I/stochastic-I-GRP.pdf
% http://www.ifp.illinois.edu/~sgorant2/gambler.html

\textbf{Planteamiento del problema}
 
Dos apostadores Alicia y Bob participan en el siguiente juego:

Alicia repetidamente lanza una moneda justa al aire. 
\begin{itemize}

	\item Cada que resulta \emph{sol}, Bob le paga a Alicia un peso.

	\item Cada que resulta \emph{águila}, Alicia le paga a Bob un peso.

\end{itemize}

El juego continua hasta que alguno de los dos termina sin dinero.
Si Alicia empieza el juego con una cantidad de dinero $A$ y Bob con $B$, entonces:
\begin{itemize}

	\item ¿Cuál es la probabilidad de que Alicia gane todo el dinero?

	\item ¿Cuál es la duración esperada del juego?

\end{itemize}

\textbf{Solución}

Para esta demostración se usa el Teorema de de Paro Opcional de Doob para martingalas.

Sean $X_1,X_2,\cdots$ los incrementos de la riqueza de Alicia. 

Nótese que $X_i= 1$ depende de si resulta \emph{águila} o \emph{sol}.

Ahora, el cambio en el dinero de Alicia es:
\[S_n = \sum_{i=1}^n X_i\]

Se define $\tau$ como:

\[\tau = \min\{t: S_t = B \mbox{ ó } S_t = -A \}\]

Claramente, $\tau$ es un tiempo de paro relativo a filtración natural

$\mathcal{F}_n = \sigma(X_1,X_2,X_3,\cdots,X_n)$. 

Y $\tau' = \tau \wedge n$ es también un tiempo de paro.\\

\textbf{La probabilidad de que Alicia gane.}

La sucesion $S_n$, es una martingala relativa a la filtración natural $\mathcal{F}_n$.
 
Por lo que, usando el Teorema de Paro Opcional para $n < \infty$,

\[ 0 = {E}[S_0] = {E}[S_{\tau \wedge n}] =\]
\[ (-A) p(\tau \leq n \mbox{ and } S_{\tau} = -A) + (B) p(\tau \leq n \mbox{ and } S_{\tau} = B) + E[S_n \chi_{\tau > n}]\]
                                               
Conforme $n \to \infty$, la probabilidad de que $\tau > n$ converge a cero.

El último término en la ecuación de arriba es la esperanza de una martingala acotada $S_n$ entre $A$ y $B$ y converge a $0$.


Por esto, 
 \[
  0 = (- A) p(S_{\tau} = -A) + (B) p(S_{\tau} = B) 
 \]
 
 Por lo tanto, la probabilidad de que Alicia gane todo el dinero $S_{\tau} = B$ es:
 

 \[
 p(S_{\tau} = B) = \frac{A}{A+B}.
 \]\\

\textbf{La duración esperada del juego}

La sucesión $(S_n^2-n)$ es una martingala relativa a la filtración natural $\mathcal{F}_n$.

Entonces, usando el Teorema de Paro Opcional para $n < \infty$,
\[
 0 = {E}[S^2_0] = {E}[S_{\tau \wedge n}^2 - (\tau \wedge n)]
 \]
                                                                                                           
Cuando $n \to \infty$, la probabilidad de que $\tau > n$ converge a cero. 

Por lo que,
\[
E[\tau] = E[S_{\tau}^2] = (A^2)  p(S_{\tau} = -A) + (B^2)  p(S_{\tau} = B) = AB 
\]         
                      
Por lo tanto, la duración esperada del juego es $AB$



 