\graphicspath{{/Users/brunomedina/Dropbox/Tesis-Egobets/egobets-notas/resources/marco/}}

\chapter{Ligas europeas de futbol}\label{chap:equipos}
En esta sección se presenta información relevante de cada una de las ligas europeas de futbol. Esta información hace más comprensible los procesos de calificació y eliminación de los equipos. También se da un listado de los equipos participantes de las ligas.


\section{Bundesliga (Alemania)}


Fundada el 28 de Julio de 1962 en la convención anual de la \emph{DFL Deutsche Fußball Liga GmbH}, la primer temporada se jugó en 1963. La liga evolucionó en función de la reunificación de Alemania y la integración de la liga del Este \cite{hesse2003tor} Hoy en día la \emph{Bundesliga} es conocida como una de las ligas con mayor afluencia en sus partidos, en la temporada 2011/12 hubo un promedio de 44,293 espectadores por partido. Se vendieron 18.8 millones de entradas en total.

\begin{figure}[!htb]\centering
   \begin {minipage}{0.4\textwidth}
     \includegraphics[width=\linewidth]{logo-bundesliga}
   \end{minipage}
\end{figure}
\begin{chapquote}{Gary Lineker, \textit{ex-futbolista inglés.}}
	``El fútbol es un deporte que inventaron los ingleses, que lo saben jugar los brasileños y en el que siempre ganan los alemanes.''
\end{chapquote}


La \emph{DFL} se encarga de la operación de las ligas de futbol: \emph{Bundesliga} y \emph{2. Bundesliga}; que son las más importantes de Alemania. Cuenta con treinta y seis clubs de futbol los cuales juegan se dividen en ambas divisiones (Ver~\ref{sec:equipos-ger}) Todos miembros de la Asociación  de la Liga cuentan con una licencia\footnote{Cada temportada todos los clubs deben cumplir los criterios deportivos, legales, administrativos, financieros y de infraestructura del Lizenzierungsordnung (LO) y sus respectivos apéndices} para poder jugar y deben seguir los sistemas de entrenamiento y procedimientos disciplinarios.

Dieciocho equipos juegan en cada división, cada equipo juega una vez de local y otra de visitante contra cada uno de los otros diecisiete equipos de la liga. Esto significa que al ser $n=18$ equipos se tienen $\sum\limits_{i=1}^{n-1} i= \sum\limits_{i=1}^{17} i= 153 $ partidos en una temporada. Al término de estos partidos se calculan los puntos que cada equipo tiene y se hace la tabla de posiciones, los dos peores equipos de la \emph{Bundesliga} son intercambiados con los dos mejores de la \emph{2. Bundesliga}. Mientras que el tercer mejor equipo de la \emph{2. Bundesliga} disputa un partido con el tercer peor equipo de la \emph{Bundesliga} para decidir quien se queda en la primera división. Análogamente, el equipo con más puntos se vuelve el campeón de la liga.

Los puntos de la tabla son dados por las victorias de cada equipo, una victoria suma tres puntos a la tabla; las derrotas o empates no suman nada. Si en la tabla hay equipos con la misma cantidad de puntos, para el desempate se deben consideran criterios como: diferencias de goles, cantidad de goles anotados en la temporada,  diferencia de goles que resulten de los partidos jugados entre ellos y la cantidad de goles como visitantes. Si todos estos criterios no deciden el desempate, se deberá jugar un partido en una cancha neutral para decidir su posición en la tabla.

La regulación de la cantidad de jugadores extranjeros en los equipos sigue la regulación d ela UEFA desde el 21 de Diciembre del 2005. Actualmente hay 977 jugadores con un contrato profesional, 503 en la \emph{Bundesliga} y 474 en la \emph{2. Bundesliga}. El cuarenta y siete por ciento de la primera división son extranjeros (234 jugadores) y el treinta y seis por ciento  en la segunda liga (171 jugadores)

En total, 43 clubs han ganado la Bundesliga desde su fundación. Los tres equipos con más campeonatos son: \emph{FC Bayern Munich} con 23 títulos, \emph{BFC Dynamo Berlin} con 10 y \emph{1. FC Nürnberg} con 9. Los tres máximos goleadores de la liga son: \emph{Ger Müller} (1965-1979) con 365 goles, \emph{Klaus Fischer} (1968-1988) con 268 y \emph{Jupp Heyncke}s con 220. \cite{bundesliga}


\subsection{Equipos Alemanes}
\label{sec:equipos-ger}
\begin{multicols}{2}
	\begin{itemize}
	    \setlength{\itemsep}{1pt}
	    \setlength{\parskip}{0pt}
	    \setlength{\parsep}{0pt}

		\item 1. FC Kaiserslautern

		\item 1. FC Köln GmbH \& Co.KGaA

		\item 1. FC Nürnberg

		\item 1. FSV Mainz 05

		\item Bayer 04 Leverkusen Fußball GmbH

		\item Borussia Dortmund GmbH \& Co. KGaA

		\item Borussia VfL 1900 Mönchengladbach GmbH

		\item DSC Arminia Bielefeld GmbH \& Co. KGaA

		\item Eintracht Frankfurt Fußball AG

		\item FC Augsburg 07

		\item FC Bayern München AG

		\item FC Carl Zeiss Jena e.V.

		\item FC Energie Cottbus

		\item FC Erzgebirge Aue

		\item FC Hansa Rostock

		\item FC Schalke 04

		\item FC St. Pauli

		\item Hamburger SV

		\item Hannover 96 GmbH \& Co. KGaA

		\item Hertha BSC Berlin KGmbH aA

		\item Karlsruher SC

		\item MSV Duisburg GmbH \& Co.KGaA

		\item Offenbacher Fußballclubs Kickers 1901 e.V.

		\item SC Freiburg

		\item SC Paderborn 07 e.V.

		\item SpVgg. Greuther Fürth GmbH \& Co. KG

		\item SV Wehen 1926 Wiesbaden

		\item TSG Hoffenheim

		\item TSV Alemannia Aachen

		\item TSV München von 1860 GmbH \& Co. KGaA

		\item TuS Koblenz 1911 e.V.

		\item VfB Stuttgart 1893 e.V.

		\item VfL Bochum

		\item VfL Osnabrück

		\item VfL Wolfsburg-Fußball GmbH

		\item Werder Bremen GmbH \& Co. KGaA

	\end{itemize}
\end{multicols}


\section{Liga BBVA (España)}

La Primera División de España comenzó a disputarse en la temporada 1928-29, siendo el FC. Barcelona el primer equipo que se proclamó Campeón. Hasta ese momento, el fútbol español se organizaba en torno al Campeonato de España. Las primeras temporadas se disputaron con los primeros campeones y subcampeones del Campeonato de España. Conocida hoy en día como la \emph{Liga BBVA}\footnote{Nombre proveniente del patrocinio del Banco Bilbao Vizcaya Argentaria. Segunda División ahora se conoce como la \emph{Liga Adelante}. Curiosamente la Segunda División solía tener el nombre de \emph{Liga BBVA} } (por motivos de patrocinio, es considerada hoy en día como la liga de más fuerte del mundo y de mayor importancia. \cite{strongest-league}

\begin{figure}[!htb]\centering
   \begin {minipage}{0.5\textwidth}
     \includegraphics[width=\linewidth]{logo-bbva}
   \end{minipage}
\end{figure}

La Liga de Fútbol Profesional (LFP) se fundó el 26 de julio de 1984. Es una asociación deportiva integrada por todas las sociedades anónimas deportivas y clubes de fútbol de Primera y Segunda División que participan en competiciones oficiales profesionales de España. La LFP forma parte de la Real Federación Española de Fútbol pero tiene autonomía jurídica en su organización y funcionamiento. 

En la actualidad, la Liga de Fútbol Profesional está formada por un total de 42 equipos: 20 en Primera División y 22 en Segunda División (Ver~\ref{sec:equipos-esp}). Igual que la liga Alemana, cada equipo juega una vez de local y otra de visitante contra cada uno de los otros diecinueve equipos de la liga. Esto significa que al ser $n=19$ equipos se tienen 190 partidos en una temporada. Con estas 38 jornadas los equipos suman puntos en la tabla de posiciones, los primeros 3 entran a la fase de grupos de la \emph{Liga de Campeones de la UEFA}. Los últimos tres equipos en la tabla de posiciones descienden a la \emph{Liga Adelante}, mientras que los mejoes 2 de la Segunda División suben a Primera. El tercer ascenso a Liga BBVA es el ganador de un mini torneo entre el tercer vs quinto y cuarto vs sexto mejor clasificados.

Cada victoria suma tres puntos al Club vencedor, en caso de empate ambos equipos se llevan un punto. Las reglas de desempate son las siguientes: 
\begin{itemize}

	\item El que tenga una mayor diferencia entre goles a favor y en contra según el resultado de los partidos jugados entre ellos.

	\item El que tenga la mayor diferencia de goles a favor teniendo en cuenta todos los obtenidos y recibidos en el transcurso de la competición.

	\item El club que haya marcado más goles.

\end{itemize}

En caso de que haya tres equipos o más empatados se siguen los siguientes criterios para el desmpate:

\begin{itemize}

	\item La mejor puntuación de la que a cada uno corresponda a tenor de los resultados de los partidos jugados entre sí por los clubes implicados.

	\item La mayor diferencia de goles a favor y en contra, considerando únicamente los partidos jugados entre sí por los clubes implicados.

	\item La mayor diferencia de goles a favor y en contra teniendo en cuenta todos los encuentros del campeonato.

	\item El mayor número de goles a favor teniendo en cuenta todos los encuentros del campeonato.

	\item El club mejor clasificado en función de las regulaciones de fair play.

\end{itemize}

Se inscriben 25 jugadores cada temporada a cada Club, de los que 3 pueden ser ajenos a la Unión Europea. Sin embargo, todos aquellos que se puedan nacionalizar por sus lazos familiares pueden jugar en el equipo sin ocupar una plaza de extracomunitaria.

59 equipos han jugado en esta ligas desde su comienzo. Los únicos 3 que nunca han descendido son: Athletic Club, FC Barcelona y Real Madrid CF. Los campeones máximos son \emph{Real Madrid CF} con 32 títulos, \emph{FC Barcelona} con 22 y \emph{Club Atlético Madrid} con 10. Loa goleadores más prolíficos son: \emph{Telmo Zarra} (1921-2006) con 251 goles, \emph{Lionel Messi} (1987) con 250 y \emph{Hugo Sánchez} (1958) con 234. \cite{primera}


\subsection{Equipos Españoles}\label{sec:equipos-esp}
\begin{multicols}{2}
	\begin{itemize}
	    \setlength{\itemsep}{1pt}
	    \setlength{\parskip}{0pt}
	    \setlength{\parsep}{0pt}
		\item Alavés
		\item Albacete
		\item Alcorcón
		\item Almería
		\item Athletic
		\item Atlético
		\item Celta
		\item Córdoba
		\item Deportivo
		\item Eibar
		\item Elche
		\item Espanyol
		\item FC Barcelona
		\item FC Barcelona B
		\item Getafe
		\item Girona
		\item Granada
		\item Las Palmas
		\item Leganés
		\item Levante
		\item Llagostera
		\item Lugo
		\item Málaga
		\item Mallorca
		\item Mirandés
		\item Numancia
		\item Osasuna
		\item Ponferradina
		\item R. Betis
		\item R. Madrid
		\item R. Sociedad
		\item Racing
		\item Rayo
		\item Recreativo
		\item Sabadell
		\item Sevilla
		\item Sporting
		\item Tenerife
		\item Valencia
		\item Valladolid
		\item Villarreal
		\item Zaragoza
		
	\end{itemize}
\end{multicols}

\section{Ligue 1 (Francia)}

Fundada el 11 de septiembre de 1932 bajo el nombre de \emph{National} que después cambió a \emph{Division 1}

\begin{figure}[!htb]\centering
   \begin {minipage}{0.5\textwidth}
     \includegraphics[width=\linewidth]{logo-ligue1}
   \end{minipage}
\end{figure}

\subsection{Equipos Franceses}\label{sec:equipos-fra}
\begin{multicols}{2}
	\begin{itemize}
	    \setlength{\itemsep}{1pt}
	    \setlength{\parskip}{0pt}
	    \setlength{\parsep}{0pt}
		\item AC Ajaccio
		\item AC Arles Avignon
		\item AJ Auxerre
		\item Amiens SC
		\item Angers SCO
		\item AS Beauvais
		\item AS Cannes
		\item AS Monaco
		\item AS Nancy-Lorraine
		\item AS Red Star 93
		\item AS Saint-Etienne
		\item ASOA Valence
		\item Besançon RC
		\item CA Bastia
		\item Chamois Niortais
		\item Châteauroux
		\item Clermont Foot
		\item CS Louhans-Cuiseaux
		\item CS Sedan
		\item Dijon FCO
		\item EA Guingamp
		\item ES Wasquehal
		\item ESTAC Troyes
		\item Evian TG FC
		\item FC Gueugnon
		\item FC Libourne Saint Seurin
		\item FC Lorient
		\item FC Martigues
		\item FC Metz
		\item FC Mulhouse
		\item FC Nantes
		\item FC Rouen 1899
		\item FC Sète 34
		\item FC Sochaux-Montbéliard
		\item GF38
		\item GFC Ajaccio
		\item Girondins de Bordeaux
		\item Havre AC
		\item Le Mans FC
		\item LOSC Lille
		\item Montpellier Hérault SC
		\item Nîmes Olympique
		\item OGC Nice
		\item Olympique d'Alès
		\item Olympique de Charleville
		\item Olympique de Marseille
		\item Olympique Lyonnais
		\item Paris Saint-Germain
		\item Perpignan FC
		\item RC Lens
		\item RC Strasbourg
		\item SA Epinal
		\item SC Toulon
		\item SM Caen
		\item Stade Brestois 29
		\item Stade Briochin
		\item Stade de Reims
		\item Stade Lavallois
		\item Stade Poitevin
		\item Stade Rennais FC
		\item Toulouse FC
		\item Tours FC
		\item US Boulogne CO
		\item US Créteil-Lusitanos
		\item US Dunkerque
		\item US Orléans
		\item Valenciennes FC
		\item Vannes OC
	\end{itemize}
\end{multicols}


\section{Premier (Inglaterra)}

La \emph{DFL Deutsche Fußball Liga GmbH} se encarga de la operación de la \emph{Bundesliga} y \emph{2. Bundesliga}, las más importantes ligas de futbol de alemania. Cuenta con treinta y seis clubs 

\subsection{Equipos Ingleses}\label{sec:equipos-eng}
\begin{multicols}{2}
	\begin{itemize}
	    \setlength{\itemsep}{1pt}
	    \setlength{\parskip}{0pt}
	    \setlength{\parsep}{0pt}

	\item Arsenal

		\item Aston Villa

		\item Barnsley

		\item Birmingham City

		\item Blackburn Rovers

		\item Blackpool

		\item Bolton Wanderers

		\item Bradford City

		\item Burnley

		\item Cardiff City

		\item Charlton Athletic

		\item Chelsea

		\item Coventry City

		\item Crystal Palace

		\item Derby County

		\item Everton

		\item Fulham

		\item Hull City

		\item Ipswich Town

		\item Leeds United

		\item Leicester City

		\item Liverpool

		\item Manchester City

		\item Manchester United

		\item Middlesbrough

		\item Newcastle United

		\item Norwich City

		\item Nottingham Forest

		\item Oldham Athletic

		\item Portsmouth

		\item Queens Park Rangers

		\item Reading

		\item Sheffield United

		\item Sheffield Wednesday

		\item Southampton
		
	\end{itemize}
\end{multicols}


\section{Serie A (Italia)}

La \emph{DFL Deutsche Fußball Liga GmbH} se encarga de la operación de la \emph{Bundesliga} y \emph{2. Bundesliga}, las más importantes ligas de futbol de alemania. Cuenta con treinta y seis clubs 


\subsection{Equipos Italianos}\label{sec:equipos-ita}
\begin{multicols}{2}
	\begin{itemize}
	    \setlength{\itemsep}{1pt}
	    \setlength{\parskip}{0pt}
	    \setlength{\parsep}{0pt}
		\item AC Milan
		\item AS Roma
		\item Atalanta
		\item Cagliari
		\item Cesena
		\item Chievo Verona
		\item Empoli
		\item Fiorentina
		\item Genoa
		\item Verona
		\item Inter Milan
		\item Juventus
		\item Lazio
		\item Napoli
		\item Palermo
		\item Parma
		\item Sampdoria
		\item Sassuolo
		\item Torino
		\item Udinese
	\end{itemize}
\end{multicols}

